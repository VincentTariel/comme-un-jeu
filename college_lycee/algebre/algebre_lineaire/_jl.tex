\usepackage{t1enc}
\usepackage[utf8]{inputenc}
\usepackage[french]{minitoc}
 \usepackage{amsmath}
\usepackage{fancyhdr,amsmath,amsthm,amssymb,fancybox}
\usepackage[francais]{babel}
\usepackage{amsmath,thmtools}
\usepackage{TikZ}
\usepackage{tkz-fct}   
\usepackage{a4wide} 
\usepackage{graphicx}
\usepackage{thmbox}
\usepackage{changepage}
\usepackage{xcolor}
\usepackage{sectsty}
 \usepackage{enumitem}
\usepackage{eurosym}
\usepackage{mathrsfs}
\usepackage{mathtools}

\usetikzlibrary{matrix,arrows,decorations.pathmorphing}
\usepackage[framemethod=TikZ]{mdframed}
\usepackage{geometry}
\usepackage{pgfplots}
\usepackage[most]{tcolorbox}

\geometry{hmargin=2cm,vmargin=1.5cm}
\pagestyle{fancy}

\usepackage{titlesec}

\titleformat
{\chapter} % command
[display] % shape
{\bfseries\huge} % format
{Story No. \ \thechapter} % label
{-5ex} % sep
{
    \rule{\textwidth}{1pt}
    \vspace{1ex}
    \centering
} % before-code
[
\vspace{-2ex}%
\rule{\textwidth}{0.3pt}
]
\titlespacing*{\chapter}{0pt}{-50pt}{10pt}
\renewcommand{\chaptermark}[1]{\markboth{#1}{}}
\renewcommand{\sectionmark}[1]{\markright{\thesection\ #1}}
\fancyhf{}
\fancyhead[LE,RO]{\bfseries\thepage}
\fancyhead[LO]{\bfseries\rightmark}
\fancyhead[RE]{\bfseries\leftmark}
\fancypagestyle{plain}{%
   \fancyhead{} % get rid of headers
   \renewcommand{\headrulewidth}{0pt} % and the line
}



\renewcommand{\thesection}{\Roman{section}} 
\renewcommand{\thesubsection}{\Alph{subsection}}



\definecolor{colorprop}{rgb}{0.0, 0.4, 0.74}
\colorlet{colordef}{red!60}

\sectionfont{\color{black}}
\subsectionfont{\color{black}}
\subsubsectionfont{\color{black}}


\newcommand{\definebox}[3]{%
  \newcounter{#1}
  \newenvironment{#1}[1][]{%
    \stepcounter{#1}%
    \mdfsetup{%
        frametitle={%
            \tikz[baseline=(current bounding box.east),outer sep=0pt]
            \node[anchor=east,rectangle,fill=white]
            {\strut \color{#3}\MakeUppercase#2~\csname \endcsname\ifstrempty{##1}{}{:~##1}};}}%
    \mdfsetup{innertopmargin=10pt,linecolor=#3,%
        linewidth=1pt,topline=true,roundcorner=5pt,
        frametitleaboveskip=\dimexpr-\ht\strutbox\relax,}%
    \begin{mdframed}[]\ignorespaces\relax
    }{\end{mdframed}}%
    \ignorespaces
}
\newcommand{\defineboxx}[3]{%
  \newcounter{#1}
  \newenvironment{#1}[1][]{%
    \stepcounter{#1}%
    \mdfsetup{%
        frametitle={%
            \hspace*{-40pt}\tikz[baseline=(current bounding box.east),outer sep=0pt]
            \node[anchor=east,rectangle,fill=white]
            {\strut \MakeUppercase#2~\csname \endcsname\ifstrempty{##1}{}{:~##1}};}}%
    \mdfsetup{innertopmargin=0pt,linecolor=#3,%
        linewidth=1pt,topline=false,innerleftmargin=20pt,bottomline=false,rightline=false,
        frametitleaboveskip=\dimexpr-\ht\strutbox\relax,}%
    \begin{mdframed}[]\relax%
    }{\end{mdframed}}%
}
\newcommand{\defineboxxx}[3]{%
  \newcounter{#1}
  \newenvironment{#1}[1][]{%
    \stepcounter{#1}%
    \mdfsetup{%
        frametitle={%
            \hspace*{-40pt}\tikz[baseline=(current bounding box.east),outer sep=0pt]
            \node[anchor=east,rectangle,fill=white]
            {\strut \MakeUppercase#2~\csname \endcsname\ifstrempty{##1}{}{:~##1}};}}%
    \mdfsetup{innertopmargin=0pt,linecolor=#3,%
        linewidth=1pt,topline=false,innerleftmargin=20pt,bottomline=false,rightline=false,leftline=false,
        frametitleaboveskip=\dimexpr-\ht\strutbox\relax,}%
    \begin{mdframed}[]\relax%
    }{\end{mdframed}}%
}

\defineboxx{Proof}{Démonstration}{colorprop}
\defineboxx{Ex}{Exemple}{black}
\defineboxx{Activite}{Activité}{black}
\defineboxxx{NB}{Remarque}{black}
\usepackage[colorlinks = true,urlcolor  = colorprop]{hyperref}
\newenvironment{Text}{}{}
\newenvironment{Correction}{\textbf{\newline Correction } }{}
   
\newcommand{\defi}[1]{\textbf{\textcolor{colordef}{#1}}} 
\newcommand{\propri}[1]{\textbf{\textcolor{colorprop}{#1}}}
 \newcommand{\impo}[1]{\textbf{\textcolor{colorprop}{#1}}}

\newtcolorbox{Df}[1][]{
                lower separated=false,
                colback=white,
colframe=colordef,fonttitle=,
colbacktitle=white,
coltitle=colordef,
enhanced,
attach boxed title to top left={yshift=-0.1in,xshift=0.15in},
                 boxed title style={boxrule=0pt,colframe=white,},
title= Définition  \bfseries{#1}}


\newtcolorbox{DfProp}[1][]{
                lower separated=false,
                colback=white,
colframe=colordef,fonttitle=,
colbacktitle=white,
coltitle=colordef,
enhanced,
attach boxed title to top left={yshift=-0.1in,xshift=0.15in},
                 boxed title style={boxrule=0pt,colframe=white,},
title=Définition-Proposition \bfseries{#1}}

\newtcolorbox{Voc}[1][]{
                lower separated=false,
                colback=white,
colframe=colordef,fonttitle=,
colbacktitle=white,
coltitle=colordef,
enhanced,
attach boxed title to top left={yshift=-0.1in,xshift=0.15in},
                 boxed title style={boxrule=0pt,colframe=white,},
title=Vocabulaire  \bfseries{#1}}

\newtcolorbox{Alg}[1][]{
                lower separated=false,
                colback=white,
colframe=colordef,fonttitle=,
colbacktitle=white,
coltitle=colordef,
enhanced,
attach boxed title to top left={yshift=-0.1in,xshift=0.15in},
                 boxed title style={boxrule=0pt,colframe=white,},
title=Algorithme  \bfseries{#1}}

\newtcolorbox{Meth}[1][]{
                lower separated=false,
                colback=white,
colframe=colordef,fonttitle=,
colbacktitle=white,
coltitle=colordef,
enhanced,
attach boxed title to top left={yshift=-0.1in,xshift=0.15in},
                 boxed title style={boxrule=0pt,colframe=white,},
title=Méthode  \bfseries{#1}}


\newtcolorbox{Th}[1][]{
                lower separated=false,
                colback=white,
colframe=colorprop,fonttitle=,
colbacktitle=white,
coltitle=colorprop,
enhanced,
attach boxed title to top left={yshift=-0.1in,xshift=0.15in},
                 boxed title style={boxrule=0pt,colframe=white,},
title=Théorème  \bfseries{#1}}

\newtcolorbox{Lem}[1][]{
                lower separated=false,
                colback=white,
colframe=colorprop,fonttitle=,
colbacktitle=white,
coltitle=colorprop,
enhanced,
attach boxed title to top left={yshift=-0.1in,xshift=0.15in},
                 boxed title style={boxrule=0pt,colframe=white,},
title=Lemme  \bfseries{#1}}


\newtcolorbox{Prop}[1][]{
                lower separated=false,
                colback=white,
colframe=colorprop,fonttitle=,
colbacktitle=white,
coltitle=colorprop,
enhanced,
attach boxed title to top left={yshift=-0.1in,xshift=0.15in},
                 boxed title style={boxrule=0pt,colframe=white,},
title=Proposition  \bfseries{#1}}

\newtcolorbox{Propriete}[1][]{
                lower separated=false,
                colback=white,
colframe=colorprop,fonttitle=,
colbacktitle=white,
coltitle=colorprop,
enhanced,
attach boxed title to top left={yshift=-0.1in,xshift=0.15in},
                 boxed title style={boxrule=0pt,colframe=white,},
title=Propriété  \bfseries{#1}}


\newtcolorbox{Cor}[1][]{
                lower separated=false,
                colback=white,
colframe=colorprop,fonttitle=,
colbacktitle=white,
coltitle=colorprop,
enhanced,
attach boxed title to top left={yshift=-0.1in,xshift=0.15in},
                 boxed title style={boxrule=0pt,colframe=white,},
title=Corollaire  \bfseries{#1}}




\newcommand{\myunit}{1 cm}
\tikzset{
    node style sp/.style={draw,circle,minimum size=\myunit},
    node style ge/.style={circle,minimum size=\myunit},
    arrow style mul/.style={draw,sloped,midway,fill=white},
    arrow style plus/.style={midway,sloped,fill=white},
}

\newcommand*\commentterm[4][]{%
   \begin{tikzpicture}[anchor=base west,baseline,inner sep=0pt, outer sep=0pt,minimum size=0pt]
      \node(xa){$#3$};
      \node[overlay,at=(xa),shift=(#2),color=colorprop](xb){#4};
      \draw[overlay,->,shorten <=2pt,shorten >=2pt,#1,,color=colorprop](xb)to(xa);
   \end{tikzpicture}%
}
 \setcounter{MaxMatrixCols}{20}
%----------------------------------------------------
% Diverses macros pour l'utilisation de TeX
%----------------------------------------------------



%--------------------------------------------------
% Typographie
%--------------------------------------------------

\frenchspacing
%\ThinSpaceInFrenchNumbers


\newcommand{\CS}{\makebox[0cm][r]{$\boxed{\Longleftarrow}$\hspace{1em}}}
  % condition suffisante, entourée dans une boîte et placée
  % à l'extérieur du texte
\newcommand{\CN}{\makebox[0cm][r]{$\boxed{\Longrightarrow}$\hspace{1em}}}
  % condition nécessaire, entourée dans une boîte et placée
  % à l'extérieur du texte

\newcommand{\Implique}[2]%
{\vspace{1ex}\hspace*{-5em}%
  $({\romannumeral#1})\implique({\romannumeral#2})$\hspace{1em}}
% pour écrire (i) => (ii) à l'extérieur du texte
\newcommand{\fImplique}[2]%
{\vspace{1ex}\hspace*{-5em}%
  {\fbox{$({\romannumeral#1})\implique({\romannumeral#2})$}\hspace{1em}}}
  % pour écrire la même chose entouré d'une boîte
\newcommand{\Iff}[2]%
{\vspace{1ex}\hspace*{-5em}%
  $({\romannumeral#1})\iff({\romannumeral#2})$\hspace{1em}}
  % pour écrire (i) <=> (ii) à l'extérieur du texte
\newcommand{\fIff}[2]%
{\vspace{1ex}\hspace*{-5em}%
  {\fbox{$({\romannumeral#1})\iff({\romannumeral#2})$}\hspace{1em}}}
  % pour écrire la même chose entouré d'une boîte




%--------------------------------------------------
% Quelques raccourcis ...
%--------------------------------------------------

\newcommand{\mcal}{\mathcal}
\newcommand{\mrm}{\mathrm}

\newcommand{\dsp}{\displaystyle}
\newcommand{\dps}{\displaystyle} 

% Les grecs ------------------------------
\newcommand{\eps}{\varepsilon}
\newcommand{\beps}{\boldsymbol{\varepsilon}}

\newcommand{\vphi}{\varphi}
\newcommand{\bphi}{\boldsymbol{\varphi}}
\newcommand{\bPhi}{\boldsymbol{\Phi}}

\newcommand{\bpsi}{\boldsymbol{\psi}}
\newcommand{\btau}{\boldsymbol{\tau}}

\newcommand{\la}{\lambda}
\newcommand{\bla}{\boldsymbol{\lambda}}
%----------------------------------------

% Les constantes ------------------------------------
%\newcommand{\ee}{\mathrm{e}}
\DeclareMathOperator{\ee}{e}
  % le nombre e, base de l'exponentielle

\newcommand{\ii}{\mathrm{i}}
  % le célèbre complexe dont le carré vaut -1
%----------------------------------------------------

\newcommand{\ra}{\frac}
  % rapport
\newcommand{\dra}{\dfrac}
  % rapport en displaystyle
\newcommand{\qqs}{\forall}
    % quel que soit ...

\newcommand{\implique}{\;\Longrightarrow\;}
\newcommand{\vide}{\varnothing}
  % l'ensemble vide de "amssymb"

\newcommand{\et}{\mbox{ et }}
\newcommand{\ie}{{}{\emph{i.e.}}}
\newcommand{\etc}{{}{\emph{etc.}}}
\newcommand{\cad}{\hbox{c.-\`a-d.}}


%----------------------------------------------------------
% Les inégalités
%----------------------------------------------------------

\renewcommand{\leq}{\leqslant}
\renewcommand{\geq}{\geqslant}

%--------------------------------------------------
% Ensembles, opérateurs ensemblistes, suites
%--------------------------------------------------

\newcommand{\union}{\mathrel{\cup}}
\newcommand{\Union}{\mathrel{\bigcup}}
\newcommand{\inter}{\mathrel{\cap}}
\newcommand{\Inter}{\mathrel{\bigcap}}
\newcommand{\Sum}{\mathop{\sum}\limits}

\newcommand{\card}{\mathop{\mathrm{card}}\nolimits}
  % cardinal d'un ensemble

\newcommand{\ens}[3][]{\mathopen#1\{ #2\mathbin#1/ #3 \mathclose#1\}}
   % définit un ensemble { x / P(x) }

\newcommand{\sym}[1][n]{\mathfrak{S}_{#1}}
  % groupe symétrique d'ordre n

\DeclareMathSymbol{\complement}{\mathord}{AMSa}{"7B}
  % complément d'un ensemble
\newcommand{\prive}{\setminus}
  % "E\A" se code $E\prive A$, "privé de"

\newcommand{\rond}{\circ}
  % symbole de composition des applications

\newcommand{\combinaison}[2]{\mbox{\large\bf\sf C}_{#1}^{#2}}
  % définit les coefficients binomiaux
\newcommand{\comb}{\combinaison}

% n-uplets--------------------------------------------------

\newcommand{\Dots}{,\ldots,}
\newcommand{\nuple}[2][]{#1( #2_1,\ldots,#2_n #1)}
  % n-uplet (x_1,...,x_n) avec x=#2 et parenthèses extensibles
\newcommand{\puple}[2][]{#1( #2_1,\ldots,#2_p #1)}
  % p-uplet (x_1,...,x_p) avec x=#2 et parenthèses extensibles
\newcommand{\Nuple}[3][]{#1( #2_1,\ldots,#2_#3 #1)}
  % uplet (x_1,...,x_q) avec x=#2, q=#3 et parenthèses extensibles

% suites--------------------------------------------------
\newcommand{\suite}[2][]{#1( #2_n #1)_n}
  % suite (x_n)_n avec x=#2 et parenthèses extensibles
\newcommand{\ssuite}[3][]{#1( #2_#3 #1)_n}
  % suite (x_q)_q avec x=#2, q=#3 et parenthèses extensibles
\newcommand{\Suite}[2][]{#1( \vc{#2}_n #1)_n}
  % suite de vecteurs (x_n)_n avec x=#2 et parenthèses extensibles
\newcommand{\SSuite}[3][]{#1( \vc{#2}_#3 #1)_n}
  % suite de vecteurs (x_q)_q avec x=#2, q=#3 et parenthèses extensibles

% séries--------------------------------------------------
\newcommand{\serie}[1]{\sum#1_n}
\newcommand{\Serie}[2]{\sum(#1_n+#2_n)}


%--------------------------------------------------
% Les ensembles de nombres
%--------------------------------------------------

%\newcommand{\N}{\mathsf{N}}
%\newcommand{\Z}{\mathsf{Z}}
%\newcommand{\Q}{\mathsf{Q}}
%\newcommand{\R}{\mathsf{R}}
%\newcommand{\C}{\mathsf{C}}
%\newcommand{\K}{\mathsf{K}}
%\newcommand{\U}{\mathsf{U}}

\newcommand{\N}{\mathbb{N}}
\newcommand{\Z}{\mathbb{Z}}
\newcommand{\Q}{\mathbb{Q}}
\newcommand{\R}{\mathbb{R}}
\newcommand{\C}{\mathbb{C}}
\newcommand{\K}{\mathbb{K}}
\newcommand{\U}{\mathbb{U}}

\newcommand{\Nbb}{\mathbb{N}}
\newcommand{\Zbb}{\mathbb{Z}}
\newcommand{\Qbb}{\mathbb{Q}}
\newcommand{\Rbb}{\mathbb{R}}
\newcommand{\Cbb}{\mathbb{C}}
\newcommand{\Kbb}{\mathbb{K}}
\newcommand{\Ubb}{\mathbb{U}}

\newcommand{\ccro}[2]{\{#1,\dots,#2\}}

%--------------------------------------------------
% Nombres complexes
%--------------------------------------------------
\newcommand{\conjug}[1]{\overline{#1}}
  % conjuguaison
\newcommand{\IM}{\mathop{\Im\mathrm{m}}\nolimits}
  % partie imaginaire
\newcommand{\RE}{\mathop{\Re\mathrm{e}}\nolimits}
  % partie réelle

%--------------------------------------------------
% Limites
%--------------------------------------------------

\def\buildrel#1_#2^#3{\mathrel{\mathop{\kern 0pt#1}\limits_{#2}^{#3}}}
  % définit une macro pour mettre #2 en dessous et #3 au dessus de #1
\newcommand{\tend}[1][n]{\buildrel{\longrightarrow}_{#1}^{}}
  % -> avec un "n" dessous par défaut
\newcommand{\tendpas}[1][n]{\buildrel{\not\longrightarrow}_{#1}^{}}
  % /-> avec un "n" dessous par défaut
\newcommand{\equivalent}[1][n]{\buildrel{\sim}_{#1}^{}}
  % le signe "équivalent" avec un "n" dessous par défaut
\newcommand{\egal}[1][n]{\buildrel{=}_{#1}^{}}
  % le signe "=" avec un "n" dessous par défaut

%--------------------------------------------------
% Dérivée et intégrale
%--------------------------------------------------

\DeclareMathOperator{\D}{D}
  % pour la dérivation

\newcommand{\sub}[1]{\sigma_{\vc#1}}
  % subdivision
\newcommand{\intd}{\int\!\!\!\int}
  % intégrale double
\newcommand{\intdepi}{\ra1{2\pi}\int_0^{2\pi}}
  % intégrale entre 0 et 2\pi
\newcommand{\intab}{\int_{\intf ab}}
  % intégrale sur le segment [a,b]

\newcommand{\dt}[1][t]{\mathrm{d}\mspace{-0.5mu}#1}
  % élément différentiel dans une intégrale, dt par défaut
  
\newcommand{\entre}[3][\big]{#1\rvert_{#2}^{#3}}
  % accroissement d'une fonction entre #2 et #3, \big par défaut


\newcommand{\del}[2]{\ra{\partial#1}{\partial#2}}
  % dérivée partielle : del f/del x
\newcommand{\ddel}[3]{\ra{\partial^2#1}{\partial#2\,\partial#3}}
  % dérivée partielle seconde : del^2 f/del x del y
\newcommand{\Del}[3]{\ra{\partial^#1#2}{\partial#3^#1}}
\newcommand{\dd}[2]{\ra{\mathrm{d}#1}{\mathrm{d}#2}}
\DeclareMathOperator{\J}{J}
  %jacobien


%--------------------------------------------------
% Définition des vecteurs
%--------------------------------------------------

\newcommand{\Vect}[1]{\vbox{\halign{##\cr 
  \tiny\rightarrowfill\cr\noalign{\nointerlineskip\vskip1pt} 
  $#1\mskip2mu$\cr}}}
  % vecteur avec une flèche de hauteur constante
\newcommand{\vect}[1]{\vbox{\halign{##\cr 
  \tiny\rightarrowfill\cr\noalign{\nointerlineskip\vskip1pt} 
  $#1\mskip2mu$\cr}}}
  % vecteur avec une flèche

\newcommand{\vc}[1]{\mathbf{#1}}
  % vecteur écrit en gras
\newcommand{\bvc}[1]{\boldsymbol{#1}}
  % vecteur écrit en gras, cas des grecs

\newcommand{\grad}{\mathop{\vect{\mathrm{grad}}}\nolimits}
  % le vecteur gradient
\DeclareMathOperator{\divergence}{div}
\newcommand{\diver}{\divergence}
  % l'opérateur divergence

%--------------------------------------------------
% Diverses normes
%--------------------------------------------------

\newcommand{\abs}[2][]{\mathopen#1\lvert #2 \mathclose#1\rvert}
  % valeur absolue...
\newcommand\PS[2]{\langle#1,#2\rangle}
  % produit scalaire
  
\newcommand{\norme}[2][]{\mathopen#1\lVert #2 \mathclose#1\rVert}
  % norme d'une fonction
\newcommand{\norm}{\norme}

\newcommand{\normeun}[2][]{\mathopen#1\lVert #2 \mathclose#1\rVert_1}
  % norme d'une fonction
\newcommand{\normu}{\normeun}

\newcommand{\normedeux}[2][]{\mathopen#1\lVert #2 \mathclose#1\rVert_2}
  % norme d'une fonction
\newcommand{\normd}{\normedeux}

\newcommand{\normeinfinie}[2][]{\mathopen#1\lVert #2 \mathclose#1\rVert_\infty}
  % norme d'une fonction
\newcommand{\normi}{\normeinfinie}

\newcommand{\Norme}[1][N]{\mathcal{#1}}

\DeclareMathOperator{\dist}{d}
  % distance

\newcommand{\Bo}[3][]{\mathcal{B}#1(\vc{#2},#3 #1)}
  % Boule ouverte de centre #2 et de rayon #3, avec parenthèses variables
\newcommand{\Bf}[3][]{\mathcal{B}_f#1(\vc{#2},#3 #1)}
  % Boule fermée de centre #2 et de rayon #3, avec parenthèses variables

%--------------------------------------------------
% Produit scalaire
%   JLQ le 29/06/98
%--------------------------------------------------

\newcommand{\scal}[3][]{#1\langle #2 \mathrel{#1\vert} #3 #1\rangle}
    % le produit scalaire
\newcommand{\Scal}[3][]{#1\langle \vc{#2} \mathrel{#1\vert} \vc{#3} #1\rangle}
    % le produit scalaire et des vecteurs écrits en gras

%--------------------------------------------------
% Ecriture des intervalles
%--------------------------------------------------

\newcommand{\into}[3][]{\mathopen{#1]}#2,#3\mathclose{#1[}}
  % intervalle ouvert ]#2;#3[
\newcommand{\intof}[3][]{\mathopen{#1]}#2,#3 \mathclose#1\rbrack}
  % intervalle ouvert-fermé ]#2;#3]
\newcommand{\intfo}[3][]{\mathopen#1\lbrack #2,#3 \mathclose{#1[}}
  % intervalle fermé-ouvert [#2;#3[
\newcommand{\intf}[3][]{\mathopen#1\lbrack  #2,#3 \mathclose#1\rbrack}
  % intervalle fermé [#2,#3] 

\newcommand{\Intf}[3][]{\mathopen#1\lbrack\!#1\lbrack#2,#3 \mathclose#1\rbrack\!#1\rbrack}
       %intervalle fermé [|#2;#3|] pour les nombres entiers

%--------------------------------------------------
% Les fonctions, les espaces fonctionnels
%--------------------------------------------------

\DeclareMathOperator{\ent}{Ent}
  % partie entière
\DeclareMathOperator{\Arg}{Arg}
  % détermination principale de l'argument d'un nombre complexe
\newcommand{\restr}[2][]{#1\rvert_{#2}}
  % restriction d'une fonction à #2

\newcommand{\oo}[2][]{\mathrm{o}#1(#2#1)}
  % notation de Landau
\newcommand{\OO}[2][]{\mathrm{O}#1(#2#1)}
  % notation de Landau


% Fonctions arcsinus, ... --------------------------------------

\DeclareMathOperator{\ch}{ch}
\DeclareMathOperator{\sh}{sh}
%\DeclareMathOperator{\th}{th}

\DeclareMathOperator{\cn}{cn}
\DeclareMathOperator{\sn}{sn}
\DeclareMathOperator{\dn}{dn}

\DeclareMathOperator{\argth}{Arg\,th}
\DeclareMathOperator{\argsh}{Arg\,sh}
%----------------------------------------------------------------------

% les fonctions ... --------------------------------------------------
\newcommand{\FIE}[1][I,E]{\mathcal{F}(#1)}
\newcommand{\FabE}[2][{a}{b}]{\mathcal{F}(\intf #1,#2)}
\newcommand{\Fab}[1][{a}{b}]{\mathcal{F}(\intf #1)}

% les fonctions bornées ... ------------------------------------------
\newcommand{\BIE}[1][I,E]{\mathcal{B}(#1)}
\newcommand{\BabE}[2][{a}{b}]{\mathcal{B}(\intf #1,#2)}
\newcommand{\Bab}[1][{a}{b}]{\mathcal{B}(\intf #1)}

% les fonctions en escalier ... --------------------------------------
\newcommand{\EscIE}[1][I,E]{\mathcal{E}sc(#1)}
\newcommand{\EscabE}[2][{a}{b}]{\mathcal{E}sc(\intf #1,#2)}
\newcommand{\Escab}[1][{a}{b}]{\mathcal{E}sc(\intf #1)}

% les fonctions continues ... -----------------------------------------
\newcommand{\CIE}[1][I,E]{\mathcal{C}(#1)}
\newcommand{\CI}[1][I]{\mathcal{C}(#1)}
\newcommand{\CabE}[2][{a}{b}]{\mathcal{C}(\intf #1,#2)}
\newcommand{\Cab}[1][{a}{b}]{\mathcal{C}(\intf #1)}

% les fonctions continues par morceaux ... ----------------------------
\newcommand{\CMIE}[1][I,E]{\mathcal{CM}(#1)}
\newcommand{\CMI}[1][I]{\mathcal{CM}(#1)}
\newcommand{\CMplI}[1][I]{\mathcal{CM}^{+}(#1)}
\newcommand{\CMabE}[2][{a}{b}]{\mathcal{CM}(\intf #1,#2)}
\newcommand{\CMab}[1][{a}{b}]{\mathcal{CM}(\intf #1)}

% les fonctions sommables ... -----------------------------------------
\newcommand{\LI}[2][1]{\mathcal{L}^{#1}(#2)}
\newcommand{\LCI}[2][1]{\mathcal{L}_{\mathcal{C}}^{#1}(#2)}

% les fonctions dérivables ... ----------------------------------------
\newcommand{\DIE}[1][I,E]{\mathcal{D}(#1)}
\newcommand{\DabE}[2][{a}{b}]{\mathcal{D}(\intf #1,#2)}
\newcommand{\Dab}[1][{a}{b}]{\mathcal{D}(\intf #1)}

% les fonctions de classe Ck ... --------------------------------------
\newcommand{\CkIE}[2][I,E]{\mathcal{C}^{#2}(#1)}
\newcommand{\CkabE}[3][{a}{b}]{\mathcal{C}^{#3}(\intf #1,#2)}
\newcommand{\Ckab}[2][{a}{b}]{\mathcal{C}^{#2}(\intf #1)}

% les fonctions k fois dérivables ... ---------------------------------
\newcommand{\DkIE}[2][I,E]{\mathcal{D}^{#2}(#1)}
\newcommand{\DkabE}[3][{a}{b}]{\mathcal{D}^{#3}(\intf #1,#2)}
\newcommand{\Dkab}[2][{a}{b}]{\mathcal{D}^{#2}(\intf #1)}

%les fonctions de classe Ck par morceaux ... --------------------------
\newcommand{\CMkIE}[2][I,E]{\mathcal{C}^{#2}\mahtcal{M}(#1)}
\newcommand{\CMkabE}[3][{a}{b}]{\mathcal{C}^{#3}\mathcal{M}(\intf #1,#2)}
\newcommand{\CMkab}[2][{a}{b}]{\mathcal{C}^{#2}\mathcal{M}(\intf #1)}

% les fonctions périodiques ... ----------------------------------------
\newcommand{\Cdepi}{\mathcal{C}_{2\pi}}
\newcommand{\CMdepi}{\mathcal{CM}_{2\pi}}
\newcommand{\Tndepi}{\mathcal{T}_{n,2\pi}}
\newcommand{\CT}{\mathcal{C}_{T}}
\newcommand{\Tdepi}{\mathcal{T}_{2\pi}}
\newcommand{\TT}{\mathcal{T}_{T}}

%--------------------------------------------------
% Les matrices,
%--------------------------------------------------

\newcommand{\mat}{\mathrm{Mat}}
  % matrice Mat dans la base #1 (B)
\newcommand{\Mat}[2]{\mathcal{M}\mathrm{at}_{\mathcal{#1},\mathcal{#2}}}
  % matrice dans les bases #1 et #2
\newcommand{\diag}[2][]{\mathrm{Diag}#1( #2_1,\ldots,#2_n #1)}
  % matrice diagonale d'ordre n
\newcommand{\Diag}[3][]{\mathrm{Diag}#1( #2_1,\ldots,#2_{#3} #1)}
  % matrice diagonale d'ordre #2




\newcommand{\transposee}[1]{#1^{\mathsf {T}}}
    % nouvelle version;
    % attention, les textes doivent être revus pour
    % l'utiliser.
  %transposition d'une matrice
\newcommand{\trans}{\mbox{${}^t\hspace{-1pt}$}} % ancienne version de jlq
  % transposition
\newcommand{\tc}[1]{\mbox{${}^t\hspace{-1pt}$}\conjug{#1}}
  % transposition et conjugaison
\DeclareMathOperator{\com}{Com}
  % comatrice ou matrice des cofacteurs


\DeclareMathOperator{\Vectt}{Vect}
% Espace vectoriel engendré
%\DeclareMathOperator{\sp}{sp}
\renewcommand{\sp}{\mathop{\mathrm{sp}}\nolimits}
  % spectre d'une matrice, ...
\DeclareMathOperator{\rg}{rg}
\DeclareMathOperator{\Rang}{rg}
  % rang d'une matrice, ...
\DeclareMathOperator{\Ker}{Ker}
\DeclareMathOperator{\im}{Im}
\DeclareMathOperator{\Ima}{Im}
  % image d'une application linéaire
\DeclareMathOperator{\tr}{tr}
\DeclareMathOperator{\Tr}{tr}
  % trace d'une matrice, ...
%\DeclareMathOperator{\deter}{det}
\renewcommand{\det}{\mathop{\mathrm{det}}\nolimits}
  % déterminant, \det est "mal défini" chez AmsTeX!
\newcommand{\Det}[1][B]{\det_{\mathcal{#1}}}
  % déterminant dans la base B ...
\newcommand{\M}[3]{\mathcal{M}_{#1,#2}(#3)}
\newcommand{\Mn}[2]{\mathcal{M}_{#1}(#2)}

\newcommand{\MnK}{\Mn{n}{\K}}
\newcommand{\MnpK}{\M{n}{p}{\K}}
\newcommand{\GLnK}{\mathcal{GL}_n(\K)}
\newcommand{\GLn}[2][n]{\mathcal{GL}_{#1}(#2)}
\newcommand{\In}{I_n}
  % trace d'une matrice, ...
%--------------------------------------------------
% Espaces vectoriels
%--------------------------------------------------

\newcommand{\somdir}{\mathop{\oplus}\nolimits}
\newcommand{\Somdir}{\mathop{\oplus}\limits}
%\newcommand{\somDir}{\mathop{\bigoplus}\nolimits}
%\newcommand{\SomDir}{\mathop{\bigoplus}\limits}

\newcommand{\lin}[1]{\mathcal{L}(#1)}
\newcommand{\Lin}[2]{\mathcal{L}(#1,#2)}

\newcommand{\LEF}[1][E,F]{\mathcal{L}(#1)}
\newcommand{\LE}[1][E]{\mathcal{L}(#1)}
\newcommand{\LK}[2][K]{\mathcal{L}_{\#1}(#2)}

\newcommand{\SE}[1][E]{\mathscr{S}(#1)}
  % ensemble des endomorphismes symétriques sur #1, E par défaut
\newcommand{\ASE}[1][E]{\mathscr{A}(#1)}
  % ensemble des endomorphismes antisymétriques sur #1, E par défaut
\newcommand{\SnR}[1][\R]{\mathscr{S}_n(#1)}
  % ensemble des matrices symétriques d'ordre n sur #1, \R par défaut
\newcommand{\AnR}[1][\R]{\mathscr{A}_n(#1)}
  % ensemble des matrices antisymétriques d'ordre n sur #1, \R par défaut


\newcommand{\GLE}[1][E]{\mathcal{GL}(#1)}
  % groupe des automorphismes de #1, par défaut E
\newcommand{\GLK}[2][K]{\mathcal{GL}_{\#1}(#2)}

\newcommand{\OrE}[1][E]{\mathcal{O}(#1)}
  % groupe des automorphismes orthogonaux de #1, par défaut E
\newcommand{\SOE}[1][E]{\mathcal{SO}(#1)}
  % groupe spécial orthogonal de #1, par défaut E
\newcommand{\OpE}[1][E]{\mathcal{O}^+(#1)}
  % groupe spécial orthogonal de #1, par défaut E
\newcommand{\OmE}[1][E]{\mathcal{O}^-(#1)}
  % ensemble des automorphismes orthogonaux de det=-1 de #1, par défaut E
\newcommand{\OnR}[1][n]{\mathcal{O}_{#1}(\R)}
  % groupe des automorphismes orthogonaux de #2^#1, par défaut #2^n

\newcommand{\SOnR}[1][n]{\mathcal{SO}_{#1}(\R)}
  % groupe spécial orthogonal d'ordre #1, par défaut n
\newcommand{\Opn}[1][n]{\mathcal{O}^+(#1)}
  % groupe spécial orthogonal d'ordre #1, par défaut n
\newcommand{\Omn}[1][n]{\mathcal{O}^-(#1)}
  % ensemble des matrices orthogonales de det=-1, d'ordre #1, par défaut n


\newcommand{\LpEF}{\mathcal{L}_{p}(E,F)}
\newcommand{\Lp}[2][p]{\mathcal{L}_{#1}(#2)}

%---------------------------------------------------------------------- from yann
\newcommand\Fonction[5]{{#1}\left|\begin{aligned}{#2}&\;\longrightarrow\;{#3}\\{#4}&\;\longmapsto\;{#5}\end{aligned}\right.}
\newcommand\ninf{{n\to \infty}}
\DeclareMathOperator*\PetitO{o}
\DeclareMathOperator*\GrandO{O}
\DeclareMathOperator*\Sim{\sim}




\usepackage{tkz-tab}
\usepackage{pgf}
\usetikzlibrary{arrows}
\usetikzlibrary{patterns}  
\definecolor{CyanTikz40}{cmyk}{.4,0,0,0}
\definecolor{CyanTikz20}{cmyk}{.2,0,0,0}
\tikzstyle{general}=[line width=0.3mm, >=stealth, x=1cm, y=1cm,line cap=round, line join=round]
\tikzstyle{quadrillage}=[line width=0.3mm, color=CyanTikz40]
\tikzstyle{quadrillageNIV2}=[line width=0.3mm, color=CyanTikz20]
\tikzstyle{quadrillage55}=[line width=0.3mm, color=CyanTikz40, xstep=0.5, ystep=0.5]
\tikzstyle{cote}=[line width=0.3mm, <->]
\tikzstyle{epais}=[line width=0.5mm, line cap=butt]
\tikzstyle{tres epais}=[line width=0.8mm, line cap=butt]
\tikzstyle{axe}=[line width=0.3mm, ->, color=Noir, line cap=rect]
\newcommand{\quadrillageSeyes}[2]{\draw[line width=0.3mm, color=A1!10, ystep=0.2, xstep=0.8] #1 grid #2;
\draw[line width=0.3mm, color=A1!30, xstep=0.8, ystep=0.8] #1 grid #2; }
\newcommand{\axeX}[4][0]{\draw[axe] (#2,#1)--(#3,#1); \foreach \x in {#4} {\draw (\x,#1) node {\small $+$}; \draw (\x,#1) node[below] {\small $\x$};}}
\newcommand{\axeY}[4][0]{\draw[axe] (#1,#2)--(#1,#3); \foreach \y in {#4} {\draw (#1, \y) node {\small $+$}; \draw (#1, \y) node[left] {\small $\y$};}}
\newcommand{\axeOI}[3][0]{\draw[axe] (#2,#1)--(#3,#1);  \draw (1,#1) node {\small $+$}; \draw (1,#1) node[below] {\small $I$};}
\newcommand{\axeOJ}[3][0]{\draw[axe] (#1,#2)--(#1,#3); \draw (#1, 1) node {\small $+$}; \draw (#1, 1) node[left] {\small $J$};}
\newcommand{\axeXgraduation}[2][0]{\foreach \x in {#2} {\draw (\x,#1) node {\small $+$};}}
\newcommand{\axeYgraduation}[2][0]{\foreach \y in {#2} {\draw (#1, \y) node {\small $+$}; }}
\newcommand{\origine}{\draw (0,0) node[below left] {\small $0$};}
\newcommand{\origineO}{\draw (0,0) node[below left] {$O$};}
\newcommand{\point}[4]{\draw (#1,#2) node[#4] {$#3$};}
\newcommand{\pointGraphique}[4]{\draw (#1,#2) node[#4] {$#3$};
\draw (#1,#2) node {$+$};}
\newcommand{\pointFigure}[4]{\draw (#1,#2) node[#4] {$#3$};
\draw (#1,#2) node {$\times$};}
\newcommand{\pointC}[3]{\draw (#1) node[#3] {$#2$};}
\newcommand{\pointCGraphique}[3]{\draw (#1) node[#3] {$#2$};
\draw (#1) node {$+$};}
\newcommand{\pointCFigure}[3]{\draw (#1) node[#3] {$#2$};
\draw (#1) node {$\times$};}
\definecolor{A1}                {cmyk}{1.00, 0.00, 0.00, 0.50}
\definecolor{A2}                {cmyk}{0.60, 0.00, 0.00, 0.10}
\definecolor{A3}                {cmyk}{0.30, 0.00, 0.00, 0.05}
\definecolor{A1}                {cmyk}{1.00, 0.00, 0.00, 0.50}
\definecolor{A2}                {cmyk}{0.60, 0.00, 0.00, 0.10}
\definecolor{A3}                {cmyk}{0.30, 0.00, 0.00, 0.05}
\definecolor{A4}                {cmyk}{0.10, 0.00, 0.00, 0.00}
\definecolor{B1}                {cmyk}{0.00, 1.00, 0.60, 0.40}
\definecolor{B2}                {cmyk}{0.00, 0.85, 0.60, 0.15}
\definecolor{B3}                {cmyk}{0.00, 0.20, 0.15, 0.05}
\definecolor{B4}                {cmyk}{0.00, 0.05, 0.05, 0.00}
\definecolor{C1}                {cmyk}{0.00, 1.00, 0.00, 0.50}
\definecolor{C2}                {cmyk}{0.00, 0.60, 0.00, 0.20}
\definecolor{C3}                {cmyk}{0.00, 0.30, 0.00, 0.05}
\definecolor{C4}                {cmyk}{0.00, 0.10, 0.00, 0.05}
\definecolor{D1}                {cmyk}{0.00, 0.00, 1.00, 0.50}
\definecolor{D2}                {cmyk}{0.20, 0.20, 0.80, 0.00}
\definecolor{D3}                {cmyk}{0.00, 0.00, 0.20, 0.10}
\definecolor{D4}                {cmyk}{0.00, 0.00, 0.20, 0.05}
\definecolor{F1}                {cmyk}{0.00, 0.80, 0.50, 0.00}
\definecolor{F2}                {cmyk}{0.00, 0.40, 0.30, 0.00}
\definecolor{F3}                {cmyk}{0.00, 0.15, 0.10, 0.00}
\definecolor{F4}                {cmyk}{0.00, 0.07, 0.05, 0.00}
\definecolor{G1}                {cmyk}{1.00, 0.00, 0.50, 0.00}
\definecolor{G2}                {cmyk}{0.50, 0.00, 0.20, 0.00}
\definecolor{G3}                {cmyk}{0.20, 0.00, 0.10, 0.00}
\definecolor{G4}                {cmyk}{0.10, 0.00, 0.05, 0.00}
\definecolor{H1}                {cmyk}{0.40, 0.00, 1.00, 0.10}
\definecolor{H2}                {cmyk}{0.20, 0.00, 0.50, 0.05}
\definecolor{H3}                {cmyk}{0.10, 0.00, 0.20, 0.00}
\definecolor{H4}                {cmyk}{0.07, 0.00, 0.15, 0.00}
\definecolor{J1}                {cmyk}{0.00, 0.50, 1.00, 0.00}
\definecolor{J2}                {cmyk}{0.00, 0.20, 0.50, 0.00}
\definecolor{J3}                {cmyk}{0.00, 0.10, 0.20, 0.00}
\definecolor{J4}                {cmyk}{0.00, 0.07, 0.15, 0.00}
\definecolor{FondOuv}           {cmyk}{0.00, 0.05, 0.10, 0.00}
\definecolor{FondAutoEvaluation}{cmyk}{0.00, 0.03, 0.15, 0.00}
\definecolor{FondTableaux}      {cmyk}{0.00, 0.00, 0.20, 0.00}
\definecolor{FondAlgo}          {cmyk}{0.07, 0.00, 0.30, 0.00}
\definecolor{BleuOuv}           {cmyk}{1.00, 0.00, 0.00, 0.00}
\definecolor{PartieFonction}    {cmyk}{1.00, 0.00, 0.00, 0.00}
\definecolor{PartieGeometrie}   {cmyk}{0.80, 0.80, 0.00, 0.00}
\definecolor{PartieStatistique} {cmyk}{0.60, 0.95, 0.00, 0.20}
\definecolor{PartieStatistiqueOLD} {cmyk}{0.95, 0.60, 0.20, 0.00}
\definecolor{PartieStatistique*}{cmyk}{0.30, 1.00, 0.00, 0.00}
\definecolor{U1}                {cmyk}{0.50, 0.10, 0.00, 0.10}
\definecolor{U2}                {cmyk}{0.20, 0.15, 0.00, 0.00}
\definecolor{CouleurRenvoiManuelNumerique}{cmyk}{0.40, 0.00, 0.00, 0.00}
\definecolor{Blanc}             {gray}{1.00}
\definecolor{gris1}             {gray}{0.80}
\definecolor{gris2}             {gray}{0.60}
\definecolor{gris3}             {gray}{0.40}
\definecolor{Noir}              {gray}{0.00}