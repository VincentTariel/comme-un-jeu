\documentclass{book} 
\usepackage{t1enc}
\usepackage[utf8]{inputenc}
\usepackage[french]{minitoc}
 \usepackage{amsmath}
\usepackage{fancyhdr,amsmath,amsthm,amssymb,fancybox}
\usepackage[francais]{babel}
\usepackage{amsmath,thmtools}
\usepackage{tikz}
\usepackage{tkz-fct}   
\usepackage{a4wide} 
\usepackage{graphicx}
\usepackage{thmbox}
\usepackage{changepage}
\usepackage{xcolor}
\usepackage{sectsty}
 \usepackage{enumitem}
\usepackage{eurosym}
\usepackage{mathrsfs}
\usepackage{mathtools}

\usetikzlibrary{matrix,arrows,decorations.pathmorphing}
\usepackage[framemethod=TikZ]{mdframed}
\usepackage{geometry}
\usepackage{pgfplots}
\usepackage[most]{tcolorbox}

\geometry{hmargin=2cm,vmargin=1.5cm}
\pagestyle{fancy}

\usepackage{titlesec}

\titleformat
{\chapter} % command
[display] % shape
{\bfseries\huge} % format
{Story No. \ \thechapter} % label
{-5ex} % sep
{
    \rule{\textwidth}{1pt}
    \vspace{1ex}
    \centering
} % before-code
[
\vspace{-2ex}%
\rule{\textwidth}{0.3pt}
]
\titlespacing*{\chapter}{0pt}{-50pt}{10pt}
\renewcommand{\chaptermark}[1]{\markboth{#1}{}}
\renewcommand{\sectionmark}[1]{\markright{\thesection\ #1}}
\fancyhf{}
\fancyhead[LE,RO]{\bfseries\thepage}
\fancyhead[LO]{\bfseries\rightmark}
\fancyhead[RE]{\bfseries\leftmark}
\fancypagestyle{plain}{%
   \fancyhead{} % get rid of headers
   \renewcommand{\headrulewidth}{0pt} % and the line
}



\renewcommand{\thesection}{\Roman{section}} 
\renewcommand{\thesubsection}{\Alph{subsection}}



\definecolor{colorprop}{rgb}{0.0, 0.4, 0.74}
\colorlet{colordef}{red!60}

\sectionfont{\color{black}}
\subsectionfont{\color{black}}
\subsubsectionfont{\color{black}}


\newcommand{\definebox}[3]{%
  \newcounter{#1}
  \newenvironment{#1}[1][]{%
    \stepcounter{#1}%
    \mdfsetup{%
        frametitle={%
            \tikz[baseline=(current bounding box.east),outer sep=0pt]
            \node[anchor=east,rectangle,fill=white]
            {\strut \color{#3}\MakeUppercase#2~\csname \endcsname\ifstrempty{##1}{}{:~##1}};}}%
    \mdfsetup{innertopmargin=10pt,linecolor=#3,%
        linewidth=1pt,topline=true,roundcorner=5pt,
        frametitleaboveskip=\dimexpr-\ht\strutbox\relax,}%
    \begin{mdframed}[]\ignorespaces\relax
    }{\end{mdframed}}%
    \ignorespaces
}
\newcommand{\defineboxx}[3]{%
  \newcounter{#1}
  \newenvironment{#1}[1][]{%
    \stepcounter{#1}%
    \mdfsetup{%
        frametitle={%
            \hspace*{-40pt}\tikz[baseline=(current bounding box.east),outer sep=0pt]
            \node[anchor=east,rectangle,fill=white]
            {\strut \MakeUppercase#2~\csname \endcsname\ifstrempty{##1}{}{:~##1}};}}%
    \mdfsetup{innertopmargin=0pt,linecolor=#3,%
        linewidth=1pt,topline=false,innerleftmargin=20pt,bottomline=false,rightline=false,
        frametitleaboveskip=\dimexpr-\ht\strutbox\relax,}%
    \begin{mdframed}[]\relax%
    }{\end{mdframed}}%
}
\newcommand{\defineboxxx}[3]{%
  \newcounter{#1}
  \newenvironment{#1}[1][]{%
    \stepcounter{#1}%
    \mdfsetup{%
        frametitle={%
            \hspace*{-40pt}\tikz[baseline=(current bounding box.east),outer sep=0pt]
            \node[anchor=east,rectangle,fill=white]
            {\strut \MakeUppercase#2~\csname \endcsname\ifstrempty{##1}{}{:~##1}};}}%
    \mdfsetup{innertopmargin=0pt,linecolor=#3,%
        linewidth=1pt,topline=false,innerleftmargin=20pt,bottomline=false,rightline=false,leftline=false,
        frametitleaboveskip=\dimexpr-\ht\strutbox\relax,}%
    \begin{mdframed}[]\relax%
    }{\end{mdframed}}%
}

\defineboxx{Proof}{Démonstration}{colorprop}
\defineboxx{Ex}{Exemple}{black}
\defineboxx{Activite}{Activité}{black}
\defineboxxx{NB}{Remarque}{black}
\usepackage[colorlinks = true,urlcolor  = colorprop]{hyperref}
\newenvironment{Text}{}{}
\newenvironment{Correction}{\textbf{\newline Correction } }{}
   
\newcommand{\defi}[1]{\textbf{\textcolor{colordef}{#1}}} 
\newcommand{\propri}[1]{\textbf{\textcolor{colorprop}{#1}}}
 \newcommand{\impo}[1]{\textbf{\textcolor{colorprop}{#1}}}

\newtcolorbox{Df}[1][]{
                lower separated=false,
                colback=white,
colframe=colordef,fonttitle=,
colbacktitle=white,
coltitle=colordef,
enhanced,
attach boxed title to top left={yshift=-0.1in,xshift=0.15in},
                 boxed title style={boxrule=0pt,colframe=white,},
title= Définition  \bfseries{#1}}


\newtcolorbox{DfProp}[1][]{
                lower separated=false,
                colback=white,
colframe=colordef,fonttitle=,
colbacktitle=white,
coltitle=colordef,
enhanced,
attach boxed title to top left={yshift=-0.1in,xshift=0.15in},
                 boxed title style={boxrule=0pt,colframe=white,},
title=Définition-Proposition \bfseries{#1}}

\newtcolorbox{Voc}[1][]{
                lower separated=false,
                colback=white,
colframe=colordef,fonttitle=,
colbacktitle=white,
coltitle=colordef,
enhanced,
attach boxed title to top left={yshift=-0.1in,xshift=0.15in},
                 boxed title style={boxrule=0pt,colframe=white,},
title=Vocabulaire  \bfseries{#1}}

\newtcolorbox{Alg}[1][]{
                lower separated=false,
                colback=white,
colframe=colordef,fonttitle=,
colbacktitle=white,
coltitle=colordef,
enhanced,
attach boxed title to top left={yshift=-0.1in,xshift=0.15in},
                 boxed title style={boxrule=0pt,colframe=white,},
title=Algorithme  \bfseries{#1}}

\newtcolorbox{Meth}[1][]{
                lower separated=false,
                colback=white,
colframe=colordef,fonttitle=,
colbacktitle=white,
coltitle=colordef,
enhanced,
attach boxed title to top left={yshift=-0.1in,xshift=0.15in},
                 boxed title style={boxrule=0pt,colframe=white,},
title=Méthode  \bfseries{#1}}


\newtcolorbox{Th}[1][]{
                lower separated=false,
                colback=white,
colframe=colorprop,fonttitle=,
colbacktitle=white,
coltitle=colorprop,
enhanced,
attach boxed title to top left={yshift=-0.1in,xshift=0.15in},
                 boxed title style={boxrule=0pt,colframe=white,},
title=Théorème  \bfseries{#1}}

\newtcolorbox{Lem}[1][]{
                lower separated=false,
                colback=white,
colframe=colorprop,fonttitle=,
colbacktitle=white,
coltitle=colorprop,
enhanced,
attach boxed title to top left={yshift=-0.1in,xshift=0.15in},
                 boxed title style={boxrule=0pt,colframe=white,},
title=Lemme  \bfseries{#1}}


\newtcolorbox{Prop}[1][]{
                lower separated=false,
                colback=white,
colframe=colorprop,fonttitle=,
colbacktitle=white,
coltitle=colorprop,
enhanced,
attach boxed title to top left={yshift=-0.1in,xshift=0.15in},
                 boxed title style={boxrule=0pt,colframe=white,},
title=Proposition  \bfseries{#1}}

\newtcolorbox{Propriete}[1][]{
                lower separated=false,
                colback=white,
colframe=colorprop,fonttitle=,
colbacktitle=white,
coltitle=colorprop,
enhanced,
attach boxed title to top left={yshift=-0.1in,xshift=0.15in},
                 boxed title style={boxrule=0pt,colframe=white,},
title=Propriété  \bfseries{#1}}


\newtcolorbox{Cor}[1][]{
                lower separated=false,
                colback=white,
colframe=colorprop,fonttitle=,
colbacktitle=white,
coltitle=colorprop,
enhanced,
attach boxed title to top left={yshift=-0.1in,xshift=0.15in},
                 boxed title style={boxrule=0pt,colframe=white,},
title=Corollaire  \bfseries{#1}}




\newcommand{\myunit}{1 cm}
\tikzset{
    node style sp/.style={draw,circle,minimum size=\myunit},
    node style ge/.style={circle,minimum size=\myunit},
    arrow style mul/.style={draw,sloped,midway,fill=white},
    arrow style plus/.style={midway,sloped,fill=white},
}

\newcommand*\commentterm[4][]{%
   \begin{tikzpicture}[anchor=base west,baseline,inner sep=0pt, outer sep=0pt,minimum size=0pt]
      \node(xa){$#3$};
      \node[overlay,at=(xa),shift=(#2),color=colorprop](xb){#4};
      \draw[overlay,->,shorten <=2pt,shorten >=2pt,#1,,color=colorprop](xb)to(xa);
   \end{tikzpicture}%
}
 \setcounter{MaxMatrixCols}{20}
%----------------------------------------------------
% Diverses macros pour l'utilisation de TeX
%----------------------------------------------------



%--------------------------------------------------
% Typographie
%--------------------------------------------------

\frenchspacing
%\ThinSpaceInFrenchNumbers


\newcommand{\CS}{\makebox[0cm][r]{$\boxed{\Longleftarrow}$\hspace{1em}}}
  % condition suffisante, entourée dans une boîte et placée
  % à l'extérieur du texte
\newcommand{\CN}{\makebox[0cm][r]{$\boxed{\Longrightarrow}$\hspace{1em}}}
  % condition nécessaire, entourée dans une boîte et placée
  % à l'extérieur du texte

\newcommand{\Implique}[2]%
{\vspace{1ex}\hspace*{-5em}%
  $({\romannumeral#1})\implique({\romannumeral#2})$\hspace{1em}}
% pour écrire (i) => (ii) à l'extérieur du texte
\newcommand{\fImplique}[2]%
{\vspace{1ex}\hspace*{-5em}%
  {\fbox{$({\romannumeral#1})\implique({\romannumeral#2})$}\hspace{1em}}}
  % pour écrire la même chose entouré d'une boîte
\newcommand{\Iff}[2]%
{\vspace{1ex}\hspace*{-5em}%
  $({\romannumeral#1})\iff({\romannumeral#2})$\hspace{1em}}
  % pour écrire (i) <=> (ii) à l'extérieur du texte
\newcommand{\fIff}[2]%
{\vspace{1ex}\hspace*{-5em}%
  {\fbox{$({\romannumeral#1})\iff({\romannumeral#2})$}\hspace{1em}}}
  % pour écrire la même chose entouré d'une boîte




%--------------------------------------------------
% Quelques raccourcis ...
%--------------------------------------------------

\newcommand{\mcal}{\mathcal}
\newcommand{\mrm}{\mathrm}

\newcommand{\dsp}{\displaystyle}
\newcommand{\dps}{\displaystyle} 

% Les grecs ------------------------------
\newcommand{\eps}{\varepsilon}
\newcommand{\beps}{\boldsymbol{\varepsilon}}

\newcommand{\vphi}{\varphi}
\newcommand{\bphi}{\boldsymbol{\varphi}}
\newcommand{\bPhi}{\boldsymbol{\Phi}}

\newcommand{\bpsi}{\boldsymbol{\psi}}
\newcommand{\btau}{\boldsymbol{\tau}}

\newcommand{\la}{\lambda}
\newcommand{\bla}{\boldsymbol{\lambda}}
%----------------------------------------

% Les constantes ------------------------------------
%\newcommand{\ee}{\mathrm{e}}
\DeclareMathOperator{\ee}{e}
  % le nombre e, base de l'exponentielle

\newcommand{\ii}{\mathrm{i}}
  % le célèbre complexe dont le carré vaut -1
%----------------------------------------------------

\newcommand{\ra}{\frac}
  % rapport
\newcommand{\dra}{\dfrac}
  % rapport en displaystyle
\newcommand{\qqs}{\forall}
    % quel que soit ...

\newcommand{\implique}{\;\Longrightarrow\;}
\newcommand{\vide}{\varnothing}
  % l'ensemble vide de "amssymb"

\newcommand{\et}{\mbox{ et }}
\newcommand{\ie}{{}{\emph{i.e.}}}
\newcommand{\etc}{{}{\emph{etc.}}}
\newcommand{\cad}{\hbox{c.-\`a-d.}}


%----------------------------------------------------------
% Les inégalités
%----------------------------------------------------------

\renewcommand{\leq}{\leqslant}
\renewcommand{\geq}{\geqslant}

%--------------------------------------------------
% Ensembles, opérateurs ensemblistes, suites
%--------------------------------------------------

\newcommand{\union}{\mathrel{\cup}}
\newcommand{\Union}{\mathrel{\bigcup}}
\newcommand{\inter}{\mathrel{\cap}}
\newcommand{\Inter}{\mathrel{\bigcap}}
\newcommand{\Sum}{\mathop{\sum}\limits}

\newcommand{\card}{\mathop{\mathrm{card}}\nolimits}
  % cardinal d'un ensemble

\newcommand{\ens}[3][]{\mathopen#1\{ #2\mathbin#1/ #3 \mathclose#1\}}
   % définit un ensemble { x / P(x) }

\newcommand{\sym}[1][n]{\mathfrak{S}_{#1}}
  % groupe symétrique d'ordre n

\DeclareMathSymbol{\complement}{\mathord}{AMSa}{"7B}
  % complément d'un ensemble
\newcommand{\prive}{\setminus}
  % "E\A" se code $E\prive A$, "privé de"

\newcommand{\rond}{\circ}
  % symbole de composition des applications

\newcommand{\combinaison}[2]{\mbox{\large\bf\sf C}_{#1}^{#2}}
  % définit les coefficients binomiaux
\newcommand{\comb}{\combinaison}

% n-uplets--------------------------------------------------

\newcommand{\Dots}{,\ldots,}
\newcommand{\nuple}[2][]{#1( #2_1,\ldots,#2_n #1)}
  % n-uplet (x_1,...,x_n) avec x=#2 et parenthèses extensibles
\newcommand{\puple}[2][]{#1( #2_1,\ldots,#2_p #1)}
  % p-uplet (x_1,...,x_p) avec x=#2 et parenthèses extensibles
\newcommand{\Nuple}[3][]{#1( #2_1,\ldots,#2_#3 #1)}
  % uplet (x_1,...,x_q) avec x=#2, q=#3 et parenthèses extensibles

% suites--------------------------------------------------
\newcommand{\suite}[2][]{#1( #2_n #1)_n}
  % suite (x_n)_n avec x=#2 et parenthèses extensibles
\newcommand{\ssuite}[3][]{#1( #2_#3 #1)_n}
  % suite (x_q)_q avec x=#2, q=#3 et parenthèses extensibles
\newcommand{\Suite}[2][]{#1( \vc{#2}_n #1)_n}
  % suite de vecteurs (x_n)_n avec x=#2 et parenthèses extensibles
\newcommand{\SSuite}[3][]{#1( \vc{#2}_#3 #1)_n}
  % suite de vecteurs (x_q)_q avec x=#2, q=#3 et parenthèses extensibles

% séries--------------------------------------------------
\newcommand{\serie}[1]{\sum#1_n}
\newcommand{\Serie}[2]{\sum(#1_n+#2_n)}


%--------------------------------------------------
% Les ensembles de nombres
%--------------------------------------------------

%\newcommand{\N}{\mathsf{N}}
%\newcommand{\Z}{\mathsf{Z}}
%\newcommand{\Q}{\mathsf{Q}}
%\newcommand{\R}{\mathsf{R}}
%\newcommand{\C}{\mathsf{C}}
%\newcommand{\K}{\mathsf{K}}
%\newcommand{\U}{\mathsf{U}}

\newcommand{\N}{\mathbb{N}}
\newcommand{\Z}{\mathbb{Z}}
\newcommand{\Q}{\mathbb{Q}}
\newcommand{\R}{\mathbb{R}}
\newcommand{\C}{\mathbb{C}}
\newcommand{\K}{\mathbb{K}}
\newcommand{\U}{\mathbb{U}}

\newcommand{\Nbb}{\mathbb{N}}
\newcommand{\Zbb}{\mathbb{Z}}
\newcommand{\Qbb}{\mathbb{Q}}
\newcommand{\Rbb}{\mathbb{R}}
\newcommand{\Cbb}{\mathbb{C}}
\newcommand{\Kbb}{\mathbb{K}}
\newcommand{\Ubb}{\mathbb{U}}

\newcommand{\ccro}[2]{\{#1,\dots,#2\}}

%--------------------------------------------------
% Nombres complexes
%--------------------------------------------------
\newcommand{\conjug}[1]{\overline{#1}}
  % conjuguaison
\newcommand{\IM}{\mathop{\Im\mathrm{m}}\nolimits}
  % partie imaginaire
\newcommand{\RE}{\mathop{\Re\mathrm{e}}\nolimits}
  % partie réelle

%--------------------------------------------------
% Limites
%--------------------------------------------------

\def\buildrel#1_#2^#3{\mathrel{\mathop{\kern 0pt#1}\limits_{#2}^{#3}}}
  % définit une macro pour mettre #2 en dessous et #3 au dessus de #1
\newcommand{\tend}[1][n]{\buildrel{\longrightarrow}_{#1}^{}}
  % -> avec un "n" dessous par défaut
\newcommand{\tendpas}[1][n]{\buildrel{\not\longrightarrow}_{#1}^{}}
  % /-> avec un "n" dessous par défaut
\newcommand{\equivalent}[1][n]{\buildrel{\sim}_{#1}^{}}
  % le signe "équivalent" avec un "n" dessous par défaut
\newcommand{\egal}[1][n]{\buildrel{=}_{#1}^{}}
  % le signe "=" avec un "n" dessous par défaut

%--------------------------------------------------
% Dérivée et intégrale
%--------------------------------------------------

\DeclareMathOperator{\D}{D}
  % pour la dérivation

\newcommand{\sub}[1]{\sigma_{\vc#1}}
  % subdivision
\newcommand{\intd}{\int\!\!\!\int}
  % intégrale double
\newcommand{\intdepi}{\ra1{2\pi}\int_0^{2\pi}}
  % intégrale entre 0 et 2\pi
\newcommand{\intab}{\int_{\intf ab}}
  % intégrale sur le segment [a,b]

\newcommand{\dt}[1][t]{\mathrm{d}\mspace{-0.5mu}#1}
  % élément différentiel dans une intégrale, dt par défaut
  
\newcommand{\entre}[3][\big]{#1\rvert_{#2}^{#3}}
  % accroissement d'une fonction entre #2 et #3, \big par défaut


\newcommand{\del}[2]{\ra{\partial#1}{\partial#2}}
  % dérivée partielle : del f/del x
\newcommand{\ddel}[3]{\ra{\partial^2#1}{\partial#2\,\partial#3}}
  % dérivée partielle seconde : del^2 f/del x del y
\newcommand{\Del}[3]{\ra{\partial^#1#2}{\partial#3^#1}}
\newcommand{\dd}[2]{\ra{\mathrm{d}#1}{\mathrm{d}#2}}
\DeclareMathOperator{\J}{J}
  %jacobien


%--------------------------------------------------
% Définition des vecteurs
%--------------------------------------------------

\newcommand{\Vect}[1]{\vbox{\halign{##\cr 
  \tiny\rightarrowfill\cr\noalign{\nointerlineskip\vskip1pt} 
  $#1\mskip2mu$\cr}}}
  % vecteur avec une flèche de hauteur constante
\newcommand{\vect}[1]{\vbox{\halign{##\cr 
  \tiny\rightarrowfill\cr\noalign{\nointerlineskip\vskip1pt} 
  $#1\mskip2mu$\cr}}}
  % vecteur avec une flèche

\newcommand{\vc}[1]{\mathbf{#1}}
  % vecteur écrit en gras
\newcommand{\bvc}[1]{\boldsymbol{#1}}
  % vecteur écrit en gras, cas des grecs

\newcommand{\grad}{\mathop{\vect{\mathrm{grad}}}\nolimits}
  % le vecteur gradient
\DeclareMathOperator{\divergence}{div}
\newcommand{\diver}{\divergence}
  % l'opérateur divergence

%--------------------------------------------------
% Diverses normes
%--------------------------------------------------

\newcommand{\abs}[2][]{\mathopen#1\lvert #2 \mathclose#1\rvert}
  % valeur absolue...
\newcommand\PS[2]{\langle#1,#2\rangle}
  % produit scalaire
  
\newcommand{\norme}[2][]{\mathopen#1\lVert #2 \mathclose#1\rVert}
  % norme d'une fonction
\newcommand{\norm}{\norme}

\newcommand{\normeun}[2][]{\mathopen#1\lVert #2 \mathclose#1\rVert_1}
  % norme d'une fonction
\newcommand{\normu}{\normeun}

\newcommand{\normedeux}[2][]{\mathopen#1\lVert #2 \mathclose#1\rVert_2}
  % norme d'une fonction
\newcommand{\normd}{\normedeux}

\newcommand{\normeinfinie}[2][]{\mathopen#1\lVert #2 \mathclose#1\rVert_\infty}
  % norme d'une fonction
\newcommand{\normi}{\normeinfinie}

\newcommand{\Norme}[1][N]{\mathcal{#1}}

\DeclareMathOperator{\dist}{d}
  % distance

\newcommand{\Bo}[3][]{\mathcal{B}#1(\vc{#2},#3 #1)}
  % Boule ouverte de centre #2 et de rayon #3, avec parenthèses variables
\newcommand{\Bf}[3][]{\mathcal{B}_f#1(\vc{#2},#3 #1)}
  % Boule fermée de centre #2 et de rayon #3, avec parenthèses variables

%--------------------------------------------------
% Produit scalaire
%   JLQ le 29/06/98
%--------------------------------------------------

\newcommand{\scal}[3][]{#1\langle #2 \mathrel{#1\vert} #3 #1\rangle}
    % le produit scalaire
\newcommand{\Scal}[3][]{#1\langle \vc{#2} \mathrel{#1\vert} \vc{#3} #1\rangle}
    % le produit scalaire et des vecteurs écrits en gras

%--------------------------------------------------
% Ecriture des intervalles
%--------------------------------------------------

\newcommand{\into}[3][]{\mathopen{#1]}#2,#3\mathclose{#1[}}
  % intervalle ouvert ]#2;#3[
\newcommand{\intof}[3][]{\mathopen{#1]}#2,#3 \mathclose#1\rbrack}
  % intervalle ouvert-fermé ]#2;#3]
\newcommand{\intfo}[3][]{\mathopen#1\lbrack #2,#3 \mathclose{#1[}}
  % intervalle fermé-ouvert [#2;#3[
\newcommand{\intf}[3][]{\mathopen#1\lbrack  #2,#3 \mathclose#1\rbrack}
  % intervalle fermé [#2,#3] 

\newcommand{\Intf}[3][]{\mathopen#1\lbrack\!#1\lbrack#2,#3 \mathclose#1\rbrack\!#1\rbrack}
       %intervalle fermé [|#2;#3|] pour les nombres entiers

%--------------------------------------------------
% Les fonctions, les espaces fonctionnels
%--------------------------------------------------

\DeclareMathOperator{\ent}{Ent}
  % partie entière
\DeclareMathOperator{\Arg}{Arg}
  % détermination principale de l'argument d'un nombre complexe
\newcommand{\restr}[2][]{#1\rvert_{#2}}
  % restriction d'une fonction à #2

\newcommand{\oo}[2][]{\mathrm{o}#1(#2#1)}
  % notation de Landau
\newcommand{\OO}[2][]{\mathrm{O}#1(#2#1)}
  % notation de Landau


% Fonctions arcsinus, ... --------------------------------------

\DeclareMathOperator{\ch}{ch}
\DeclareMathOperator{\sh}{sh}
%\DeclareMathOperator{\th}{th}

\DeclareMathOperator{\cn}{cn}
\DeclareMathOperator{\sn}{sn}
\DeclareMathOperator{\dn}{dn}

\DeclareMathOperator{\argth}{Arg\,th}
\DeclareMathOperator{\argsh}{Arg\,sh}
%----------------------------------------------------------------------

% les fonctions ... --------------------------------------------------
\newcommand{\FIE}[1][I,E]{\mathcal{F}(#1)}
\newcommand{\FabE}[2][{a}{b}]{\mathcal{F}(\intf #1,#2)}
\newcommand{\Fab}[1][{a}{b}]{\mathcal{F}(\intf #1)}

% les fonctions bornées ... ------------------------------------------
\newcommand{\BIE}[1][I,E]{\mathcal{B}(#1)}
\newcommand{\BabE}[2][{a}{b}]{\mathcal{B}(\intf #1,#2)}
\newcommand{\Bab}[1][{a}{b}]{\mathcal{B}(\intf #1)}

% les fonctions en escalier ... --------------------------------------
\newcommand{\EscIE}[1][I,E]{\mathcal{E}sc(#1)}
\newcommand{\EscabE}[2][{a}{b}]{\mathcal{E}sc(\intf #1,#2)}
\newcommand{\Escab}[1][{a}{b}]{\mathcal{E}sc(\intf #1)}

% les fonctions continues ... -----------------------------------------
\newcommand{\CIE}[1][I,E]{\mathcal{C}(#1)}
\newcommand{\CI}[1][I]{\mathcal{C}(#1)}
\newcommand{\CabE}[2][{a}{b}]{\mathcal{C}(\intf #1,#2)}
\newcommand{\Cab}[1][{a}{b}]{\mathcal{C}(\intf #1)}

% les fonctions continues par morceaux ... ----------------------------
\newcommand{\CMIE}[1][I,E]{\mathcal{CM}(#1)}
\newcommand{\CMI}[1][I]{\mathcal{CM}(#1)}
\newcommand{\CMplI}[1][I]{\mathcal{CM}^{+}(#1)}
\newcommand{\CMabE}[2][{a}{b}]{\mathcal{CM}(\intf #1,#2)}
\newcommand{\CMab}[1][{a}{b}]{\mathcal{CM}(\intf #1)}

% les fonctions sommables ... -----------------------------------------
\newcommand{\LI}[2][1]{\mathcal{L}^{#1}(#2)}
\newcommand{\LCI}[2][1]{\mathcal{L}_{\mathcal{C}}^{#1}(#2)}

% les fonctions dérivables ... ----------------------------------------
\newcommand{\DIE}[1][I,E]{\mathcal{D}(#1)}
\newcommand{\DabE}[2][{a}{b}]{\mathcal{D}(\intf #1,#2)}
\newcommand{\Dab}[1][{a}{b}]{\mathcal{D}(\intf #1)}

% les fonctions de classe Ck ... --------------------------------------
\newcommand{\CkIE}[2][I,E]{\mathcal{C}^{#2}(#1)}
\newcommand{\CkabE}[3][{a}{b}]{\mathcal{C}^{#3}(\intf #1,#2)}
\newcommand{\Ckab}[2][{a}{b}]{\mathcal{C}^{#2}(\intf #1)}

% les fonctions k fois dérivables ... ---------------------------------
\newcommand{\DkIE}[2][I,E]{\mathcal{D}^{#2}(#1)}
\newcommand{\DkabE}[3][{a}{b}]{\mathcal{D}^{#3}(\intf #1,#2)}
\newcommand{\Dkab}[2][{a}{b}]{\mathcal{D}^{#2}(\intf #1)}

%les fonctions de classe Ck par morceaux ... --------------------------
\newcommand{\CMkIE}[2][I,E]{\mathcal{C}^{#2}\mahtcal{M}(#1)}
\newcommand{\CMkabE}[3][{a}{b}]{\mathcal{C}^{#3}\mathcal{M}(\intf #1,#2)}
\newcommand{\CMkab}[2][{a}{b}]{\mathcal{C}^{#2}\mathcal{M}(\intf #1)}

% les fonctions périodiques ... ----------------------------------------
\newcommand{\Cdepi}{\mathcal{C}_{2\pi}}
\newcommand{\CMdepi}{\mathcal{CM}_{2\pi}}
\newcommand{\Tndepi}{\mathcal{T}_{n,2\pi}}
\newcommand{\CT}{\mathcal{C}_{T}}
\newcommand{\Tdepi}{\mathcal{T}_{2\pi}}
\newcommand{\TT}{\mathcal{T}_{T}}

%--------------------------------------------------
% Les matrices,
%--------------------------------------------------

\newcommand{\mat}{\mathrm{Mat}}
  % matrice Mat dans la base #1 (B)
\newcommand{\Mat}[2]{\mathcal{M}\mathrm{at}_{\mathcal{#1},\mathcal{#2}}}
  % matrice dans les bases #1 et #2
\newcommand{\diag}[2][]{\mathrm{Diag}#1( #2_1,\ldots,#2_n #1)}
  % matrice diagonale d'ordre n
\newcommand{\Diag}[3][]{\mathrm{Diag}#1( #2_1,\ldots,#2_{#3} #1)}
  % matrice diagonale d'ordre #2




\newcommand{\transposee}[1]{#1^{\mathsf {T}}}
    % nouvelle version;
    % attention, les textes doivent être revus pour
    % l'utiliser.
  %transposition d'une matrice
\newcommand{\trans}{\mbox{${}^t\hspace{-1pt}$}} % ancienne version de jlq
  % transposition
\newcommand{\tc}[1]{\mbox{${}^t\hspace{-1pt}$}\conjug{#1}}
  % transposition et conjugaison
\DeclareMathOperator{\com}{Com}
  % comatrice ou matrice des cofacteurs


\DeclareMathOperator{\Vectt}{Vect}
% Espace vectoriel engendré
%\DeclareMathOperator{\sp}{sp}
\renewcommand{\sp}{\mathop{\mathrm{sp}}\nolimits}
  % spectre d'une matrice, ...
\DeclareMathOperator{\rg}{rg}
\DeclareMathOperator{\Rang}{rg}
  % rang d'une matrice, ...
\DeclareMathOperator{\Ker}{Ker}
\DeclareMathOperator{\im}{Im}
\DeclareMathOperator{\Ima}{Im}
  % image d'une application linéaire
\DeclareMathOperator{\tr}{tr}
\DeclareMathOperator{\Tr}{tr}
  % trace d'une matrice, ...
%\DeclareMathOperator{\deter}{det}
\renewcommand{\det}{\mathop{\mathrm{det}}\nolimits}
  % déterminant, \det est "mal défini" chez AmsTeX!
\newcommand{\Det}[1][B]{\det_{\mathcal{#1}}}
  % déterminant dans la base B ...
\newcommand{\M}[3]{\mathcal{M}_{#1,#2}(#3)}
\newcommand{\Mn}[2]{\mathcal{M}_{#1}(#2)}

\newcommand{\MnK}{\Mn{n}{\K}}
\newcommand{\MnpK}{\M{n}{p}{\K}}
\newcommand{\GLnK}{\mathcal{GL}_n(\K)}
\newcommand{\GLn}[2][n]{\mathcal{GL}_{#1}(#2)}
\newcommand{\In}{I_n}
  % trace d'une matrice, ...
%--------------------------------------------------
% Espaces vectoriels
%--------------------------------------------------

\newcommand{\somdir}{\mathop{\oplus}\nolimits}
\newcommand{\Somdir}{\mathop{\oplus}\limits}
%\newcommand{\somDir}{\mathop{\bigoplus}\nolimits}
%\newcommand{\SomDir}{\mathop{\bigoplus}\limits}

\newcommand{\lin}[1]{\mathcal{L}(#1)}
\newcommand{\Lin}[2]{\mathcal{L}(#1,#2)}

\newcommand{\LEF}[1][E,F]{\mathcal{L}(#1)}
\newcommand{\LE}[1][E]{\mathcal{L}(#1)}
\newcommand{\LK}[2][K]{\mathcal{L}_{\#1}(#2)}

\newcommand{\SE}[1][E]{\mathscr{S}(#1)}
  % ensemble des endomorphismes symétriques sur #1, E par défaut
\newcommand{\ASE}[1][E]{\mathscr{A}(#1)}
  % ensemble des endomorphismes antisymétriques sur #1, E par défaut
\newcommand{\SnR}[1][\R]{\mathscr{S}_n(#1)}
  % ensemble des matrices symétriques d'ordre n sur #1, \R par défaut
\newcommand{\AnR}[1][\R]{\mathscr{A}_n(#1)}
  % ensemble des matrices antisymétriques d'ordre n sur #1, \R par défaut


\newcommand{\GLE}[1][E]{\mathcal{GL}(#1)}
  % groupe des automorphismes de #1, par défaut E
\newcommand{\GLK}[2][K]{\mathcal{GL}_{\#1}(#2)}

\newcommand{\OrE}[1][E]{\mathcal{O}(#1)}
  % groupe des automorphismes orthogonaux de #1, par défaut E
\newcommand{\SOE}[1][E]{\mathcal{SO}(#1)}
  % groupe spécial orthogonal de #1, par défaut E
\newcommand{\OpE}[1][E]{\mathcal{O}^+(#1)}
  % groupe spécial orthogonal de #1, par défaut E
\newcommand{\OmE}[1][E]{\mathcal{O}^-(#1)}
  % ensemble des automorphismes orthogonaux de det=-1 de #1, par défaut E
\newcommand{\OnR}[1][n]{\mathcal{O}_{#1}(\R)}
  % groupe des automorphismes orthogonaux de #2^#1, par défaut #2^n

\newcommand{\SOnR}[1][n]{\mathcal{SO}_{#1}(\R)}
  % groupe spécial orthogonal d'ordre #1, par défaut n
\newcommand{\Opn}[1][n]{\mathcal{O}^+(#1)}
  % groupe spécial orthogonal d'ordre #1, par défaut n
\newcommand{\Omn}[1][n]{\mathcal{O}^-(#1)}
  % ensemble des matrices orthogonales de det=-1, d'ordre #1, par défaut n


\newcommand{\LpEF}{\mathcal{L}_{p}(E,F)}
\newcommand{\Lp}[2][p]{\mathcal{L}_{#1}(#2)}

%---------------------------------------------------------------------- from yann
\newcommand\Fonction[5]{{#1}\left|\begin{aligned}{#2}&\;\longrightarrow\;{#3}\\{#4}&\;\longmapsto\;{#5}\end{aligned}\right.}
\newcommand\ninf{{n\to \infty}}
\DeclareMathOperator*\PetitO{o}
\DeclareMathOperator*\GrandO{O}
\DeclareMathOperator*\Sim{\sim}




\usepackage{tkz-tab}
\usepackage{pgf}
\usetikzlibrary{arrows}
\usetikzlibrary{patterns}  
\definecolor{CyanTikz40}{cmyk}{.4,0,0,0}
\definecolor{CyanTikz20}{cmyk}{.2,0,0,0}
\tikzstyle{general}=[line width=0.3mm, >=stealth, x=1cm, y=1cm,line cap=round, line join=round]
\tikzstyle{quadrillage}=[line width=0.3mm, color=CyanTikz40]
\tikzstyle{quadrillageNIV2}=[line width=0.3mm, color=CyanTikz20]
\tikzstyle{quadrillage55}=[line width=0.3mm, color=CyanTikz40, xstep=0.5, ystep=0.5]
\tikzstyle{cote}=[line width=0.3mm, <->]
\tikzstyle{epais}=[line width=0.5mm, line cap=butt]
\tikzstyle{tres epais}=[line width=0.8mm, line cap=butt]
\tikzstyle{axe}=[line width=0.3mm, ->, color=Noir, line cap=rect]
\newcommand{\quadrillageSeyes}[2]{\draw[line width=0.3mm, color=A1!10, ystep=0.2, xstep=0.8] #1 grid #2;
\draw[line width=0.3mm, color=A1!30, xstep=0.8, ystep=0.8] #1 grid #2; }
\newcommand{\axeX}[4][0]{\draw[axe] (#2,#1)--(#3,#1); \foreach \x in {#4} {\draw (\x,#1) node {\small $+$}; \draw (\x,#1) node[below] {\small $\x$};}}
\newcommand{\axeY}[4][0]{\draw[axe] (#1,#2)--(#1,#3); \foreach \y in {#4} {\draw (#1, \y) node {\small $+$}; \draw (#1, \y) node[left] {\small $\y$};}}
\newcommand{\axeOI}[3][0]{\draw[axe] (#2,#1)--(#3,#1);  \draw (1,#1) node {\small $+$}; \draw (1,#1) node[below] {\small $I$};}
\newcommand{\axeOJ}[3][0]{\draw[axe] (#1,#2)--(#1,#3); \draw (#1, 1) node {\small $+$}; \draw (#1, 1) node[left] {\small $J$};}
\newcommand{\axeXgraduation}[2][0]{\foreach \x in {#2} {\draw (\x,#1) node {\small $+$};}}
\newcommand{\axeYgraduation}[2][0]{\foreach \y in {#2} {\draw (#1, \y) node {\small $+$}; }}
\newcommand{\origine}{\draw (0,0) node[below left] {\small $0$};}
\newcommand{\origineO}{\draw (0,0) node[below left] {$O$};}
\newcommand{\point}[4]{\draw (#1,#2) node[#4] {$#3$};}
\newcommand{\pointGraphique}[4]{\draw (#1,#2) node[#4] {$#3$};
\draw (#1,#2) node {$+$};}
\newcommand{\pointFigure}[4]{\draw (#1,#2) node[#4] {$#3$};
\draw (#1,#2) node {$\times$};}
\newcommand{\pointC}[3]{\draw (#1) node[#3] {$#2$};}
\newcommand{\pointCGraphique}[3]{\draw (#1) node[#3] {$#2$};
\draw (#1) node {$+$};}
\newcommand{\pointCFigure}[3]{\draw (#1) node[#3] {$#2$};
\draw (#1) node {$\times$};}
\definecolor{A1}                {cmyk}{1.00, 0.00, 0.00, 0.50}
\definecolor{A2}                {cmyk}{0.60, 0.00, 0.00, 0.10}
\definecolor{A3}                {cmyk}{0.30, 0.00, 0.00, 0.05}
\definecolor{A1}                {cmyk}{1.00, 0.00, 0.00, 0.50}
\definecolor{A2}                {cmyk}{0.60, 0.00, 0.00, 0.10}
\definecolor{A3}                {cmyk}{0.30, 0.00, 0.00, 0.05}
\definecolor{A4}                {cmyk}{0.10, 0.00, 0.00, 0.00}
\definecolor{B1}                {cmyk}{0.00, 1.00, 0.60, 0.40}
\definecolor{B2}                {cmyk}{0.00, 0.85, 0.60, 0.15}
\definecolor{B3}                {cmyk}{0.00, 0.20, 0.15, 0.05}
\definecolor{B4}                {cmyk}{0.00, 0.05, 0.05, 0.00}
\definecolor{C1}                {cmyk}{0.00, 1.00, 0.00, 0.50}
\definecolor{C2}                {cmyk}{0.00, 0.60, 0.00, 0.20}
\definecolor{C3}                {cmyk}{0.00, 0.30, 0.00, 0.05}
\definecolor{C4}                {cmyk}{0.00, 0.10, 0.00, 0.05}
\definecolor{D1}                {cmyk}{0.00, 0.00, 1.00, 0.50}
\definecolor{D2}                {cmyk}{0.20, 0.20, 0.80, 0.00}
\definecolor{D3}                {cmyk}{0.00, 0.00, 0.20, 0.10}
\definecolor{D4}                {cmyk}{0.00, 0.00, 0.20, 0.05}
\definecolor{F1}                {cmyk}{0.00, 0.80, 0.50, 0.00}
\definecolor{F2}                {cmyk}{0.00, 0.40, 0.30, 0.00}
\definecolor{F3}                {cmyk}{0.00, 0.15, 0.10, 0.00}
\definecolor{F4}                {cmyk}{0.00, 0.07, 0.05, 0.00}
\definecolor{G1}                {cmyk}{1.00, 0.00, 0.50, 0.00}
\definecolor{G2}                {cmyk}{0.50, 0.00, 0.20, 0.00}
\definecolor{G3}                {cmyk}{0.20, 0.00, 0.10, 0.00}
\definecolor{G4}                {cmyk}{0.10, 0.00, 0.05, 0.00}
\definecolor{H1}                {cmyk}{0.40, 0.00, 1.00, 0.10}
\definecolor{H2}                {cmyk}{0.20, 0.00, 0.50, 0.05}
\definecolor{H3}                {cmyk}{0.10, 0.00, 0.20, 0.00}
\definecolor{H4}                {cmyk}{0.07, 0.00, 0.15, 0.00}
\definecolor{J1}                {cmyk}{0.00, 0.50, 1.00, 0.00}
\definecolor{J2}                {cmyk}{0.00, 0.20, 0.50, 0.00}
\definecolor{J3}                {cmyk}{0.00, 0.10, 0.20, 0.00}
\definecolor{J4}                {cmyk}{0.00, 0.07, 0.15, 0.00}
\definecolor{FondOuv}           {cmyk}{0.00, 0.05, 0.10, 0.00}
\definecolor{FondAutoEvaluation}{cmyk}{0.00, 0.03, 0.15, 0.00}
\definecolor{FondTableaux}      {cmyk}{0.00, 0.00, 0.20, 0.00}
\definecolor{FondAlgo}          {cmyk}{0.07, 0.00, 0.30, 0.00}
\definecolor{BleuOuv}           {cmyk}{1.00, 0.00, 0.00, 0.00}
\definecolor{PartieFonction}    {cmyk}{1.00, 0.00, 0.00, 0.00}
\definecolor{PartieGeometrie}   {cmyk}{0.80, 0.80, 0.00, 0.00}
\definecolor{PartieStatistique} {cmyk}{0.60, 0.95, 0.00, 0.20}
\definecolor{PartieStatistiqueOLD} {cmyk}{0.95, 0.60, 0.20, 0.00}
\definecolor{PartieStatistique*}{cmyk}{0.30, 1.00, 0.00, 0.00}
\definecolor{U1}                {cmyk}{0.50, 0.10, 0.00, 0.10}
\definecolor{U2}                {cmyk}{0.20, 0.15, 0.00, 0.00}
\definecolor{CouleurRenvoiManuelNumerique}{cmyk}{0.40, 0.00, 0.00, 0.00}
\definecolor{Blanc}             {gray}{1.00}
\definecolor{gris1}             {gray}{0.80}
\definecolor{gris2}             {gray}{0.60}
\definecolor{gris3}             {gray}{0.40}
\definecolor{Noir}              {gray}{0.00}

\begin{document}


\chapter*{Vecteurs}
\section{Translations - Vecteurs associés}
\begin{Activite}[Télécabine]
Une télécabine se déplace le long d'un câble de A vers B :\\
\begin{tikzpicture}
\coordinate (A) at (3,5); 
\coordinate (B) at (9,7); 
\pointCFigure{A}{A}{above};
\pointCFigure{B}{B}{above};
\draw [quadrillage,loosely dashed] (0,0) grid (12,8);
\draw (0,4) -- (12,8);
\draw[epais] (3,5) -- (3,4) -- (4,4) -- (4,1) -- (2,1) -- (2,4) -- (3,4);
\end{tikzpicture}\\
Dessiner ci-dessus la télécabine lorsqu'elle sera arrivée au terminus B.\\
On appelle ce déplacement une $\dots\dots\dots\dots\dots$  de A vers B. \\

\begin{Correction}\newline
\begin{tikzpicture}
\coordinate (A) at (3,5); 
\coordinate (B) at (9,7); 
\pointCFigure{A}{A}{above};
\pointCFigure{B}{B}{above};
\draw [quadrillage,loosely dashed] (0,0) grid (12,8);
\draw (0,4) -- (12,8);
\draw[epais]     (3,5) -- (3,4) -- (4,4) -- (4,1) -- (2,1) -- (2,4) -- (3,4);
\draw[epais,A1] (9,7) -- (9,6) -- (10,6) -- (10,3) -- (8,3) -- (8,6) -- (9,6);
\draw [->, epais, color=colordef] (A)--(B);  
\draw [->, epais, color=colordef] (2,4)--(8,6); 
\draw [->, epais, color=colordef] (4,4)--(10,6); 
\draw [->, epais, color=colordef] (4,1)--(10,3);
\draw [->, epais, color=colordef] (2,1)--(8,3);  
\end{tikzpicture}\\
Déplacer une figure par \defi{translation}, c'est faire glisser cette figure sans la faire tourner. Pour décrire ce déplacement, il faut donc donner la \defi{direction}, le \defi{sens} et la \defi{longueur} de ce parcours. Pour cela, on va
utiliser un nouvel outil mathématique : \defi{vecteur}. Sur la figure, 
\end{Correction}
\end{Activite}

\begin{Df}[Translation]
Soit $A$ et $B$ deux points du plan. \\
On appelle \defi{translation qui transforme $A$ en $B$} la transformation qui, à tout point $M$ du plan, associe
l'unique point $M'$ tel que $[ AM' ]$ et $[ BM ]$ ont même milieu.
\begin{center}
\begin{tikzpicture}[general, scale=0.4]
\draw [quadrillage] (0,0) grid (11,8);
\coordinate (A) at (1,1); 
\coordinate (B) at (3,5); 
\coordinate (M) at (7,2); 
\coordinate (N) at (9,6);
\coordinate (F) at (10.6, 7);
\draw [loosely dashed] (A)--(N);
\draw [loosely  dashed] (B)--(M); 
\draw [->, epais, color=A1] (A)--(B); 
\draw [->, epais, color=A1] (M)--(N); 
\pointC{A}{A}{below left}
\pointC{B}{B}{above left}
\pointC{M}{M}{below right}
\pointC{N}{M'}{above}
\draw (3,2.25) node[rotate=122, color=F1]  {{\boldmath $\approx$}};
\draw (7,4.75) node[rotate=122, color=F1]  {{\boldmath $\approx$}};
\draw (4,4.25) node[rotate=53, color=C1]  {{\boldmath $\infty$}};
\draw (6,2.75) node[rotate=53, color=C1]  {{\boldmath $\infty$}};
\end{tikzpicture}
%\begin{tikzpicture}[general, scale=0.4]
%\draw [quadrillage]  (0,0) grid (16,8);
%\coordinate (D) at (1,1); 
%\coordinate (A) at (2.2,1.48); 
%\coordinate (B) at (5.8,2.92); 
%\coordinate (M) at (8.5,4); 
%\coordinate (N) at (12.1,5.44);
%\coordinate (F) at (16, 7);
%\draw [loosely dashed] (D)--(F);
%\draw [->, epais, color=A1, line cap=butt] (A)--(B); 
%\draw [->, epais, color=A1, line cap=butt] (M)--(N); 
%\pointC{A}{A}{below left}
%\pointC{B}{B}{above left}
%\pointC{M}{M}{below right}
%\pointC{N}{M'}{above}
%\end{tikzpicture}
\end{center}
\end{Df}

\begin{Voc}
\begin{itemize}
\item Le point $M'$ est appelé \defi{image} du point $M$.
\item On dit également que $M$ est le \defi{translaté} de $M'$.
\end{itemize}
\end{Voc}

\begin{NB}
Une \textbf{transformation} sert à \textbf{modéliser} mathématiquement un mouvement.
\begin{itemize}
\item La \textbf{symétrie centrale} est la transformation qui modélise le demi-tour. 
\item La \textbf{translation} est la transformation qui modélise le glissement rectiligne. Pour la définir, on indique la direction, le sens et la longueur du mouvement. 
\end{itemize}
\end{NB}



\begin{Prop}[Parallélogramme] On considère quatre points $A$,   $B$, $C$ et $D$. \\
La translation qui transforme $A$ en $B$ transforme $C$ en $D$, si et seulement si  $ABDC$ est un \propri{parallélogramme} (éventuellement aplati).  
\begin{center}
\begin{tikzpicture}[general, scale=0.4]
\draw [quadrillage] (0,0) grid (11,8);
\coordinate (A) at (1,1); 
\coordinate (B) at (3,5); 
\coordinate (M) at (7,2); 
\coordinate (N) at (9,6);
\coordinate (F) at (10.6, 7);
\draw [loosely dashed] (A)--(N);
\draw [loosely  dashed] (B)--(M); 
\draw [->, epais, color=A1] (A)--(B); 
\draw [->, epais, color=A1] (M)--(N);
\draw [dashed,  color=A1] (B)--(N); 
\draw [dashed, color=A1] (A)--(M);  
\pointC{A}{A}{below left}
\pointC{B}{B}{above left}
\pointC{M}{C}{below right}
\pointC{N}{D}{above}
%\draw (N) arc (32:52:3);
%\draw (N) arc (32:12:3);
\draw (3,2.25) node[rotate=122, color=F1]  {{\boldmath $\approx$}};
\draw (7,4.75) node[rotate=122, color=F1]  {{\boldmath $\approx$}};
\draw (4,4.25) node[rotate=53, color=C1]  {{\boldmath $\infty$}};
\draw (6,2.75) node[rotate=53, color=C1]  {{\boldmath $\infty$}};
\end{tikzpicture}
\end{center}
\end{Prop}

\begin{Proof}
C'est la conséquence de la propriété: \textit{un quadrilatère est un parallélogramme si et seulement si ses diagonales se coupent en leur milieu}\fg.
\end{Proof}


\begin{Df}[Vecteurs associés]
\`A chaque translation est associé un \defi{vecteur}. \\
Pour $A$ et $B$ deux points, le \defi{vecteur} $\Vect{AB}$ est associé à la translation qui transforme $A$ en $B$. \\
Le vecteur $\Vect{AB}$ est défini par :
\begin{enumerate}
\item la \defi{direction} (celle de la droite $(AB)$), 
\item le \defi{ sens} (de $A$ vers $B$)
\item la \defi{ longueur} $AB$. 
\end{enumerate}
$A$ est l'\defi{origine} du vecteur et $B$ son \defi{extrémité}.
\end{Df}



\begin{Df}[Egalité entre vecteurs]
Deux vecteurs qui définissent la même translation sont dits \defi{égaux}.\\
\parbox{0.50\textwidth}{Deux vecteurs égaux ont : 
\begin{enumerate}
\item même direction; \item même sens; \item même longueur.
\end{enumerate}
}\hfill\parbox{0.50\textwidth}{
\begin{center}
\begin{tikzpicture}[general, scale=0.75]
\draw [quadrillage, step=0.5] (-3,0.5) grid (4,3);
\draw [->] (-2,2) --node[above] {$\vect{AB}$} (1,1);
\draw [->] (-0,2.5) --node[above] {$\vect{CD}$} (3,1.5);
\draw[color=black] (-2,2) node[above left] {$A$};
\draw[color=black] (1,1) node[below right] {$B$};
\draw[color=black] (-0,2.5) node[above left] {$C$};
\draw[color=black] (3,1.5) node[below right] {$D$};
\end{tikzpicture}\\
$\Vect{AB}=\Vect{CD}$
\end{center}
}
\end{Df}



\begin{Propriete}[Parallélogramme]
$\Vect{AB}=\Vect{CD}$ si et seulement si $ABDC$ est un parallélogramme (éventuellement aplati).
\begin{center}
\begin{tikzpicture}[general, scale=0.75]
\draw [quadrillage, step=0.5] (-3,0.5) grid (4,3);
\draw [->] (-2,2) --node[above] {$\vect{AB}$} (1,1);
\draw [->] (-0,2.5) --node[above] {$\vect{CD}$} (3,1.5);
\draw [dashed] (-2,2) -- (-0,2.5);
\draw [dashed] (3,1.5) -- (1,1);
\draw[color=black] (-2,2) node[above left] {$A$};
\draw[color=black] (1,1) node[below right] {$B$};
\draw[color=black] (-0,2.5) node[above left] {$C$};
\draw[color=black] (3,1.5) node[below right] {$D$};
\end{tikzpicture}
\end{center}
\end{Propriete}

\begin{Meth} [Construire un vecteur] 
Soit  $A$, $B$ et $C$ trois points non alignés.\\
Pour \defi{placer} le point $D$ tel que $\Vect{CD}=\Vect{AB}$, on construit le parallélogramme $ABDC$.
\begin{center}
\begin{tikzpicture}[general, scale=0.5]
\draw [quadrillage] (-4,-2) grid (5,5);
\draw[dashed] (-3,-1)-- (2,0);
\draw[->, epais] (-3,-1)-- (-1,3);
\draw[dashed] (-1,3)-- (4,4);
\draw[->, epais] (2,0)-- (4,4);
\draw (-3,-1) node[below] {$A$};
\draw (-1,3) node[above] {$B$};
\draw (2,0) node[below] {$C$};
\draw (4,4) node[right] {$D$};
\end{tikzpicture}
\end{center}
\end{Meth}



\begin{NB}
Une translation peut être définie par un point quelconque et son translaté.\\ Il existe donc une  \textbf{infinité} de vecteurs associés à une translation. Ils sont tous égaux.  \\
Le vecteur choisi pour définir la translation est un \textbf{représentant} de tous ces vecteurs. \\ La translation  \textbf{ne dépend pas} du représentant choisi pour la définir. 
On le note souvent $\Vec u$.
\end{NB}

\begin{Df}[Vecteur nul]
Le vecteur associé à la translation qui transforme un point quelconque en lui-même\\ est le \defi{vecteur nul}{}, noté $\Vect 0$.\\ Ainsi, 
$\Vect{AA} = \Vect{BB}  = \Vect{CC}= \ldots =\Vect{0} $ 
\end{Df}

\begin{Df}[Vecteur opposé]
Le vecteur $\Vect{BA}$ de la translation qui transforme $B$ en $A$ est appelé \defi{vecteur opposé}{} à $\Vect{AB}$. 
\end{Df}
\begin{NB}
\begin{itemize}
\item Le vecteur opposé à $\Vect{AB}$ se note $-\Vect{AB}$ et on a l'égalité $\Vect{BA}=-\Vect{AB}$.
\item La notation $\overleftarrow{AB}$ n'existe pas.
\end{itemize}
\end{NB}

\begin{NB} \par Deux vecteurs  \textbf{opposés} ont même direction, même longueur mais sont de sens contraires.
\end{NB}

\section{Opérations sur les vecteurs}

\subsection{Additions}

\begin{Propriete}[Enchaînement de translations]
 L'enchaînement de deux translations est également une translation.
\end{Propriete}


\begin{DfProp}[Relation de Chasles]
Soit $A$, $B$, $C$ trois points. \\
L'encha\^inement de la translation de vecteur $\Vect{AB}$ puis de la translation de vecteur $\Vect{BC}$  est la translation de vecteur $\Vect{AC}$ et on a : $$\Vect{AB}+\Vect{BC}=\Vect{AC}.$$
Le vecteur $\Vect{AC}$ est le vecteur \defi{somme}.
\begin{center}
\begin{tikzpicture}[general, scale=0.5]
\draw [quadrillage] (-4,-2) grid (5,5);
\draw[->,epais] (-3,-1)--node[below]{$\Vect{AB}$}  (2,0);
\draw[->,epais] (2,0)--node[right]{$\Vect{BC}$} (4,4);
\draw[->,epais] (-3,-1)--node[above,left]{$\Vect{AB}+\Vect{BC}=\Vect{AC}$} (4,4);
\draw (-3,-1) node[below] {$A$};
\draw (2,0) node[below] {$B$};
\draw (4,4) node[right] {$C$};
\end{tikzpicture}
\end{center}
\end{DfProp}
%\begin{NB}
% $\Vect{AB}+\Vect{BA}=\Vect{AA}=\Vec 0$.
%\end{NB}
%
%
%\begin{Propriete}Soit $\Vec {AB}$ et $\Vec {CD}$ deux vecteurs. Alors : 
%\begin{itemize}
%\item  $\Vec {AB} +\Vec {CD}=\Vec {CD} +\Vec {AB}$ \item $\Vec {AB} + \Vec 0 = \Vec 
%{AB}$
%\end{itemize}
%\end{Propriete}

\begin{Prop}[Diagonale du parallélogramme]
\parbox{0.60\textwidth}{
Soit $A$, $B$, $C$, $D$ quatre points. \\
$\Vect{AD}=\Vect{AB}+\Vect{AC}$ si et seulement si $ABDC$ est un parall\'elogramme. 
}\hfill
\parbox{0.35\textwidth}{
\begin{center}
\begin{tikzpicture}[general, scale=0.5]
\draw [quadrillage] (-4,-2) grid (5,5);
\draw[->, epais] (-3,-1)--node[below]{$\Vect{AC}$}  (2,0);
\draw[->, epais] (-3,-1)--node[left]{$\Vect{AB}$} (-1,3);
\draw[->, epais] (-3,-1)--node[left]{$\Vect{AD}$} (4,4);
\draw[dashed] (-1,3)-- (4,4);
\draw[dashed] (2,0)-- (4,4);
\draw (-3,-1) node[below] {$A$};
\draw (-1,3) node[above] {$B$};
\draw (2,0) node[below] {$C$};
\draw (4,4) node[right] {$D$};
\end{tikzpicture}
\end{center}
}
\end{Prop}

\begin{Proof}
\begin{align*}
\Vect{AD}=&\Vect{AB}+\Vect{AC}&\\
\Leftrightarrow \Vect{AC}+\Vect{CD}=&\Vect{AB}+\Vect{AC}&\text{d'après la relation de Chasles}\\
\Leftrightarrow \Vect{CD}=&\Vect{AB}\\
 \Leftrightarrow ABDC \text{est un parall\'elogramme} 
\end{align*}
\end{Proof}


\begin{Meth}[Construire la somme de deux vecteurs]\label{2G3_M_construire_somme}
On remplace l'un des deux vecteurs par un représentant: 
\begin{itemize}
\item soit de même origine afin d'utiliser la règle du parallélogramme;
\item soit d'origine l'extrémité de l'autre afin d'utiliser la relation de Chasles.
\end{itemize}
%\vspace{-0.5em}
%\begin{enumerate}
%\item  Construire un carré $ABCD$ de centre $O$.
%\item Construire les vecteurs \par
%$\Vec u=\Vect{AB}+\Vect{OD}$ et 
%
%$\Vec v=\Vect{AD}+\Vect{OC}$
%\end{enumerate}
%Correction :
%\hfill
%\begin{tikzpicture}[general]
%  \draw (0,0) node[below] {$A$};
%  \draw (2,0) node[below] {$B$};
%  \draw (2,2) node[above] {$C$};
%  \draw (0,2) node[above] {$D$};
%  \draw (1,1) node[right] {$O$};
%  \draw[color=B1] (0.5,0.5) node[above] {{\boldmath $\Vec u$}};
%  \draw (1,-1) node {Avec la relation};
%  \draw (1,-1.5) node {de Chasles};
%  \draw (1,3) node[below, white] {$A$};
%\draw (0,0)--(2,0)--(2,2)--(0,2)--cycle;
%\draw (0,0)--(2,2);
% \draw[->, color=A1, epais] (0,0)--(2,0);
% \draw[->, color=A1, epais] (1,1)--(0,2);
% \draw[->, color=A1, epais, dashed, line cap=butt] (2,0)--(1,1);
% \draw[->, color=B1, epais, line cap=butt] (0,0)--(1,1);
%\end{tikzpicture} 
%\hfill
%\begin{tikzpicture}[general]
%  \draw (0,0) node[below] {$A$};
%  \draw (2,0) node[below] {$B$};
%  \draw (2,2) node[above] {$C$};
%  \draw (0,2) node[above] {$D$};
%  \draw (1,1) node[right] {$O$};
%\draw (0,0)--(2,0)--(2,2)--(0,2)--cycle;
%\draw[dashed] (1,1)--(1,3)--(0,2);
%\draw (0,2)--(2,0);
%\draw[->, color=A1, epais] (0,0)--(0,2);
%\draw[->, color=A1, epais, line cap=butt] (1,1)--(2,2);
%\draw[->, color=A1, epais, dashed] (0,0)--(1,1);
%\draw[->, color=B1, epais] (0,0)--(1,3);
%\draw[color=B1] (1,3) node[right] {$\Vec v$};
%\draw (1,-1) node {Avec la règle};
%\draw (1,-1.5) node { du parallélogramme};
%\end{tikzpicture} 
\end{Meth}


\begin{NB} 
$\Vect{AB}+\Vect{AC}=\Vec 0$ si et seulement si $A$ est le milieu du segment $[BC]$.
\end{NB}


\subsection{Soustraction}

\begin{Df} \defi{Soustraire un vecteur}{}, c'est additionner son oppos\'e.
\end{Df}

\begin{Ex}
Soit trois points $A$, $B$ et $C$ non align\'es. \\
Donner un repr\'esentant du vecteur $\Vect u =\Vect{AB}-\Vect{AC}$.\\
On a :\\
$\Vect u=\Vect{AB}-\Vect{AC}$\\
$\Vect u=\Vect{AB}+\Vect{CA}$\\
$\Vect u=\Vect{CA}+\Vect{AB}$\\
$\Vect u =\Vect{CB}$ en utilisant la relation de Chasles.
\end{Ex}



\section{Coordonnées d'un vecteur}


\begin{Df}[Coordonnées d'un vecteur]
\parbox{0.55\textwidth}{
Dans un rep\`ere $(O;I,J)$,  on considère la translation de vecteur $\Vect u$ qui translate l'origine $O$ en un point $M$ de coordonnées $(a;b)$. \\
Les \defi{coordonnées du vecteur} $\Vect{u}$ sont les coordonn\'ees du point  $M$ tel que  $\Vec u =\Vect{OM}$.\\
On note $\Vec u\left( \begin{array}{c} a\\b\end{array}\right)$.
}\hfill\parbox{0.40\textwidth}{
\begin{tikzpicture}[general, scale=0.5]
\draw [quadrillage] (-2,-2) grid (8,9);
\foreach \x/\y/\P/\N in {6/0/below/a,0/5/left/b} 
{\draw[color=C1] (\x,\y) node {\small  $+$};
\draw[color=C1]  (\x,\y) node[\P] {{\boldmath $\N$}};}
\draw[->,color=black] (-2,0) -- (8,0);
\draw[->,color=black] (0,-2) -- (0,9);
\origineO
\draw [dashed, color=F1, epais] (0,5) -- (6,5);
\draw [dashed, color=J1, epais] (6,0) -- (6,5);
\foreach \x/\y/\P/\N in {1/0/below/I,0/1/left/J,6/5/right/M} 
{\draw (\x,\y) node {\small  $+$};
\draw (\x,\y) node[\P] {$\N$};}
\draw [->, color=A1, tres epais, line cap=butt] (0,0) --node[left,above] {{\boldmath $\Vect{OM}$}} (6,5);
\draw [->, color=A1, tres epais, line cap=butt] (1,3) --node[left] {{\boldmath $\Vect u$}} (7,8);
\end{tikzpicture}
}
\end{Df}


\begin{Prop}[\'Egalité des coordonnées]
 Deux vecteurs sont égaux si et seulement si ces vecteurs ont les mêmes coordonnées.
\end{Prop}


\begin{Prop}[Coordonnées de $\Vect{AB}$]
Dans un repère $(O; I,J)$, les  coordonnées du vecteur $\Vect{AB}$ sont $\left( \begin{array}{c} x_B-x_A\\y_B-y_A\end{array}\right)$.
\end{Prop}

\begin{Proof}
 Soit $A$, $B$ et $M$ de coordonnées respectives $(x_A; y_A)$, $(x_B; y_B)$ et $(x_M; y_M)$ dans un repère $(O; I, J)$ tels que $\Vect{OM}=\Vect{AB}$ et $OMBA$ est un parallélogramme.\\
Donc $[AM]$ et $[OB]$ ont même milieu.
    $\renewcommand{\arraystretch}{2}\left \lbrace \begin{array}{c}
                      \dfrac{x_A+x_M}{2}=\dfrac{x_B+x_O}{2}\\
                      \dfrac{y_A+y_M}{2}=\dfrac{y_B+y_O}{2}
                     \end{array} \right.\renewcommand{\arraystretch}{1}
$   soit  $\left \lbrace \begin{array}{c}
                     x_A+x_M=x_B\\
                    y_A+y_M=y_B
                     \end{array} \right.
$ soit $\left \lbrace \begin{array}{c}
                     x_M=x_B-x_A\\
                    y_M=y_B-y_A
                     \end{array} \right.
$
\end{Proof}

\begin{Meth}[Lire les coordonnées d'un vecteur]
\label{2G3_M_lire_coord}
  Lire les coordonnées du vecteur $\Vec u$ sur la figure ci-dessous. 
  \vspace{-0.5em}
\begin{center}
\begin{tikzpicture}[general, yscale=0.6]
\foreach \a/\b/\c/\d in {0/0/5/0, 0/0.87/5/0.87, 0/2.61/5/2.61, 0/3.48/5/3.48, 0/1.74/1/3.48, 1/0/3/3.48, 2/0/4/3.48, 3/0/5/3.48, 4/0/5/1.74} {\draw[quadrillage] (\a,\b)--(\c, \d);}
\draw[axe] (0,0)--(2,3.48); 
\draw[axe] (0,1.74)--(5,1.74);
\coordinate (I) at (2,1.74); 
\coordinate (J) at (1.5,2.61); 
\draw (I) node[rotate=-25] {{\boldmath $/$}} node[below]{$I$};
\draw (J) node {{\boldmath $-$}} node[left]{$J$};
\draw[epais, ->, line cap=butt, color=F1] (0.5,2.61)--(3,0);
\draw (1.25,1.305) node[below] {{\boldmath $\Vect{u}$}};
\end{tikzpicture}
\end{center}
Les coordonnées de $\Vec u$ sont $\left(\begin{array}{c}4 \\ -3
\end{array}\right)
$ car 
\begin{center}
\begin{tikzpicture}[general, yscale=0.6]
\foreach \a/\b/\c/\d in {0/0/5/0, 0/0.87/5/0.87, 0/2.61/5/2.61, 0/3.48/5/3.48, 0/1.74/1/3.48, 1/0/3/3.48, 2/0/4/3.48, 3/0/5/3.48, 4/0/5/1.74} {\draw[quadrillage] (\a,\b)--(\c, \d);}
\draw[axe] (0,0)--(2,3.48); 
\draw[axe] (0,1.74)--(5,1.74);
\coordinate (I) at (2,1.74); 
\coordinate (J) at (1.5,2.61); 
\draw (I) node[rotate=-25] {{\boldmath $/$}} node[below]{$I$};
\draw (J) node {{\boldmath $-$}} node[above left]{$J$};
\draw[epais, ->, line cap=butt, color=F1] (0.5,2.61)--(3,0);
\draw (1.25,1.305) node[below, color=F1] {{\boldmath $\Vect{u}$}};
\draw[epais, ->, line cap=butt, color=A1] (0.5,2.61)--(4.5,2.61);
\draw (2.5,2.61) node[above, color=A1] {{\boldmath $+4$}};
\draw[epais, ->, line cap=butt, color=A1] (4.5,2.61)--(3,0);
\draw (3.75,1.305) node[right, color=A1] {{\boldmath $-3$}};
\end{tikzpicture}
\end{center}
\end{Meth}

 
\begin{Meth}[Construire un vecteur à partir de ses coordonnées]
 Dans un repère orthonormé, construire le représentant d'origine $A(6; 2)$ du vecteur $\Vec u$ de coordonnées $\left( \begin{array}{c} -4 \\3\end{array}\right)$.  \\
 On a :
\begin{tikzpicture}[general, scale=0.4]
\draw [quadrillage] (-2,-2) grid (8,6);
\axeOI{-2}{8}
\axeOJ{-2}{6}
\origineO
\draw [<-, color=A1, tres epais] (2,5) -- (6,2);
\draw [<-,dashed, color=F1, epais] (2,2) -- (6,2);
\draw [<-,dashed, color=J1, epais] (2,5) -- (2,2);
\draw[color=black, color=F1] (4,2) node[below] {{\boldmath $-4$}};
\draw[color=black, color=J1] (2,3.5) node[left] {{\boldmath $+3$}};
\draw[color=black, color=A1] (4.5,4) node {{\boldmath $\Vec u$}};
\pointGraphique{6}{2}{A}{below right}
\end{tikzpicture}
\end{Meth}



\begin{Meth}[Repérer un point défini par une égalité vectorielle]
 Dans un repère orthogonal $(O;I,J)$, on a les points $A(-2;3)$, $B(4;-1)$ et $C(5;3)$. \\Calculer les coordonnées 
 \begin{enumerate}
  \item du vecteur $\Vect{AB}$;
  \item du point $D$ tel que $\Vect{AB}=\Vect{CD}$.
 \end{enumerate}
On a:
 \begin{enumerate}
  \item Les coordonnées du  vecteur $\Vect{AB}$ sont 
$\left( \begin{array}{c}
                    x_B-x_A \\ y_B-y_A
                   \end{array}
\right)$ soit $\left( \begin{array}{c}
                    4-(-2) \\ -1-3
                   \end{array}
\right)$.
Donc $\Vect{AB}$ a pour coordonnées $\left( \begin{array}{c}
                    6 \\ -4
                   \end{array}
\right)$.

\item On cherche $(x_D;y_D)$, les coordonnées du 
point $D$ tel que $\Vect{AB}=\Vect{CD}$.\\ 
Or, \og  \textit{ si deux vecteurs sont égaux 
alors ils ont mêmes coordonnées}\fg. \\
Donc le couple $(x_D;y_D)$ est la solution du système: 


$\left\lbrace \begin{array}{c}
                    x_D-x_C=x_B-x_A \\ y_D-y_C=y_B-y_A
                   \end{array}\right.$ 
soit  $\left \lbrace \begin{array}{l}
                                                        x_D-5=6 \\y_D-3=-4
                                                       \end{array}
\right.$
soit  $\left \lbrace \begin{array}{l}
                                                        x_D=6+5=11 \\y_D=-4+3=-1
                                                       \end{array}
\right.$
\vspace{0.1em}
\par Les coordonnées du point $D$ sont $(11;-1)$.
 \end{enumerate}
\end{Meth}



\begin{Prop}[Somme de deux vecteurs]
Si $\Vect{u}$ et $\Vect{v}$ sont deux vecteurs de coordonn\'ees respectives $\left (\begin{array}{c}
x\\y
\end{array}\right)$ et 
$\left(\begin{array}{c}
x'\\y'
\end{array}\right)$,
alors les coordonn\'ees du \propri{vecteur somme}, $\Vect{u}+\Vect{v}$, sont 
$\left(\begin{array}{c}
x+x'\\y+y'
\end{array}\right)$.
\end{Prop}

\begin{Meth}[Rep\'erer un point d\'efini par une somme vectorielle]
\label{2G3_M_repere_point_somme}
Dans un rep\`ere orthogonal $(O; I, J)$, on place les points $A(2;3)$, $B(4;-1)$, $C(5;3)$ et $D(-2;-1)$.
Quelles sont les coordonn\'ees du point $E$ tel que $\Vect{AE}=\Vect{AD}+\Vect{CB}$?


On cherche les coordonn\'ees $(x_E;y_E)$ du point $E$ tel que  $\Vect{AE}=\Vect{AD}+\Vect{CB}$. \\
Donc le couple $(x_E;y_E)$ est solution du syst\`eme:

$\left\lbrace\begin{array}{l}
x_E-x_A=(x_D-x_A)+(x_B-x_C)\\y_E-y_A=(y_D-y_A)+(y_B-y_C)
\end{array}\right.$
soit 
$\left \lbrace \begin{array}{l}
x_E-2=(-2-2)+(4-5)\\y_E-3=(-1-3)+(-1-3)
\end{array}
\right.$ 
soit 
$\left \lbrace \begin{array}{l}
x_E-2=-5 \\y_E-3=-8
\end{array}
\right.$ 
soit 
$\left \lbrace \begin{array}{l}
x_E=-5+2=-3 \\y_E=-8+3=-5
\end{array}
\right.$
\par
Les coordonn\'ees du point $E$ sont $(-3;-5)$.
\end{Meth}




\section{Multiplication par un réel}


\begin{Df}
Soit $\Vect{u}$ un vecteur de coordonn\'ees $(x;y)$ et $\lambda$ un r\'eel. \\
La \defi{multiplication de {\boldmath $\Vect{u}$} par {\boldmath $\lambda$}}{} est le vecteur \defi{$\lambda\Vect{u}$} de coordonn\'ees \defi{$(\lambda x;\lambda y)$}.
\end{Df}

\begin{Meth}[Rep\'erer le produit d'un vecteur par un réel]
\label{2G3_M_reperer_produit}
 Dans un rep\`ere orthogonal, construire le repr\'esentant d'origine $A(1; 4)$ du vecteur $-0,5 \Vec u$ avec $\Vec u \left( \begin{array}{c} 2 \\-3\end{array}\right)$.\\
 On a : \\\parbox{0.49\linewidth}{
$\Vec u$ a pour coordonn\'ees $\left( \begin{array}{c} 2 \\-3\end{array}\right)$. 
Donc $-0,5 \Vec u$ a pour coordonn\'ees $\left( \begin{array}{c} -0,5\times 2 \\-0,5\times (-3)\end{array}\right)$  
soit $\left( \begin{array}{c} -1 \\1,5\end{array}\right)$. 
}\hfill\parbox{0.49\linewidth}{
\begin{tikzpicture}[general, yscale=0.75, xscale=0.75]
\draw [quadrillage, xstep=1] (-1,-1) grid (4,7);
\axeX{-1}{4}{1}
\axeY{-1}{7}{1}
\draw [->, color=A1,epais] (1,4) -- (3,1);
\draw [->, color=C1,epais] (1,4) -- (0,5.5);
\draw [->,dashed, color=A1, epais](1,4) -- (1,5.5);
\draw [->,dashed, color=C1, epais] (1,5.5)--(0,5.5);
\draw[color=C1] (0.5,5.5) node[above] {{\boldmath $-1$}};
\draw[color=A1] (1,4.75) node[right] {{\boldmath $+1,5$}};
\draw[color=C1] (0.5,4) node[left] {{\boldmath $-0,5\Vec u$}};
\draw[color=A1] (2,3) node[right] {{\boldmath $\Vec u$}};
\pointGraphique{1}{4}{A}{below}
\end{tikzpicture}}
\end{Meth}

\begin{Prop}[Sens en fonction du signe de $\lambda$]
Soient deux vecteurs $\Vect{AB}$ et $\Vect{CD}$ et $\lambda$ un r\'eel tels que  $\Vect{AB}=\lambda\Vect{CD}$. 
\begin{itemize}
 \item si $\lambda>0$, $\Vect{AB}$ et $\Vect{CD}$ sont de m\^eme sens.
\item si $\lambda<0$, $\Vect{AB}$ et $\Vect{CD}$ sont de sens contraires.
\end{itemize}
\end{Prop}

\begin{NB}
 $\Vec u$ et $\lambda \Vec u$ ont la m\^eme direction.  Leurs sens et leurs longueurs d\'ependent de $\lambda$.
\end{NB}


\section{Colin\'earit\'e}


\begin{Df}[Colinéaire]
On dit que deux vecteurs non nuls sont \defi{colinéaires} si  leurs coordonn\'ees dans un m\^eme rep\`ere sont proportionnelles. 
\end{Df}

 \begin{NB}
Par convention, le vecteur nul est colin\'eaire \`a tout vecteur $\Vect{u}$. %En effet, $\Vect{0}=0.\Vect{u}$.
 \end{NB}



\begin{Prop}
Deux vecteurs $\Vect{u}$ et $\Vect{v}$ non nuls sont colin\'eaires lorsqu'il existe un r\'eel $\lambda$ tel que $\Vect{v}=\lambda\Vect{u}$.
\end{Prop}
 

 
\begin{Meth}[V\'erifier la colin\'earit\'e de deux vecteurs]\label{2G3_M_colinearite}
Pour v\'erifier que  deux vecteurs non nuls $\Vect{u}\left (\begin{array}{c}
x\\y
\end{array}\right)$ et $\Vect{v}\left(\begin{array}{c}
x'\\y'
\end{array}\right)$  sont colin\'eaires, il suffit de:
\begin{description}
\item[possibilité 1] trouver un r\'eel $\lambda$ non nul tel que $x'=\lambda x$ et $y'=\lambda y$; 
\item[possibilité 2] vérifier que les produits en croix, $xy'$  et $x'y$, sont égaux.
\end{description}

Soit $(O; I, J)$ un rep\`ere orthogonal. Les vecteurs suivants sont-ils colin\'eaires? 
\begin{enumerate}
\item $\Vec u \left( \begin{array}{c}
2\\6 
\end{array}
\right)$ et $\Vec v \left( \begin{array}{c}
-6 \\ -18
\end{array}
\right)$. 
\item $\Vec w \left( \begin{array}{c}
-5 \\ 3\end{array}\right)$ et $\Vec z \left( \begin{array}{c}12 \\ -7\end{array}\right)$.
\end{enumerate}
On a :
\begin{enumerate}
\item $-6=-3\times 2$ et $-18=-3\times 6$ donc $\Vec v=-3\Vec u$.
 $\Vec u$ et $\Vec v$ sont donc colin\'eaires.
\item $-5\times(-7)=35$ et $3\times 12=36$.  Les produits en croix ne sont pas \'egaux.  
  Donc $\Vec w$ et $\Vec z$ ne sont pas colin\'eaires.
\end{enumerate}
\end{Meth}


\begin{Prop}[Caractérisation vectorielle des droites parallèles]
\begin{itemize}
\item Deux droites $(AB)$ et $(CD)$ sont \textbf{parall\`eles} si et seulement si les vecteurs $\Vect{AB}$ et $\Vect{CD}$ sont colin\'eaires;
\item Trois points $A$, $B$ et $C$ sont \textbf{align\'es} si et seulement si les vecteurs $\Vect{AB}$ et $\Vect{AC}$ sont colin\'eaires.
\end{itemize}
\end{Prop}
\begin{Ex}
Les points $A(1;2 )$, $B(3;1 )$ et $C(5; 3)$  sont-ils alignés ?\\
On calcule les coordonnées des vecteurs $\Vect{AB}$ et $\Vect{AC}$ puis les produits en croix :
$$(x_B-x_A)(y_C-y_A)=(3-1)(3-2)=2\times 1=2.$$
et ainsi : \\
$$(y_B-y_A)(x_C-x_A)=(1-2)(5-1)=-1\times4=-4.$$
Les coordonnées des vecteurs $\Vect{AB}$ et $\Vect{AC}$ ne sont pas proportionnelles. Donc $A$, $B$, $C$ ne sont pas alignés.
\end{Ex}

\end{document}
