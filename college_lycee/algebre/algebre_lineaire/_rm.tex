\documentclass{book} 
\usepackage{jlq2eams} 


\newenvironment{Correction}{\textbf{\newline Correction } }{}
\begin{document}
\chapter*{Vecteurs}
\section{Translations - Vecteurs associ�s}
\begin{Activite}[T�l�cabine]
Une t�l�cabine se d�place le long d'un c�ble de A vers B :\\
\begin{tikzpicture}
\coordinate (A) at (3,5); 
\coordinate (B) at (9,7); 
\pointCFigure{A}{A}{above};
\pointCFigure{B}{B}{above};
\draw [quadrillage,loosely dashed] (0,0) grid (12,8);
\draw (0,4) -- (12,8);
\draw[epais] (3,5) -- (3,4) -- (4,4) -- (4,1) -- (2,1) -- (2,4) -- (3,4);
\end{tikzpicture}\\
Dessiner ci-dessus la t�l�cabine lorsqu'elle sera arriv�e au terminus B.\\
On appelle ce d�placement une $\dots\dots\dots\dots\dots$  de A vers B. \\
D�placer une figure par translation, c'est faire glisser cette figure sans la faire tourner. Pour d�crire ce d�placement, il faut donc donner la \defi{direction}, le \defi{sens} et la \defi{longueur} de ce parcours. Pour cela, on va
utiliser un nouvel outil math�matique : les \defi{vecteurs}.
\begin{Correction}
ksksk
$$x^2$$
\end{Correction}

\end{Activite}



\end{document}
