\documentclass{book}
\usepackage{commeunjeustyle}
\begin{document}
\chapter*{Application linéaires : exercices}
\section{Applications linéaires}
\subsection{Proportionnalité} 
\begin{Exercice}[Dilution]
Quelle quantité d'alcool à 70°  dois-je mette dans  1L l'alcool a 90 ° pour  diluer  75 ° ?
\end{Exercice}

\subsection{Calcul matriciel} 
\begin{Exercice}[Des calculs de produits]
Calculer lorsqu'ils sont définis les produits $AB$ et $BA$ dans chacun des cas suivants :
\begin{enumerate}
\item $\displaystyle A= \left(\begin{array}{cc}
1  &  0  \\
0  &  0 \\
\end{array}\right),\quad 
B=\left(\begin{array}{cc}
0  &  0  \\
0  &  1 \\
\end{array}\right)
$
\item $\displaystyle A=\left(\begin{array}{ccc}
0 & 2 & 1\\
1 & 1 & 0\\
-1&-2 &-1\\
\end{array}\right),\quad
B=\left(\begin{array}{ccc}
2 & 0 & 1\\
-1& 1 & 2\\
\end{array}\right)
$
\item $\displaystyle A=\left(\begin{array}{cc}
1 & 2 \\
1 & 1 \\
0 & 3 \\
\end{array}\right),
\quad B=\left(\begin{array}{cccc}
-1 & 1 & 0& 1 \\
2 & 1 & 0& 0 \\
\end{array}\right)
$
\end{enumerate}
\end{Exercice}
\begin{Exercice}[Commutant]
Soient $a$ et $b$ des réels non nuls, et $A=\left( \begin{array}{cc} a & b\\ 
 0 &a \end{array} \right).$ Trouver toutes les matrices $B\in\mathcal M_2(\mathbb R)$ qui commutent avec $A$,
c'est-à-dire telles que $AB=BA$.

\end{Exercice}

\begin{Exercice}[Annulateur]

On considère les matrices $A=\left(\begin{array}{ccc}
1&0&0\\
0&1&1\\
3&1&1\end{array}\right)$, $B=\left(\begin{array}{ccc}
1&1&1\\
0&1&0\\
1&0&0\end{array}\right)$ et $C=\left(\begin{array}{ccc}
1&1&1\\
1&2&1\\
0&-1&-1\end{array}\right)$. Calculer $AB$, $AC$. Que constate-t-on? La matrice $A$ peut-elle être inversible? 
Trouver toutes les matrices $F\in\mathcal M_3(\mathbb R)$ telles que $AF=0$ (où $0$ désigne la matrice nulle).
\end{Exercice}

\begin{Exercice}[Produit non commutatif]

Déterminer deux éléments $A$ et $B$ de
$\mathcal M_2({\mathbb R})$ tels que : $AB=0$ et $BA\not = 0$.
\end{Exercice}


\begin{Exercice}[Matrices stochastiques en petite taille]
On dit qu'une matrice $A\in\mathcal M_n(\mathbb R)$ est une matrice stochastique si la somme des coefficients
sur chaque colonne de $A$ est égale à 1. Démontrer que le produit de deux matrices stochastiques est
une matrice stochastique si $n=2$. Reprendre la question si $n\leq 1$.
\end{Exercice}

\begin{Exercice}[Puissance $n$-ième, par récurrence]

Calculer la puissance $n$-ième des matrices suivantes : 
$$A=\left(\begin{array}{cc}
1&-1\\
-1&1\\
\end{array}\right),\ B=\left(\begin{array}{cc}
1&1\\
0&2\\
\end{array}\right).$$
\end{Exercice}

\begin{Exercice}[Puissance $n$-ième - avec la formule du binôme]

Soit $$A=\left(
\begin{array}{ccc}
1&1&0\\
0&1&1\\
0&0&1
\end{array}\right),\quad
I=\left(
\begin{array}{ccc}
1&0&0\\
0&1&0\\
0&0&1
\end{array}\right)\textrm{ et }
B=A-I.$$
Calculer $B^n$ pour tout $n\in\mathbb N$. En déduire $A^n$.
\end{Exercice}

\begin{Exercice}[Puissance $n$-ième - avec un polynôme annulateur]

\begin{enumerate}
\item Pour $n\geq 2$, déterminer le reste de la division euclidienne de $X^n$ par $X^2-3X+2$.
\item Soit  $A=\begin{pmatrix} 
0&1&-1\\
-1&2&-1\\
1&-1&2
\end{pmatrix}$. Déduire de la question précédente la valeur de $A^n$, pour $n\geq 2$.
\end{enumerate}
\end{Exercice}

\begin{Exercice}[Inverser une matrice  sans calculs!]

\begin{enumerate}
\item Soit $A=\left(
\begin{array}{ccc}
-1&1&1\\
1&-1&1\\
1&1&-1
\end{array}\right)$. Montrer que $A^2=2I_3-A$, en déduire que $A$ est inversible et calculer $A^{-1}$.
\item  Soit $ A=\begin{pmatrix} 1 & 0 & 2 \cr
0 & -1 & 1 \cr
1 & -2 & 0 \cr \end{pmatrix} .$ Calculer $
A^3-A .$ En déduire
que $ A $ est inversible puis déterminer $ A^{-1} .$ 
\item Soit $A=\begin{pmatrix} 
0&1&-1\\
-1&2&-1\\
1&-1&2
\end{pmatrix}$. Calculer $A^2-3A+2I_3$. En déduire que $A$ est inversible, et calculer $A^{-1}$.
\end{enumerate}
\end{Exercice}

\begin{Exercice}[Inverse avec calculs!]

Dire si les matrices suivantes sont inversibles et, le
cas échéant, calculer leur inverse :
$$A=\left(
\begin{array}{rcl}
1&1&2\\
1&2&1\\
2&1&1
\end{array}
\right),\quad
B=\left(
\begin{array}{rcl}
0&1&2\\
1&1&2\\
0&2&3
\end{array}
\right).$$
\end{Exercice}

\begin{Exercice}[Matrice nilpotente]

Soit $A\in\mathcal M_n(\mathbb R)$ une matrice nilpotente, c'est-à-dire qu'il existe $p\geq 1$ tel que $A^p=0$. Démontrer que la matrice $I_n-A$ est inversible, et déterminer son inverse.
\end{Exercice}



\subsection{Applications linéaires}
\begin{Exercice}[Applications linéaires ou non?]
Dire si les applications suivantes sont des applications linéaires :
\begin{enumerate}
\item $f:\mathbb R^2\to\mathbb R^3,\ (x,y)\mapsto (x+y,x-2y,0)$;
\item $f:\mathbb R^2\to\mathbb R^3,\ (x,y)\mapsto (x+y,x-2y,1)$;
\item $f:\mathbb R^2\to\mathbb R,\ (x,y)\mapsto x^2-y^2$;
\item $f:\mathbb R[X]\to \mathbb R^2,\ P\mapsto \big(P(0),P'(1)\big)$.
\end{enumerate}
\end{Exercice}

\begin{Exercice}[Définie par une base]

On considère dans $\mathbb R^2$ les trois vecteurs $u=(1,1)$, $v=(2,-1)$ et $w=(1,4)$.
\begin{enumerate}
\item Démontrer que $(u,v)$ est une base de $\mathbb R^2$.
\item Pour quelle(s) valeur(s) du réel $a$ existe-t-il une application linéaire $f:\mathbb R^2\to \mathbb R^2$
telle que $f(u)=(2,1)$, $f(v)=(1,-1)$ et $f(w)=(5,a)$?
\end{enumerate}
\end{Exercice}

\begin{Exercice}[Du local au global...]
Soit $E$ un espace vectoriel de dimension finie et $f\in\mathcal L(E)$.
On suppose que, pour tout $x\in E$, il existe un entier $n_x\in\mathbb N$ tel
que $f^{n_x}(x)=0.$ Montrer qu'il existe un entier $n$ tel que $f^n=0$.
\end{Exercice}

\subsection{Dualité}
\begin{Exercice}[Application linéaire définie sur les matrices]
Soient $A=\left(\begin{array}{cc}-1&2\\1&0\end{array}\right)$
et $f$ l'application de $M_2(\mathbb R)$ dans $M_2(\mathbb R)$
définie par $f(M)=AM$.
\begin{enumerate}
\item Montrer que $f$ est linéaire.
\item Déterminer sa matrice dans la base canonique de $M_2(\mathbb R)$.
\end{enumerate}
\end{Exercice}
\section{Images et noyaux}

\begin{Exercice}[Noyau et image]
Soit $f:\mathbb R^2\to\mathbb R^3$ l'application linéaire définie par
$$f(x,y)=(x+y,x-y,x+y).$$
Déterminer le noyau de $f$, son image. $f$ est-elle injective? surjective?
\end{Exercice}


\begin{Exercice}[Application linéaire donnée par l'image d'une base]
Soit $E=\mathbb R^3$. On note ${\cal B}=\{e_1,e_2,e_3\}$ la base canonique de $E$ et $u$ l'endomorphisme de $\mathbb R^3$ défini par la donnée des images des vecteurs de la base : 
$$u(e_1) = -2e_1 +2e_3 \; , u(e_2)=3e_2 \; , u(e_3)=-4e_1 + 4e_3.$$ 
\begin{enumerate}
\item Déterminer une base de $\ker~u$. 
$u$ est-il injectif? peut-il être surjectif? Pourquoi?
\item Déterminer une base de $\textrm{Im}~u$. Quel est le rang de u ?
\item Montrer que $E=\ker~u\bigoplus \textrm{Im}~u$.
\end{enumerate}
\end{Exercice}

\begin{Exercice}[Noyau prescrit ?]
Soit $E=\mathbb R^4$ et $F=\mathbb R^2$. On considère
$H=\{(x,y,z,t)\in\mathbb R^4;\ x=y=z=t\}$. Existe-t-il des applications linéaires de $E$ dans $F$
dont le noyau est $H$?
\end{Exercice}

\begin{Exercice}[A noyau fixé]
Soit $E$ le sous-espace vectoriel de $\mathbb R^3$ engendré par les vecteurs
$u=(1,0,0)$ et $v=(1,1,1)$. Trouver un endomorphisme $f$ de $\mathbb R^3$ dont le noyau est $E$.
\end{Exercice}

\begin{Exercice}[Application linéaire à contraintes]
Montrer qu'il existe un unique endomorphisme $f$ de $\mathbb R^4$ tel que, si 
$(e_1,e_2,e_3,e_4)$ désigne la base canonique, alors on a
\begin{enumerate}
\item $f(e_1)=e_1-e_2+e_3$ et $f(2e_1+3e_4)=e_2$.
\item $\ker(f)=\{(x,y,z,t)\in\mathbb R^4,\ x+2y+z=0\textrm{ et }x+3y-t=0\}.$
\end{enumerate}
\end{Exercice}

\begin{Exercice}[Espace vectoriel des polynômes de dimension infinie]
\begin{enumerate}
\item Montrer que l'application $\Fonction{\phi}{\K[X]}{\K\times\K[X]}{P}{(P(0),P')}$ est un isomorphisme.
\item En déduire que $\K[X]$ est de dimension infinie. 
\end{enumerate}
\end{Exercice}


\section{Matrices par blocs}
\begin{Exercice}[*, Trace du produit tensoriel de deux matrices]
Pour $A\in\mathcal M_n(\mathbb R)$ et $B\in\mathcal M_p(\mathbb R)$, on définit le produit tensoriel de $A$ et $B$ par 
$$A\otimes B=\left(\begin{array}{ccc}
a_{1,1}B&\dots&a_{1,n}B\\
\vdots&&\vdots\\
a_{n,1}B&\dots&a_{n,n}B
\end{array}\right).$$
Quelle est la taille de la matrice $A\otimes B$? Démontrer que $\textrm{tr}(A\otimes B)=\textrm{tr}(A)\textrm{tr}(B)$.
\end{Exercice}
\begin{Exercice}[**, Matrices de Walsh]
La suite de matrices de Walsh, $(W_n)_{n\in\N}$, est définie par :
$$W_0=(1)\text{ et }\forall n\in\N:\quad W_{n+1}=\begin{pmatrix}W_n & W_n\\W_n &-W_n
\end{pmatrix}.$$ 
Déterminer la taille de $W_n$ et calculer $w_n^2$, pour tout $n\in\N$.
\end{Exercice}



\begin{Exercice}[**, Déterminant d'une matrice triangulaire supérieur par blocs]
 On définit par blocs une matrice $A$ par $A=\left(
\begin{array}{cc}
B&D\\
0&C
\end{array}
\right)$ où $A$, $B$ et $C$ sont des matrices carrées de formats respectifs $n$, $p$ et $q$ avec $p+q=n$. Montrer que $\text{det}(A)=\text{det}(B)\times\text{det}(C)$.
\end{Exercice}
%
%
%
%
%\section*{Matrices}
%
%
%
%
%
%

%
%
%\Exercice[Explicite...]
%
%Calculer le rang des matrices suivantes :
%$$A=\left(
%\begin{array}{ccc}
%1&2&3\\
%2&3&4\\
%3&4&5
%\end{array}\right)\quad
%B=\left(
%\begin{array}{ccc}
%1&1&1\\
%1&2&4\\
%1&3&9
%\end{array}\right)$$
%$$
%C=\left(\begin{array}{cccc}
%1&2&3&2\\
%2&3&4&2\\
%3&4&5&2\\
%\end{array}
%\right)
%$$
%
%
%\Exercice[Avec un paramètre]
%
%Déterminer, suivant la valeur du réel $a$, le rang de la matrice suivante :
%$$A=\left(
%\begin{array}{cccc}
%1&a&a^2&a^3\\
%a&a^2&a^3&1\\
%a^2&a^3&1&a\\
%a^3&1&a&a^2
%\end{array}\right).$$
%
%
%
%
%
%
%
%\Exercice[Matrices et suites]
%
%Soient $(a_n)$, $(b_n)$ et $(c_n)$ trois suites réelles telles que $a_0=1$, $b_0=2$, $c_0=7$, et vérifiant les relations de récurrence :
%$$
%\left\{
%\begin{array}{rcccc}
%a_{n+1}&=&3a_n+&b_n&\\
%b_{n+1}&=&&3b_n+&c_n\\
%c_{n+1}&=&&&3c_n
%\end{array}
%\right.$$
%On souhaite exprimer $a_n$, $b_n$, et $c_n$ uniquement en fonction de $n$.
%\begin{enumerate}
%\item On considère le vecteur colonne $X_n=\left(\begin{array}{c}a_n\\b_n\\c_n\end{array}\right)$. Trouver une matrice $A$ telle que $X_{n+1}=AX_n$. En déduire que $X_n=A^n X_0$.
%\item Soit $N=\left(\begin{array}{ccc}0&1&0\\
%0&0&1\\
%0&0&0\end{array}\right)$. Calculer $N^2$, $N^3$, puis $N^p$ pour $p\geq 3$.
%\item Montrer que :
%$$A^n=3^{n}I+3^{n-1}nN+3^{n-2}\frac{n(n-1)}{2}N^2.$$
%\item En déduire $a_n$, $b_n$ et $c_n$ en fonction de $n$.
%\end{enumerate}
%
%
%\Exercice[Donnée par une matrice]
%
%On considère l'endomorphisme $f$ de $\mathbb R^3$ dont la matrice 
%dans la base canonique est :
%$$A=\left(
%\begin{array}{ccc}
%1&1&1\\
%-1&2&-2\\
%0&3&-1
%\end{array}\right).$$
%Donner une base de $\ker(f)$ et de $\textrm{Im}(f)$.
%
%
%\Exercice[Réduction]
%
%On considère l'endomorphisme $f$ de $\mathbb R^3$ dont la matrice 
%dans la base canonique est :
%$$M=\left(
%\begin{array}{ccc}
%1&1&-1\\
%-3&-3&3\\
%-2&-2&2
%\end{array}\right).$$
%Donner une base de $\ker(f)$ et de $\textrm{Im}(f)$. En déduire que $M^n=0$ pour tout $n\geq 2$.
%
%
%\Exercice[Changement de base]
%
%Soit $u$ l'application linéaire de $\mathbb R^3$ dans $\mathbb R^2$ dont la matrice dans leur base
%canonique respective est
%$$A=\left(
%\begin{array}{ccc}
%2&-1&1\\
%3&2&-3
%\end{array}\right).$$
%On appelle $(e_1,e_2,e_3)$ la base canonique de $\mathbb R^3$ et $(f_1,f_2)$ celle de $\mathbb R^2$. On pose
%$$e_1'=e_2+e_3,\ e_2'=e_3+e_1,\ e_3'=e_1+e_2$$
%$$f_1'=\frac{1}{2}(f_1+f_2),\ f_2'=\frac{1}{2}(f_1-f_2).$$
%\begin{enumerate}
%\item Montrer que $(e_1',e_2',e_3')$ est une base de $\mathbb R^3$ puis que $(f_1',f_2')$ est une base de
%$\mathbb R^2$.
%\item Quelle est la matrice de $u$ dans ces nouvelles bases?
%\end{enumerate}
%
%
%
%
%
%
%

%
%
%
%\section*{Déterminants}
%
%
%\Exercice[Pour s'échauffer...]
%
%\begin{enumerate}
%\item Calculer le déterminant suivant :
%$$\left|\begin{array}{cccc}
%1&1&1&1\\
%1&-1&1&1\\
%1&1&-1&1\\
%1&1&1&-1
%\end{array}\right|.$$
%\item Soit $E$ un $\mathbb R$-espace vectoriel et $f\in\mathcal L(E)$ tel que $f^2=-Id_E$. Que dire de la dimension de $E$?
%\end{enumerate}
%
%\Exercice[Divisible sans calculs!]
%Montrer, sans le calculer, que le déterminant suivant est divisible par 13 :
%$$\left|
%\begin{array}{ccc}
%5&2&1\\
%4&7&6\\
%6&3&9\\
%\end{array}
%\right|.$$
%
%
%\Exercice[Calcul sans développer]
%Montrer que $D=\left| 
%\begin{array}{ccc}
%1+a & a & a \\ 
%b & 1+b & b \\ 
%c & c & 1+c
%\end{array}
%\right| =1+a+b+c$ sans le développer.
%
%
%\Exercice[Sous forme factorisée]
%
%Calculer en mettant en évidence la factorisation le déterminant suivant :
%$$D=\left|\begin{array}{ccc}
%1&\cos a&\cos 2a\\
%1&\cos b&\cos 2b\\
%1&\cos c&\cos 2c
%\end{array}
%\right|.$$
%
%\Exercice[Tridiagonal]
%Soit $\Delta_n$ le déterminant de taille $n$ suivant : 
%$$\Delta_n=\left|
%\begin{array}{ccccc}
%3&1&0&\dots&0\\
%2&3&1&\ddots&\vdots\\
%0&2&3&\ddots&0\\
%\vdots&\ddots&\ddots&\ddots&1\\
%0&\dots&0&2&3
%\end{array}\right|.$$
%\begin{enumerate}
%\item Démontrer que, pour tout $n\geq 1$, on a $\Delta_{n+2}=3\Delta_{n+1}-2\Delta_n$.
%\item En déduire la valeur de  $\Delta_n$ pour tout $n\geq 1$.
%\end{enumerate}
%
%
%\Exercice[Calcul à l'aide d'une fonction affine]
%Soit $A=(a_{i,j})\in M_n(\mathbb R)$. On note $A(x)$ la matrice dont le terme général est $a_{i,j}+x$. 
%\begin{enumerate}
%\item Montrer que la fonction $x\mapsto \det(A(x))$ est une fonction polynômiale de degré inférieur ou égal à 1.
%\item Pour $a$ et $b$ deux réels distincts et $\alpha_1,\dots,\alpha_n\in\mathbb R$, en déduire la valeur du déterminant suivant
%$$\left|
%\begin{array}{cccc}
%\alpha_1&a&\dots&a\\
%b&\alpha_2&\ddots&\vdots\\
%\vdots&\ddots&\ddots&a\\
%b&\dots&b&\alpha_n
%\end{array}\right|.$$
%\end{enumerate}
%
%
%\Exercice[Imbriqué...]
%Soient $s_1,\dots,s_n\in\mathbb R$. Calculer le déterminant suivant :
%$$
%\left|
%\begin{array}{cccc}
%s_1&\dots&\dots&s_1\\
%\vdots&s_2&\dots&s_2\\
%\vdots&\vdots&\ddots&\vdots\\
%s_1&s_2&\dots&s_n
%\end{array}\right|.$$
%
%
%\Exercice[Tridiagonal]
%Soient $a,b,c$ des réels et $\Delta_n$ le déterminant de la matrice $n\times n$ suivant :
%$$\Delta_n=\left|\begin{array}{ccccc}
%a&b&0&\dots&0\\
%c&a&b&\ddots&\vdots\\
%0&\ddots&\ddots&\ddots&0\\
%\vdots&\ddots&\ddots&\ddots&b\\
%0&\dots&0&c&a
%\end{array}\right|.$$
%\begin{enumerate}
%\item Démontrer que, pour tout $n\geq 1$, on a $\Delta_{n+2}=a\Delta_{n+1}-bc\Delta_n.$
%\item On suppose que $a^2=4bc$. Démontrer que, pour tout $n\geq 1$, on a $\Delta_n=\frac{(n+1)a^n}{2^n}$.
%\end{enumerate}
%
%
%\Exercice[Pleins de $-1$!]
%Soit $A=(a_{i,j})_{1\leq i,j\leq n}\in\mathcal M_n(\mathbb K)$ et soit $B=(b_{i,j})_{1\leq i,j\leq n}$ définie par $b_{i,j}=(-1)^{i+j}a_{i,j}$. Calculer $\det(B)$ en fonction de $\det(A)$.
%
%
%\Exercice[Matrice compagnon]
%Soient $a_0,\dots,a_{n-1}$ $n$ nombres complexes et soit 
%$$A=\left( \begin{array}{ccccc}
%0&\dots&\dots&0&a_0\\
%1&\ddots&&\vdots&a_1\\
%0&\ddots&\ddots&\vdots&\vdots\\
%\vdots&\ddots&\ddots&0&\vdots\\
%0&\dots&0&1&a_{n-1}
%\end{array}\right).$$
%Calculer $\det(A-xI_n)$.
%
%
%\Exercice[ Sur des polynômes]
%Soit $u\in\mathcal L(\mathbb R_n[X])$. Calculer $\det(u)$ dans chacun des cas suivants :
%\begin{enumerate}
%\item $u(P)=P+P'$;
%\item $u(P)=P(X+1)-P(X)$;
%\item $u(P)=XP'+P(1)$.
%\end{enumerate}
%
%
%\Exercice[Inversibilité]
%Pour $\alpha\in\mathbb R$, on considère
%$$M_\alpha=\left(\begin{array}{ccc}
%1&3&\alpha\\
%2&-1&1\\
%-1&1&0
%\end{array}
%\right).$$
%Déterminer les valeurs de $\alpha$ pour lesquelles l'application linéaire associée à $M_\alpha$
%est bijective.
%
%
%\Exercice[Inversibilité d'une matrice à paramètres]
%\'Etudier, suivant la valeur du paramètre $a\in\mathbb R$ ou $m\in\mathbb R$, l'inversibilité des matrices suivantes :
%$$A=\left(\begin{array}{cccc}
%a&-1&0&-1\\
%-1&a&-1&0\\
%0&-1&a&-1\\
%-1&0&-1&a
%\end{array}\right)\textrm{ et }B=\left(\begin{array}{cccc}
%0&m&m&m^2-m\\
%1&m-1&2m-1&m^2-m\\
%0&m&m&0\\
%1&m&3m-1&0
%\end{array}\right).$$
%
%
\section{Symétrie et projection}
%%
\begin{Exercice}[Noyau et image]
On considère $\Fonction{s}{\R^2}{\R^2}{(x,y)}{(x + 2y, -y)}.$
\begin{enumerate}
\item Montrer que $s$ est une symétrie. Préciser ses éléments caractéristiques.
\item Démontrer $p=\frac{\mathrm{Id}+s}{2}$ est une projection.
\end{enumerate}
\end{Exercice}
\begin{Exercice}[Noyau et image]
On considère les espaces $F = \{(x, y, z) \in \R^3| x+2y+z = 0\text{ et }2x+y-z = 0\}$ et $G = \{(x, y, z) \in \R^3| x+y+2z = 0\}$.
\begin{enumerate}
\item Déterminer une base de $F,$ puis démontrer que $F$ et $G$ sont supplémentaires dans $\R^3$.
\item  Soit $p$ la projection sur $F$ parallèlement à $G$ et $(x, y, z) \in \R^3$. Déterminer les
coordonnées de $p(x,y,z)$. Déterminer la matrice de $p$ dans la base canonique.\\
Même question avec $q$ la projection sur $G$ parallèlement à $F$.
\end{enumerate}
\end{Exercice}
\begin{Exercice}
Soit $E$ un espace vectoriel et $p,q$ deux projecteurs de $E$ tels que $p\neq 0$, $q\neq 0$ et $p\neq q$. Démontrer que $(p,q)$ est une famille libre de $\mathcal{L}(E)$.
\end{Exercice}
\begin{Exercice}
Soient $E_1,\dots ,E_n$ des sous-espaces vectoriels de $E$. On suppose que $E_1\oplus \dots \oplus E_n=E$. On note $p_i$ le projecteur sur $E_i$ parallèlement à $\oplus_{j\neq i}E_j$.\\ 
Montrer que $p_i\circ p_j=0$ si $i\neq j$ et $p_1+\dots+p_n=\mathrm{Id}_E$.
\end{Exercice}
%\subsubsection{Forme linéaire}
%\Exercice[Une forme linéaire]
%
%Déterminer la forme linéaire $f$ définie sur $\mathbb R^3$ telle que
%$$f(1,1,1)=0,\ f(2,0,1)=1\textrm{ et }f(1,2,3)=4.$$
%Donner une base du noyau de $f$.
%
%
%\Exercice[Une base en dimension 2]
%Soient $f_1,f_2$ les deux éléments de $\mathcal L(\mathbb R^2,\mathbb R)$
%définis par
%$$f_1(x,y)=x+y\textrm{ et }f_2(x,y)=x-y.$$
%\begin{enumerate}
%\item Montrer que $(f_1,f_2)$ forme une base de $(\mathbb R^2)^*$.
%\item Exprimer les formes linéaires suivantes dans la base $(f_1,f_2)$ :
%$$g(x,y)=x,\ h(x,y)=2x-6y.$$
%\end{enumerate}


\end{document}
\endinput




