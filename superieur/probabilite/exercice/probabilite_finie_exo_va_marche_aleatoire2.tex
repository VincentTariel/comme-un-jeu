\exo{ 6001 , tariel, 01/09/2017, 1-2, {}}[Marche al�atoire]
Une particule se trouve � l'instant 0 au point d'abscisse $a$ ($a$ entier), sur un segment gradu� de $0$ � $N$ (on suppose donc $0\leq a\leq N$).
A chaque instant, elle fait un bond de $+1$ avec la probabilit� $p$ ($0<p<1/2$), ou un bond de $-1$ avec la probabilit� $q=1-p$.
Autrement dit, si $x_n$ est l'abscisse de la particule � l'instant $n$, on a :
$$x_{n+1}=\left\{\begin{array}{ll}
x_n+1&\textrm{avec probabilit� $p$}\\
x_n-1&\textrm{avec probabilit� $1-p$.}
\end{array}\right.$$
Le processus se termine lorsque la particule atteint une des extr�mit�s du segment (i.e. s'il existe $x_n$ avec $x_n=0$ ou $x_n=N$).
\begin{enumerate}
\item \'Ecrire un algorithme qui simule cette marche al�atoire. En particulier, cet algorithme prendra en entr�e l'abscisse $a$ de d�part,
la longueur $N$ du segment,
et produira en sortie un message indiquant si la marche s'arr�te en 0 ou en $N$, et le nombre de pas n�cessaires pour que le processus s'arr�te.
On supposera qu'on dispose d'une fonction alea() qui retourne un nombre al�atoire suivant une loi uniforme sur $[0,1]$.
\item On note $u_a$ la probabilit� pour que la particule partant de $a$, le processus s'arr�te en $0$. 
\begin{enumerate}
\item Que vaut $u_0$? $u_N$?
\item Montrer que si $0<a<N$, alors $u_a={pu_{a+1}}+qu_{a-1}$.
\item En d�duire l'expression exacte de $u_a$.
\end{enumerate}
\item On note $v_a$ la probabilit� pour que la particule partant de $a$, le processus s'arr�te en $N$. Reprendre les questions pr�c�dentes avec $v_a$ au lieu de $u_a$.
\item Calculer $u_a+v_a$. Qu'en d�duisez-vous?
\end{enumerate}
%\begin{indication}
%\includegraphics[width=15cm]{3_saut_puce.png}
%\end{indication}

% correction http://www.bibmath.net/ressources/index.php?action=affiche&quoi=mathsup/colles/espaceproba&type=fexo
\finexo