%Exercices probabilit�s finies
\exo{ 0402 , tariel, 01/11/2017, 1-2, {}}[Garagiste]
Un garagiste dispose de deux voitures de location. Chacune est utilisable en moyenne 4 jours sur 5. Il loue les voitures avec une marge brute de 300 euros par jour et par voiture.
On consid�re $X$ la variable al�atoire �gale au nombre de clients se pr�sentant chaque jour pour louer une voiture. On suppose que $X(\Omega)=\{0,1,2,3\}$ avec 
$$P(X=0)=0,1\ \ P(X=1)=0,3\ \ P(X=2)=0,4\ \ P(X=3)=0,2.$$
\begin{enumerate}
\item On note $Z$ le nombre de voitures disponibles par jour. D�terminer la loi de $Z$. On pourra consid�rer dans la suite que $X$ et $Z$ sont ind�pendantes.
\item On note $Y$ la variable al�atoire : " nombre de clients satisfaits par jour". D�terminer la loi de $Y$.
\item Calculer la marge brute moyenne par jour.
\end{enumerate}
\finexo