\exo{ 0402 , tariel, 01/11/2017, 1-2, {}}[***, Lemme de Borel-Cantelli]
Soit $(A_n)_{n\in \N}$ une suite d'�v�nements dans un m�me espace probabilis�.
On note 
$A =$ �~Il y a une infinit� d'�v�nements parmi les $A_n$ qui sont r�alis�s~�.\\
a) Montrer que $A$ est un �v�nement.\\
b) Si la s�rie $\sum_k P(A_k)$ est convergente, montrer que $P(A)=0$.\\
c) Si la s�rie $\sum_k P(A_k)$ est divergente et si les $A_k$ sont mutuellement
    ind�pendants, montrer que ${P(A)=1}$.\\
d) Donner un cas o� $P(A)=\frac12$.
   
\begin{indication}
$A = \cap_{n=0}^\infty(\cup_{k=n}^\infty A_k)$\\
b) $P(\cup_{k=n}^\infty A_k)\leq \sum_{k=n}^\infty P(A_k)\rightarrow_{n\rightarrow \infty}0$.\\c) $1-P(\cup_{k=n}^\infty A_k) = 1-P(A_n) + P(\overline{A_n}\cap A_{n+1})+ P(\overline{A_n}\cap\overline{A_{n+1}}\cap A_{n+2})+\ldots                   =(1-P(A_n))(1-P(A_{n+1}))\ldots $

    La s�rie de terme g�n�ral $-\ln(1-P(A_n))$ est divergente : grossi�rement   si $P(A_n)$ ne tend pas vers z�ro, sinon par �quivalence si  $P(A_n)\to_{n\to \infty}0$. Donc le produit infini pr�c�dent est nul, ce qui suffit � conclure.\\
    d) Tous les $A_k$ �gaux � un m�me �v�nement de probabilit�~$\frac12$.
\end{indication}
\finexo