\exo{ 0402 , tariel, 01/11/2017, 1-2, {}}[*, Poker]
Lors d'une partie de poker, un joueur re�oit 5 cartes d'un jeu de 52 cartes ; ce qui constitue une main.\\
\begin{enumerate}
\item Combien y a-t-il de mains possible ? 
\item Combien y a-t-il de mains possible avec un carr� ? ex. 1$\clubsuit$ 1$\blacklozenge$ 1$\heartsuit$ 1$\spadesuit$ V $\heartsuit$
\item Combien y a-t-il de mains avec un full ? ex. R$\clubsuit$ R$\blacklozenge$ 8$\heartsuit$ 8$\spadesuit$ 8$\clubsuit$
\item  Combien y a-t-il de mains avec une double paire ? ex. D$\clubsuit$ D$\blacklozenge$ 8$\heartsuit$ 8$\clubsuit$ R$\spadesuit$
\item  Combien y a-t-il de mains avec un brelan ? ex. V $\clubsuit$ V $\blacklozenge$ V $\heartsuit$ 1$\spadesuit$ 9$\clubsuit$
\item  Combien y a-t-il de mains avec une paire ? ex. D$\clubsuit$ D$\blacklozenge$ 7$\heartsuit$ 9$\spadesuit$ V $\clubsuit$
\item  Combien y a-t-il de mains avec une quinte flush ? ex. 7$\clubsuit$ 8$\clubsuit$ 9$\clubsuit$ 10$\clubsuit$ V $\clubsuit$
\item  Combien y a-t-il de mains avec une suite ? ex. 8$\clubsuit$ 9$\blacklozenge$ 10$\heartsuit$ V $\spadesuit$ D$\blacklozenge$
\item  Combien y a-t-il de mains avec une couleur ? ex. 1$\heartsuit$ 7$\heartsuit$ 10$\heartsuit$ D$\heartsuit$ R$\heartsuit$
\end{enumerate}
\begin{correction}
\begin{enumerate}
\item Une main correspond � un tirage sans remise et sans ordre. Donc le nombre de mains est $\begin{pmatrix}52\\5\end{pmatrix}$
\item \textit{nombre de mains avec un carr� :} d'apr�s le principe de d�composition,  on a  
$$\overbrace{\overbrace{\begin{pmatrix}13\\1\end{pmatrix}}^
{\begin{subarray}{l}\text{1 hauteur}\\
    \text{parmi 13}\end{subarray}}
\times    \overbrace{\begin{pmatrix}4\\4\end{pmatrix}}^{\begin{subarray}{l}\text{4 cartes parmi}\\
    \text{les 4 de la hauteur}\end{subarray}}}^{\text{carr�}}
\times     \overbrace{\begin{pmatrix}48\\1\end{pmatrix}}^{\text{carte restante}}=624$$
\item  \textit{nombre de mains avec un full :} d'apr�s le principe de d�composition,  on a  
$$
\overbrace{\overbrace{\begin{pmatrix}13\\1\end{pmatrix}}^
{\begin{subarray}{l}\text{1 hauteur}\\
    \text{parmi 13}\end{subarray}}
\times    \overbrace{\begin{pmatrix}4\\3\end{pmatrix}}^{\begin{subarray}{l}\text{3 cartes parmi}\\
    \text{les 4 de la hauteur}\end{subarray}}}^{\text{brelan}}
\times\overbrace{\overbrace{\begin{pmatrix}12\\1\end{pmatrix}}^
{\begin{subarray}{l}\text{1 hauteur}\\
    \text{parmi 12}\end{subarray}}
\times    \overbrace{\begin{pmatrix}4\\2\end{pmatrix}}^{\begin{subarray}{l}\text{2 cartes parmi}\\
    \text{les 4 de la hauteur}\end{subarray}}}^{\text{pair}}=3744
    $$
\item \textit{nombre de mains avec une double paire :} d'apr�s le principe de d�composition,  on a  
$$
\overbrace{\overbrace{\begin{pmatrix}13\\2\end{pmatrix}}^
{\begin{subarray}{l}\text{2 hauteurs}\\
    \text{parmi 13}\end{subarray}}
\times    \overbrace{\begin{pmatrix}4\\2\end{pmatrix}}^{\begin{subarray}{l}\text{2 cartes parmi}\\
    \text{les 4 d'une pair}\end{subarray}}
\times    \overbrace{\begin{pmatrix}4\\2\end{pmatrix}}^{\begin{subarray}{l}\text{2 cartes parmi}\\
    \text{les 4 de l'autre pair}\end{subarray}}}^{\text{double pairs}}
    \times    \overbrace{\begin{pmatrix}24\\1\end{pmatrix}}^{\text{carte restante}}=3744
$$
Attention, il faut choisir deux paires diff�rentes sinon
c'est un carr�. De plus, le nombre de mains n'est pas 
$$
\overbrace{\overbrace{\begin{pmatrix}13\\1\end{pmatrix}}^
{\begin{subarray}{l}\text{1 hauteur}\\
    \text{parmi 13}\end{subarray}}
\times    \overbrace{\begin{pmatrix}4\\2\end{pmatrix}}^{\begin{subarray}{l}\text{2 cartes parmi}\\
    \text{les 4 de la hauteur}\end{subarray}}}^{\text{pair}}
\times\overbrace{\overbrace{\begin{pmatrix}12\\1\end{pmatrix}}^
{\begin{subarray}{l}\text{1 hauteur}\\
    \text{parmi 12}\end{subarray}}
\times    \overbrace{\begin{pmatrix}4\\2\end{pmatrix}}^{\begin{subarray}{l}\text{2 cartes parmi}\\
    \text{les 4 de la hauteur}\end{subarray}}}^{\text{pair}}
    \times     \overbrace{\begin{pmatrix}44\\1\end{pmatrix}}^{\text{carte restante}}$$
car on met de l'ordre  dans les paires. Par exemple, on compte deux fois la main:\\
 As Tr�fle, As Carreau, Roi Tr�fle, roi Carreau, 7 C\oe ur\\ et\\
  Roi Tr�fle, Roi Carreau, As Tr�fle, As Carreau 7 C\oe ur.  
\item  \textit{nombre de mains avec un brelan :}
 d'apr�s le principe de d�composition,  on a  
$$\overbrace{
\overbrace{\begin{pmatrix}13\\1\end{pmatrix}}^
{\begin{subarray}{l}\text{1 hauteur}\\
    \text{parmi 13}\end{subarray}}
\times    \overbrace{\begin{pmatrix}4\\3\end{pmatrix}}^{\begin{subarray}{l}\text{3 cartes parmi}\\
    \text{les 4}\end{subarray}}}^{\text{brelan}}
\times \overbrace{   \overbrace{\begin{pmatrix}12\\2\end{pmatrix}}^{\begin{subarray}{l}\text{2 hauteurs parmi}\\
    \text{les 12}\end{subarray}}
\times    \overbrace{\begin{pmatrix}4\\1\end{pmatrix}}^{\begin{subarray}{l}\text{1 carte parmi}\\
    \text{les 4}\end{subarray}}
\times    \overbrace{\begin{pmatrix}4\\1\end{pmatrix}}^{\begin{subarray}{l}\text{1 carte parmi}\\
    \text{les 4}\end{subarray}}}^{\text{2 cartes restantes}}
$$
Attention, il faut choisir deux hauteurs diff�rentes sans ordre pour les 2 cartes restantes pour ne pas avoir une pair formant alors un full.
\item \textit{nombre de mains avec une paire :} d'apr�s le principe de d�composition,  on a  
$$\overbrace{
\overbrace{\begin{pmatrix}13\\1\end{pmatrix}}^
{\begin{subarray}{l}\text{1 hauteur}\\
    \text{parmi 13}\end{subarray}}
\times    \overbrace{\begin{pmatrix}4\\2\end{pmatrix}}^{\begin{subarray}{l}\text{2 cartes parmi}\\
    \text{les 4}\end{subarray}}}^{\text{pair}}
\times \overbrace{   \overbrace{\begin{pmatrix}12\\3\end{pmatrix}}^{\begin{subarray}{l}\text{3 hauteurs parmi}\\
    \text{les 12}\end{subarray}}
\times    \overbrace{\begin{pmatrix}4\\1\end{pmatrix}}^{\begin{subarray}{l}\text{1 carte parmi}\\
    \text{les 4}\end{subarray}}
\times    \overbrace{\begin{pmatrix}4\\1\end{pmatrix}}^{\begin{subarray}{l}\text{1 carte parmi}\\
    \text{les 4}\end{subarray}}
 \times    \overbrace{\begin{pmatrix}4\\1\end{pmatrix}}^{\begin{subarray}{l}\text{1 carte parmi}\\
    \text{les 4}\end{subarray}}   
    }^{\text{3 cartes restantes}}
$$
\item \textit{nombre de mains avec une quinte flush :} d'apr�s le principe de d�composition,  on a  
$$
\overbrace{\begin{pmatrix}13\\1\end{pmatrix}}^
{\text{hauteur du d�but de la suite}}
\times
\overbrace{\begin{pmatrix}4\\1\end{pmatrix}}^
{\text{1 couleur}}
=36$$
\item  \textit{nombre de mains avec une suite :} choisir 5 cartes qui se suivent peu importe
leur couleur et retrancher les quintes flush. D'apr�s le principe de d�composition,  on a  
$$\overbrace{\begin{pmatrix}13\\1\end{pmatrix}}^
{\text{hauteur du d�but de la suite}}
\times
\overbrace{\begin{pmatrix}4\\1\end{pmatrix}^5}^
{\text{les couleur}}-36.$$
\item   \textit{nombre de mains avec une couleur :} choisir une couleur puis 5 cartes dans
la couleur et retrancher les quintes flush. D'apr�s le principe de d�composition,  on a  :
$$\overbrace{\begin{pmatrix}4\\1\end{pmatrix}}^
{\text{1 couleur}}
\times
\overbrace{\begin{pmatrix}13\\5\end{pmatrix}}^
{\text{5 cartes dans une couleur}}-36.$$
\end{enumerate}

\end{correction}

\finexo