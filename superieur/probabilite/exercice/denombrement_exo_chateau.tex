\exo{ 0402 , tariel, 01/11/2017, 1-2, {}}[*, Poker]
Lors d'une partie de poker, un joueur re�oit 5 cartes d'un jeu de 32 cartes ; ce qui constitue une main.\\
\begin{enumerate}
\item Combien y a-t-il de mains possible ? 
\item Combien y a-t-il de mains possible avec un carr� ? . 1$\clubsuit$ 1$\blacklozenge$ 1$\heartsuit$ 1$\spadesuit$ V $\heartsuit$
\item Combien y a-t-il de mains avec un full ?ex. R$\clubsuit$ R$\blacklozenge$ 8$\heartsuit$ 8$\spadesuit$ 8$\clubsuit$
\item  Combien y a-t-il de mains avec une double paire ?ex. D$\clubsuit$ D$\blacklozenge$ 8$\heartsuit$ 8$\clubsuit$ R$\spadesuit$
\item  Combien y a-t-il de mains avec un brelan ?ex. V $\clubsuit$ V $\blacklozenge$ V $\heartsuit$ 1$\spadesuit$ 9$\clubsuit$
\item  Combien y a-t-il de mains avec une paire ?ex. D$\clubsuit$ D$\blacklozenge$ 7$\heartsuit$ 9$\spadesuit$ V $\clubsuit$
\item  Combien y a-t-il de mains avec une quinte flush ?ex. 7$\clubsuit$ 8$\clubsuit$ 9$\clubsuit$ 10$\clubsuit$ V $\clubsuit$
\item  Combien y a-t-il de mains avec une suite ?ex. 8$\clubsuit$ 9$\blacklozenge$ 10$\heartsuit$ V $\spadesuit$ D$\blacklozenge$
\item  Combien y a-t-il de mains avec une couleur ?ex. 1$\heartsuit$ 7$\heartsuit$ 10$\heartsuit$ D$\heartsuit$ R$\heartsuit$
\item  Combien y a-t-il de mains avec au moins un as ?
\end{enumerate}
\begin{correction}
\begin{enumerate}
\item Une main correspond � un tirage sans remise et sans ordre. Donc le nombre de mains est $\begin{pmatrix}52\\5\end{pmatrix}$
\item nombre de mains avec un carr� = " choisir 1 carr� (parmi 13) et une autre
carte (parmi 48) ". D'apr�s le principe de d�composition,  on a  $\begin{pmatrix}13\\1\end{pmatrix}\begin{pmatrix}48\\1\end{pmatrix}=624$
\item  nombre de mains avec un full= " choisir 1 brelan et choisir une paire "  mais
attention : choisir 1 brelan=" choisir une figure (parmi 13) et choisir 3 cartes (parmi les
4 de la figure) " et choisir 1 paire=" choisir une figure (parmi les 12 restantes) et choisir 2 cartes
(parmi les 4 de la figure)". D'apr�s le principe de d�composition,  on a  
$$\overbrace{\begin{pmatrix}13\\1\end{pmatrix}\begin{pmatrix}4\\3\end{pmatrix}}^{brelan}\overbrace{\begin{pmatrix}12\\1\end{pmatrix}\begin{pmatrix}4\\2\end{pmatrix}}^{paire}=3744$$
\item nombre de mains avec une double paire= " choisir deux paires (diff�rente sinon
c'est un carr� !) et choisir une cinqui�me carte ". D'apr�s le principe de d�composition,  on a  
$$\overbrace{\begin{pmatrix}13\\2\end{pmatrix}\begin{pmatrix}4\\2\end{pmatrix}\begin{pmatrix}4\\2\end{pmatrix}}^{2 paires \neq}\begin{pmatrix}48\\1\end{pmatrix}=3744$$
Attention, le nombre de mains n'est pas = " choisir 1 paire, puis choisir une paire parmi 12 et une autre
carte (parmi 44) " car on met de l'ordre  dans les paires. 
\item  nombre de mains avec un brelan=" Choisir un brelan et deux autres cartes
(mais sans paire) ". D'apr�s le principe de d�composition,  on a  
$$\overbrace{\begin{pmatrix}13\\1\end{pmatrix}\begin{pmatrix}4\\3\end{pmatrix}}^{brelan}\overbrace{\begin{pmatrix}12\\2\end{pmatrix}\begin{pmatrix}4\\1\end{pmatrix}\begin{pmatrix}4\\1\end{pmatrix}}^{2 cartes}=54912$$
\item nombre de mains avec une paire= " choisir une paire et choisir 3 autres cartes
qui ne forment ni un brelan ni une autre paire ". D'apr�s le principe de d�composition,  on a  
$$\overbrace{\begin{pmatrix}13\\1\end{pmatrix}\begin{pmatrix}4\\2\end{pmatrix}}^{brelan}\overbrace{\begin{pmatrix}12\\3\end{pmatrix}\begin{pmatrix}4\\1\end{pmatrix}\begin{pmatrix}4\\1\end{pmatrix}\begin{pmatrix}4\\1\end{pmatrix}}^{3 cartes}=1098240$$
\item nombre de mains avec une quinte flush=" choisir une couleur puis 5
cartes qui se suivent ". Pour choisir les 5 cartes � suivre il suffit de choisir la
premi�re. On peut choisir l?as, le roi ,. . ., le 5 mais pas l'une des 4 derni�res
cartes. D'apr�s le principe de d�composition,  on a  
$$\begin{pmatrix}4\\1\end{pmatrix}*9=36$$
\item  nombre de mains avec une suite=" choisir 5 cartes qui se suivent peu importe
leur couleur et retrancher les quintes flush". On 9 choix de la premi�re l?as, le roi ,. . ., le 5 avec 4 choix de couleurs, puis pour la seconde 4 choix de couleurs  et ainsi de suite  ce qui donne : $9\time 4^5 � 9 ? 36 = 9180$
\item   nombre de mains avec une couleur=" choisir une couleur puis 5 cartes dans
la couleur et retrancher les quintes flush" ce qui donne : $4\begin{pmatrix}13\\4\end{pmatrix}- 36 = 5112$
\item nombre de mains avec au moins un as = nombre de mains possibles - nombre de mains sans as donc on a 
$\begin{pmatrix}52\\4\end{pmatrix}- \begin{pmatrix}48\\4\end{pmatrix} =886656$
\end{enumerate}

\end{correction}

\finexo