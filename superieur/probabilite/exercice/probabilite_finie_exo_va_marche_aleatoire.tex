\exo{ 0402 , tariel, 01/11/2017, 1-2, {}}[Marche al�atoire]
Un mobile se d�place de fa�on al�atoire sur un axe gradu�.
� l'instant $0$, il est � l'origine.
� chaque instant entier, il se d�place d'une unit� vers la droite avec la probabilit� $p\in ]0,1[$
ou d'un pas vers la gauche avec la probabilit� $q=1-p$,
et de ce fa�on ind�pendante.
On note $X_n$ son abscisse apr�s $n$ pas.
\begin{enumerate}
\item
  Soit $D_n$ la variable al�atoire �gale au nombre de pas vers la droite.
  Quelle est la loi de $D_n$? Exprimer $X_n$ en fonction de $D_n$.
\item
  En d�duire l'esp�rance et la variance de $X_n$.
  Pour quelle valeur de $p$ la variable $X_n$ est-elle centr�e?
  Interpr�ter.
\item
  Reprendre l'exercice avec une autre m�thode:
  on note, pour $n?1$, $Y_n = X_n - X_{n-1}$.

  \begin{enumerate}
  \item
    D�terminer la loi de $Y_n$.
  \item
    Justifier l'ind�pendance de $Y_1, \dots, Y_n$.
  \item
    En d�duire l'esp�rance et la variance de $X_n$.
  \end{enumerate}
\end{enumerate}
\finexo