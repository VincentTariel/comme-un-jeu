\exo{ 0402 , tariel, 01/11/2017, 1-2, {}}[loi hyperg�om�trique]

Une urne contient $a$ boules blanches et $b$ boules noires.
On tire une poign�e de $n$ boules dans l'urne, avec $(a,b) \in(\N^*)^2$ et $n \in\{1,\dots,a+b\}$.
On appelle $X$ le nombre de boules blanches dans la poign�e.
\begin{enumerate}
\item
  D�terminer le support de $X$.
\item
  D�terminer la loi de $X$.
\item
  Calculer l'esp�rance de $X$.
\item
  Calculer l'esp�rance de $X(X-1)$ puis la variance de $X$.
\item
  Comparer l'esp�rance et la variance de $X$ � celle d'une loi binomiale de param�tres
  $n$ et $a/(a+b)$. Commentaire?
\end{enumerate}
\finexo