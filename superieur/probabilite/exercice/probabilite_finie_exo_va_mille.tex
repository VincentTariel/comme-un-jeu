%Exercices probabilit�s finies
\exo{ 0402 , tariel, 01/11/2017, 1-2, {}}[En plein dans le mille!]
Un joueur tire sur une cible de 10 cm de rayon, constitu�e de couronnes concentriques,
d�limit�es par des cercles de rayons 1,2, ..., 10 cm, et num�rot�es respectivement de 10 � 1. La
probabilit� d'atteindre la couronne $k$ est proportionnelle � l'aire de cette couronne, et on suppose que
le joueur atteint sa cible � chaque lancer. Soit $X$ la variable al�atoire qui � chaque lancer associe le num�ro de la cible.
\begin{enumerate}
\item Quelle est la loi de probabilit� de X ?
\item Le joueur gagne $k$ euros s'il atteint la couronne num�rot�e $k$ pour $k$ compris entre 6 et 10, tandis qu'il perd 2 euros s'il atteint l'une des couronnes p�riph�riques num�rot�es de 1 � 5. Le jeu est-il
favorable au joueur ?
\end{enumerate}
\finexo