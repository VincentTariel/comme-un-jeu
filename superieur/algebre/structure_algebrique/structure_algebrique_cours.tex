\documentclass{book}
\usepackage{commeunjeustyle}

\begin{document}

\chapter*{Structures algébriques}
\begin{Texte}
L'enfant, après avoir appris 
\begin{itemize}
\item à compter le nombre d'objets en ajoutant des unités à partir de 0 c'est à dire en construisant l'ensemble $\N$  avec l'opération "suivant"
\begin{center}
\begin{tikzpicture}
\draw[color=colorprop] (0,0) rectangle (1,1);
\node at (0.5,-0.3){$0$};
\draw[color=colorprop] (1.5,0) rectangle (2.5,1);
\draw[color=colordef] (2,0.5) circle (0.15);
\node at (2,-0.3){$1$};
\draw[color=colorprop] (3,0) rectangle (4,1);
\draw[color=colordef] (3.3,0.2) circle (0.15);
\draw[color=colordef] (3.7,0.7) circle (0.15);
\node at (3.5,-0.3){$2$};
\draw[color=colorprop] (4.5,0) rectangle (5.5,1);
\draw[color=colordef] (4.8,0.75) circle (0.15);
\draw[color=colordef] (4.7,0.2) circle (0.15);
\draw[color=colordef] (5.3,0.5) circle (0.15);
\node at (5,-0.3){$3$};
\end{tikzpicture}\\
Désignation à l'aide d'un nombre de la quantité de cercles 
\end{center}
\item à ordonner en comparant des quantités c'est à munir d'une structure d'ordre à l'ensemble $\N$ 
\begin{center}
\begin{tikzpicture}
\draw[color=colorprop] (1.5,0) rectangle (2.5,1);
\draw[color=colordef] (2,0.5) circle (0.15);
\node at (3,0.5){$\leq$};
\draw[color=colorprop] (3.5,0) rectangle (4.5,1);
\draw[color=colordef] (3.8,0.75) circle (0.15);
\draw[color=colordef] (3.7,0.2) circle (0.15);
\draw[color=colordef] (4.3,0.5) circle (0.15);
\end{tikzpicture}\\
Comparaison de deux quantités de cercles 
\end{center}
commence à 

\end{itemize}



\end{Texte}


%%%%%%%%%%%%%%%%%%%%%%%%%%%%%%%%%%%%%%%%%%%%%%%%%%%%%%%%%%%%%%%%%%%%%%
\section{Loi de composition interne}
\subsection{Définition}
\begin{Definition}[Loi de composition interne]
Soit $A$ un ensemble.\\
Une \defi{loi de composition interne}, $\bigtriangleup$, est une application qui, à deux éléments de $A$, associe un élément de $A$ :
$$ \Fonction{\bigtriangleup}{A\times A}{A}{(x,y)}{x \bigtriangleup y}.$$
\end{Definition}
\begin{Exemple}
\begin{itemize}
\item Sur $\R$, l'addition définie par $\Fonction{+}{\R\times \R}{\R}{(x,y)}{x + y}$, la soustraction $\Fonction{-}{\R\times \R}{\R}{(x,y)}{x - y}$ et la multiplication $\Fonction{-}{\R\times \R}{\R}{(x,y)}{x \times y}$ sont des lois de composition internes.
\item Sur $\N$, la soustraction n'est pas une loi interne, mais elle l'est dans $\Z$.
\item Sur $\N^*$,  l'exponentiation définie par $\Fonction{}{\N^*\times\N^*}{\N^*}{(a,b)}{a^b}$, le PGCD ou le PPCM sont des lois internes.
\item Soit $X$ un ensemble. Sur l'ensemble des parties de $X$, $\mathcal{P}(X)$, l'union définie par $\Fonction{\cup}{\mathcal{P}(X)\times \mathcal{P}(X)}{\mathcal{P}(X)}{(A,B)}{A \cup B}$ et l'intersection $\Fonction{\cap}{\mathcal{P}(X)\times \mathcal{P}(X)}{\mathcal{P}(X)}{(A,B)}{A \cap B}$ sont des lois de composition internes.
\end{itemize}
\end{Exemple}

\begin{Definition}[Loi de composition externe] 
Soit $X$ et $A$ deux ensembles.\\
Une \defi{loi de composition externe}, $.$, est une application qui, à un élément de $X$ et un élément de $A$, associe un élément de $A$ :
$$ \Fonction{.}{X\times A}{A}{(\lambda,x)}{\lambda. x} $$
\end{Definition}
\begin{Exemple}[\((\R^2,.)\)]
La multiplication par un scalaire sur l'ensemble des vecteurs du plan  $\R^2$ définie par 
$$ \Fonction{.}{\R\times\R^2}{\R^2}{(\lambda,(x_1,x_2))}{(\lambda.x_1, \lambda.x_2)}.$$ est une loi de composition externe.
\end{Exemple}


\subsection{Propriétés éventuelles des lois de composition interne}
\begin{Definition}[Commutativité et associativité]
$\bigtriangleup$ est
\begin{enumerate}
\item  \defi{associative} : si $\in(x,y,z)\in A^3$, $x \bigtriangleup (y \bigtriangleup z) = (x \bigtriangleup y) \bigtriangleup z$.
  On ne considèrera que des loi associatives.
\item
 \defi{commutative} : si $\forall(x,y)\in A^2$, $x \bigtriangleup y = y \bigtriangleup x$.
 \end{enumerate}
\end{Definition} 
 \begin{Exemple}
\begin{itemize}
\item Sur $\R$, l'addition et la multiplication sont commutatives et associatives. Ce n'est pas le cas de la soustraction car $1-0\neq 0-1$ et $1-(2-3)=2\neq -4= (1-2)-3.$
\begin{center}
\begin{tikzpicture}
\draw[color=colorprop,fill=colorprop!20] (0,0) rectangle (1,1);
\draw[color=colorprop,fill=colorprop!20] (1,0) rectangle (2,1);
\draw[color=colorprop,fill=colorprop!20] (2,0) rectangle (3,1);
\draw[color=colordef,fill=colordef!20] (0,1) rectangle (1,2);
\draw[color=colordef,fill=colordef!20] (1,1) rectangle (2,2);
\draw[color=colordef,fill=colordef!20] (2,1) rectangle (3,2);
\draw[color=colorprop,fill=colorprop!20] (-4,0) rectangle (-3,1);
\draw[color=colordef,fill=colordef!20] (-3,0) rectangle (-2,1);
\draw[color=green,fill=green!20] (-2,0) rectangle (-1,1);
\draw[color=colorprop,fill=colorprop!20] (-4,1) rectangle (-3,2);
\draw[color=colordef,fill=colordef!20] (-3,1) rectangle (-2,2);
\draw[color=green,fill=green!20] (-2,1) rectangle (-1,2);
\node at (-1,-0.5) {$\overbrace{3}^{\text{Nombre de colonnes}}\times \overbrace{2}^{\text{Nombre de cubes par colonne}}\quad=\quad \overbrace{2}^{\text{Nombre de lignes}}\times \overbrace{3}^{\text{Nombre de cubes par ligne}}$};
\end{tikzpicture}\\
Démonstration géométrique élémentaire de la commutativité  de la multiplication dans $\N$ comme addition itérée.
\end{center}

\item Sur $\N^*$,  l'exponentiation n'est pas commutative $1^2=1\neq 2=2^1$ et non plus associative $\left(2^2\right)^3 = 64\neq 256 =2^{(2^3)}$.
\item Sur  $\mathcal{P}(X)$, l'union et l'intersection sont commutatives et associatives.
\end{itemize}
\end{Exemple}
\begin{Remarque}
Quand la loi est associative, la \defi{notation itérée} est
\begin{itemize}
\item en cas d'une loi de multiplication : $\forall a\in A,\forall n\in\N^*: x^n=\overbrace{x\times \dots \times x }^{\text{n fois}} $
\item en cas d'une loi d'addition : $\forall a\in A,\forall n\in\N^*: nx=\overbrace{x+ \dots + x }^{\text{n fois}} $.
\end{itemize}  
\end{Remarque}

\begin{Definition}[Distributivité d'une loi sur une autre]
Soit $A$ un ensemble et $\bigtriangleup$ et $\square$ deux lois de composition internes sur $A$.\\
On dit que $\bigtriangleup$ est \defi{distributive} sur $\square$ si:
$$ \forall x, y, z  \in A :\quad   x \bigtriangleup (y \square z) = (x \bigtriangleup y) \square (x \bigtriangleup z)\text{ et }(y \square z) \bigtriangleup x = (y \bigtriangleup x) \square (z \bigtriangleup x).$$
\end{Definition} 
\begin{Exemple}
\begin{itemize}
\item Sur $\R$, la multiplication est distributive sur l'addition mais l'addition n'est pas distributive sur la multiplication car $1+(2\times3)=7\neq 5=1\times 2+ 1\times 3$.
\item Sur  $\mathcal{P}(X)$, l'union et l'intersection sont distributives l'une par rapport à l'autre.
\end{itemize}
\end{Exemple}
\subsection{Symétrique et élément neutre}
\begin{Definition}[Elément neutre]
$\bigtriangleup$ admet \defi{un élément neutre} si il existe $ e\in A$ tel que $\forall x\in A$, $x \bigtriangleup e = e \bigtriangleup x = x$.
\end{Definition}
\begin{Proposition}[Unicité de l'élément neutre]
Si $\bigtriangleup$ admet un élément neutre, alors celui-ci est unique.
\end{Proposition}
\begin{Demonstration}
Supposons qu'il existe deux éléments neutres $ e$ et $e'$.\\
On a $e \bigtriangleup e'\overbrace{=}^{e \text{ elt neutre}}e'$ et $e \bigtriangleup e'\overbrace{=}^{e' \text{ elt neutre}}e$. Ainsi $ e=e'$.
\end{Demonstration}
\begin{Exemple}
\begin{itemize}
\item Sur $\R$, $1$ est l'élément neutre de la multiplication et $0$ de l'addition.
\item Sur  $\mathcal{P}(X)$, l'ensemble vide $\emptyset$ est l'élément neutre de l'union et  $X$ de  l'intersection.
\end{itemize}
\end{Exemple}

\begin{Definition}[Élément symétrique et loi symétrique]
Soit $ x\in A$ et la loi $\bigtriangleup$  admettant  un élément neutre $e$ .\\
$x$ admet un \defi{symétrique} pour $\bigtriangleup$ si il existe $x'\in A$ tel que $x \bigtriangleup x' = x' \bigtriangleup x = e$. Dans ce cas, $x'$ est appelé le \defi{symétrique} de $x$.
\end{Definition}
\begin{Proposition}[Unicité de l'élément symétrique]
Soit $\bigtriangleup$ une loi associative et admettant  un élément neutre $e$.\\
Si $x$ admet un symétrique $x'$, alors celui-ci est unique.
\end{Proposition}
\begin{Demonstration}
Supposons qu'il existe deux éléments symétriques $x'$ et $x''$.\\
On a $(x' \bigtriangleup x)\bigtriangleup x''=e\bigtriangleup x''=x''$ et $(x' \bigtriangleup x)\bigtriangleup x''\overbrace{=}^{\bigtriangleup \text{ associative}} x'\bigtriangleup(x\bigtriangleup x'')=x'\bigtriangleup e=x'$.  Ainsi $x'=x''$.
\end{Demonstration}
\begin{Vocabulaire}
Le symétrique est appelé :
\begin{itemize}
\item \defi{opposé} en cas d'une loi additive $+$
\item \defi{inverse} en cas d'une loi multiplicative $\times$
\end{itemize}
\end{Vocabulaire}

\begin{Exemple}
Sur $\R$,  l'inverse de la multiplication d'un réel non nul $x$ est $\frac{1}{x}$ et l'opposé de l'addition d'un réel $x$ est $-x$.
\end{Exemple}

\begin{Definition}[Loi symétrique]
La loi $\bigtriangleup$ est  \defi{symétrique} si la loi est associative et si tout élément de $A$ admet un symétrique.
\end{Definition}

\begin{Exemple}
Sur $\R$, la multiplication n'est pas inversible car $0$ n'a pas d'inverse. En revanche sur $\R^*$, la multiplication est inversible.
\end{Exemple}
\subsection{Parties stables}
\begin{Definition}[Partie stable]
Soit $B$ une partie non vide de $A$.\\
$B$ est \defi{stable} pour $\bigtriangleup$ si  
$$\forall x, y \in B :\quad   x \bigtriangleup y \in B.$$
\end{Definition}
\begin{Exemple}
\begin{itemize}
\item Sur $\R$, les ensembles $\Q$, $\Z$, $\N$ et les nombres pairs sont stables pour l'addition. 
\item Sur $\C$, l'ensemble $ \mathcal{U}$ des nombres complexes de module 1 est stable pour la multiplication car le produit de deux nombres
complexes de module 1 est un nombre complexe de module 1.
\end{itemize}
\end{Exemple}
\begin{Definition}[Loi induite]
Soit $\bigtriangleup$ une loi sur $A$ et $B$ une partie de $A$ stable pour $\bigtriangleup$.\\
La \defi{loi induite $\tilde{\bigtriangleup}$} est définie par  :
$$ \Fonction{\tilde{\bigtriangleup}}{B\times B}{B}{(x,y)}{x \bigtriangleup y}.$$
Pour alléger les notations, on identifie $\tilde{\bigtriangleup}$ à $\bigtriangleup$. 
\end{Definition}
\begin{Exemple}On munit  l'ensemble $ \mathcal{U}$ des nombres complexes de module 1 avec la loi induite $*$ sur $\C$.
\end{Exemple}
\section{Groupes}
\subsection{Définition}
\begin{Definition}[Groupe]
Un \defi{groupe} est un couple $(G,\ast)$ où $G$ est un ensemble et $\ast$ une loi de composition interne sur $G$ associative, admettant un neutre et pour laquelle tout élément de $G$ admet un symétrique pour la loi $\ast$.
Un groupe est dit \defi{abélien} ou \defi{commutatif} si la loi $\ast$ est de plus commutative.
\end{Definition}
\begin{Proposition}[Groupes de référence]
$(\Z, +), (\Q, +), (\R, +)$ et  $(\C, +)$  sont des groupes commutatifs.\\
$(\Q^*, \times), (\R^*, \times)$ et  $(\C^*,\times)$  sont des groupes commutatifs.\\ 
\end{Proposition}
\begin{Demonstration}
Les hypothèses à vérifier ont été énoncées dans la section précédente.
\end{Demonstration}
\begin{Remarque}
Lors de l'introduction d'un groupe $(G,*)$, on omet de mentionner la loi $*$ afin d'alléger les notations.\\
L'ensemble $G$ ne peut pas être vide car il contient au moins l'élément neutre.
\end{Remarque}
\subsection{Sous-groupes}
\begin{Definition}[Sous-Groupe]
Soit $(G, *)$ un groupe et $H$ une partie de $G$.\\
$(H, *)$ est un \defi{sous-groupe} de $(G, *)$  si $H$ est stable pour $*$ et, muni de la loi induite, est un groupe.\\
\end{Definition}





%\begin{Exemple}[Le Groupe \((\R^n,+)\) ]
%La loi d'addition sur $\R^n$ est définie par 
%$$ \Fonction{+}{\R^n\times\R^n}{\R^n}{(\vec{x}=(x_1,x_2,\dots,x_n),\Vect{y}=(y_1,y_2,\dots,y_n))}{\Vect{x}+\Vect{y}=(x_1+y_1, x_2+y_2, \dots, x_n+y_n)}.$$
%
%\begin{itemize}
%\item  \defi{associative} : Soit $\Vect{x}=(x_1,x_2,\dots,x_n),\Vect{y}=(y_1,y_2,\dots,y_n),\Vect{z}=(z_1,z_2,\dots,z_n)\in\R^n .$\\
% On a :
% $$\begin{aligned}
% \Vect{x}+(\Vect{y}+\Vect{z})&=(x_1,x_2,\dots,x_n)+\left((y_1,y_2,\dots,y_n)+(z_1,z_2,\dots,z_n)\right)\\
% &=(x_1,x_2,\dots,x_n)+(y_1+z_1, y_2+z_2, \dots, y_n+z_n)\\
% &=(x_1+y_1+z_1, x_2+y_2+z_2, \dots, x_n+y_n+z_n)\\
%  &=(x_1+y_1, x_2+y_2, \dots, x_n+y_n)+(z_1,z_2,\dots,z_n)\\
%  &=\left((x_1,x_2,\dots,x_n)+(y_1,y_2,\dots,y_n)\right)+(z_1,z_2,\dots,z_n)\\
%  &=(\Vect{x}+\Vect{y})+\Vect{z}
% \end{aligned}$$
% \item  \defi{commutative} : Soit $\Vect{x}=(x_1,x_2,\dots,x_n),\Vect{y}=(y_1,y_2,\dots,y_n)\in\R^n .$\\
% On a :
% $$\begin{aligned}
% \Vect{x}+\Vect{y}&=(x_1,x_2,\dots,x_n)+(y_1,y_2,\dots,y_n)\\
%  &=(x_1+y_1, x_2+y_2, \dots, x_n+y_n)\\
%  &=(y_1+x_1, y_2+x_2, \dots, y_n+x_n)\\
%  &=\Vect{y}+\Vect{x}
% \end{aligned}$$
% \item  \defi{élément neutre} : Soit $\Vect{x}=(x_1,x_2,\dots,x_n)\in\R^n$\\
% Montrons que $\Vect{0}=(0,0,\dots,0)$ est l'élément neutre\\
% On a :
% $$\begin{aligned}
% \Vect{x}+\Vect{0}&=(x_1+0,x_2+0,\dots,x_n+0)\\
%  &=(x_1, x_2, \dots, x_n)\\
%  &=\Vect{x}
% \end{aligned}$$
%  \item  \defi{symétrique} : Soit $\Vect{x}=(x_1,x_2,\dots,x_n)\in\R^n$\\
% Montrons que $-\Vect{x}=(-x_1,-x_2,\dots,-x_n)$ est l'opposé de $\Vect{x}$.\\
% On a :
% $$\begin{aligned}
% \Vect{x}+(-\Vect{x})&=(x_1,x_2,\dots,x_n)+(-x_1,-x_2,\dots,-x_n)\\
%  &=(x_1-x_1, x_2-x_2, \dots, x_n-x_n)\\
%  &=(0, 0, \dots, 0)\\
%  &=\Vect{0}
% \end{aligned}$$
%\end{itemize}
%Donc $(\R^n,+)$ est un groupe commutatif.
%\end{Exemple}




\begin{Definition}[Corps \(\K\) : \(\R\) ou \(C\)] Dans ce cours%\samepage\footnote{%\begin{tiny}
%Un \defi{corps} (commutatif) $(K,+,\times )$ est un ensemble muni de deux lois internes possédant les propriétés suivantes :
%\begin{itemize}
%\item $(K,+)$ est un groupe abélien, dont l'élément neutre est noté $0$ ;
%\item $(K\setminus \{0\},\times )$ est également un groupe abélien, dont l'élément neutre est noté $1$ ;
%\item $\times$ est distributive par rapport à $+$.
%\end{itemize}
%\impo{Remarque} Lorsque le contexte est clair, on écrit souvent $\K$ au lieu de $(\K,+,×)$.\\
%\impo{Exemples}
%\begin{itemize}
%\item le corps des réels $\R$, des complexes $\C$, des rationnels $\Q$;
%\item si $\K$ est un corps, le corps des fractions rationnelles à coefficients dans $\K$, noté $\K(X)$;
%\item $\Q [i] = \{a+ib:(a,b)\in \Q ^2\}$;
%\item le corps des entiers modulo un nombre premier $p$, noté $\Z/p\Z$.
%\end{itemize}
%}
, un corps $\K$ désigne soit l'ensemble des nombres réels $\R$ ou soit l'ensemble des nombres complexes $\C$. 
\end{Definition}
\begin{Definition}[Espace vectoriel : \(\lambda\Vect{x}+\mu\Vect{y}\)]
Soit $\K$ un corps.\\
Un \defi{$\K$-espace vectoriel} est un triplet $(E,+,.)$ où
$+$ est une loi de composition interne sur $E$ et
$.$ est une loi de composition externe sur $E$,
vérifiant les propriétés suivantes:
\begin{enumerate}
\item $(E, +)$ est un groupe commutatif;
\item la loi $.$ est compatible avec la structure de groupe $(E, +)$, i.e.  
 \begin{enumerate}
  \item $\forall(\lambda ,\mu )\in\K^2$, $\forall \Vect{x}\in E$, $(\lambda +\mu ) . \Vect{x} = (\lambda . \Vect{x}) + (\mu . \Vect{x})$;
  \item $\forall\lambda \in\K$, $\forall(\Vect{x},\Vect{y})\in E^2$, $\lambda . (\Vect{x}+\Vect{y}) = (\lambda . \Vect{x}) + (\lambda . \Vect{y})$;
  \item $\forall\Vect{x}\in E$, $1_\K. \Vect{x} = \Vect{x}$;
  \item $\forall(\lambda ,\mu )\in\K^2$, $\forall\Vect{x}\in E$, $\lambda . (\mu . \Vect{x}) = (\lambda \mu ) . \Vect{x}$.
  \end{enumerate}
\end{enumerate}
Un élément d'un $\K$-espace vectoriel est appelé un \defi{vecteur} et est noté dans ce cours avec une flèche  $\Vect{x}$. Un élément du corps $\K$ est un \defi{scalaire} et est noté dans ce cours à l'aide d'une lettre grecque, $\lambda$.   
\end{Definition}


\begin{Exemple}

\begin{itemize}
\item les n-uplets $\K^n$ muni des lois usuelles, 
\item
  les matrices $\mathcal{M}_{n,p}(\K)$ muni des lois usuelles,
\item
 si $X$ est un ensemble et $E$ un $\K$-espace vectoriel, l'ensemble des fonctions $\mathcal{F}(X,E)$ muni des lois usuelles.
\end{itemize}
\end{Exemple}


\end{document}
