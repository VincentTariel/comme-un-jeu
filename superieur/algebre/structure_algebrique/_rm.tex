\documentclass{book}
\usepackage{commeunjeustyle}
\newtcolorbox{Exemple1}[1][]{
    breakable,
	colback=white,
	colbacktitle=white,
	coltitle=black,
	bottomrule=0pt,
	toprule=0pt,
	leftrule=1pt,
	rightrule=0pt,
	titlerule=0pt,
	arc=0pt,
	outer arc=0pt,
	colframe=black, 
	title= Exemple  \bfseries{#1},
	enhanced,
	attach boxed title to top left={yshift=-0.1in,xshift=-0.1in},
	boxed title style={colframe=white }
}


\begin{document}
\begin{Exemple1}[toto]%label=introduction
L'enfant apprend d'abord : 
\begin{itemize}
\item à \impo{compter} le nombre d'objets en itérant l'ajout d'une unité à partir de 0 c'est à dire en construisant l'ensemble $\N$  avec l'opération "suivant"
\begin{center}
\begin{tikzpicture}
\draw[color=colorprop] (0,0) rectangle (1,1);
\node at (0.5,-0.3){$0$};
\draw[color=colorprop] (1.5,0) rectangle (2.5,1);
\draw[color=colordef] (2,0.5) circle (0.15);
\node at (2,-0.3){$1$};
\draw[color=colorprop] (3,0) rectangle (4,1);
\draw[color=colordef] (3.3,0.2) circle (0.15);
\draw[color=colordef] (3.7,0.7) circle (0.15);
\node at (3.5,-0.3){$2$};
\draw[color=colorprop] (4.5,0) rectangle (5.5,1);
\draw[color=colordef] (4.8,0.75) circle (0.15);
\draw[color=colordef] (4.7,0.2) circle (0.15);
\draw[color=colordef] (5.3,0.5) circle (0.15);
\node at (5,-0.3){$3$};
\end{tikzpicture}
\end{center}
\end{itemize}
\end{Exemple1}

\begin{Exemple}%label=introduction
L'enfant apprend d'abord : 
\begin{itemize}
\item à \impo{compter} le nombre d'objets en itérant l'ajout d'une unité à partir de 0 c'est à dire en construisant l'ensemble $\N$  avec l'opération "suivant"
\begin{center}
\begin{tikzpicture}
\draw[color=colorprop] (0,0) rectangle (1,1);
\node at (0.5,-0.3){$0$};
\draw[color=colorprop] (1.5,0) rectangle (2.5,1);
\draw[color=colordef] (2,0.5) circle (0.15);
\node at (2,-0.3){$1$};
\draw[color=colorprop] (3,0) rectangle (4,1);
\draw[color=colordef] (3.3,0.2) circle (0.15);
\draw[color=colordef] (3.7,0.7) circle (0.15);
\node at (3.5,-0.3){$2$};
\draw[color=colorprop] (4.5,0) rectangle (5.5,1);
\draw[color=colordef] (4.8,0.75) circle (0.15);
\draw[color=colordef] (4.7,0.2) circle (0.15);
\draw[color=colordef] (5.3,0.5) circle (0.15);
\node at (5,-0.3){$3$};
\end{tikzpicture}
\end{center}
\end{itemize}
\end{Exemple}





\end{document}
