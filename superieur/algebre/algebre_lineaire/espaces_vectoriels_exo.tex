
\documentclass{book}
\usepackage{commeunjeustyle}

\begin{document}
\chapter*{Espaces vectoriels : exercices}



\section{Espace vectoriel}
\subsection{Structure}
\begin{Exercice}[Propriétés des espaces-vectoriels]
Soit $(E,+,.)$ un $\K     $-espace vectoriel.
Pour tout $x\in E$, on note $-x$ le symétrique de $x$ pour la loi $+$.
Pour $(\lambda   ,\mu    )\in \K     ^2$ et $(x,y)\in E^2$, montrer que:
\begin{enumerate}
\item $\lambda   x = 0_E$ si et seulement si $\lambda   = 0_\K     $ ou $x = 0_E$;
\item $(-\lambda   ) x = \lambda   (-x) = -(\lambda   x)$;
\item $(\lambda   - \mu    )x = \lambda   x - \mu    x$;
\item $\lambda   (x-y) = \lambda   x - \lambda   y$.
\end{enumerate}
\end{Exercice}
\subsection{Sous-espace vectoriel}
\begin{Exercice}[Sous espaces-vectoriels de $R^n$]
Parmi les ensembles suivants, lesquels sont, ou ne sont pas, des sous-espaces vectoriels?
\begin{enumerate}
\item $E_1=\{(x,y,z)\in\mathbb R^3;\ x+y+3z=0\}$;
\item $E_2=\{(x,y,z)\in\mathbb R^3;\ x+y+3z=2\}$;
\item $E_3=\{(x,y,z,t)\in\mathbb R^4;\ x=y=2z=4t\}$;
\item $E_4=\{(x,y)\in\mathbb R^2;\ xy=0\}$;
\item $E_5=\{(x,y)\in\mathbb R^2;\ y=x^2\}$;
\item $E_6=\{(x,y,z)\in\mathbb R^3;\ 2x+3y-5z=0\}\cap\{(x,y,z)\in\mathbb R^3;\ x-y+z=0\}$;
\item $E_7=\{(x,y,z)\in\mathbb R^3;\ 2x+3y-5z=0\}\cup\{(x,y,z)\in\mathbb R^3;\ x-y+z=0\}$.
\end{enumerate}
\end{Exercice}


\begin{Exercice}[Sous espaces-vectoriels]
Déterminer si les ensembles suivants sont ou ne sont pas des sous-espaces vectoriels :
\begin{enumerate}
\item $E_1=\{P\in\mathbb R[X];\ P(0)=P(2)\}$;
\item $E_2=\{P\in\mathbb R[X];\ P'(0)=0\}$;
\item Pour $A\in\mathbb R[X]$ non-nul fixé, $E_3=\{P\in\mathbb R[X]; A\text{ divise }P\}$;
\item $\mathcal D$ l'ensemble des fonctions de $\mathbb R$ dans $\mathbb R$ qui sont dérivables;
\item $E_4$, l'ensemble des solutions de l'équation différentielle $y'+a(x) y=0$, où $a$ est une fonction.
\item $E_5$, l'ensemble des solutions de l'équation différentielle $y'+a(x) y=x$,où $a$ est une fonction.
\end{enumerate}
\end{Exercice}

 \begin{Exercice}[Sous espaces-vectoriels des suites]
Les ensembles suivants sont-ils des $\K  $-espaces vectoriels?
\begin{enumerate}
\item L'ensemble des suites réelles convergentes.
\item L'ensemble des suites réelles convergentes vers $0$.
\item L'ensemble des suites réelles convergentes vers $1$.
\item L'ensemble des suites réelles bornées.
\item L'ensemble des suites réelles croissantes.
\item L'ensemble des suites réelles monotones.
\item L'ensemble des suites réelles non convergentes.
\item L'ensemble des suites réelles périodiques à partir d'un certain rang.
\item L'ensemble des fonctions lipschitziennes de $\R $ dans $\R $.
\item L'ensemble des fonctions paires de $\R $ dans $\R $.
\item L'ensemble des fonctions de $\R $ dans $\R $ qui prennent la valeur $\beta$ en $\alpha$.
\end{enumerate}
\end{Exercice}

\subsection{Espace vectoriel engendré par une famille}
 \begin{Exercice}[Polynôme]
Démontrer que :
$$\R_2[X]=\Vectt \left( (X-1)^2,(X-1)(X+1),(X+1)^2\right).$$ 
\end{Exercice}

 \begin{Exercice}[Système d'équations]
Donner un système d'équations des espaces vectoriels engendrés par les vecteurs suivants :
\begin{enumerate}
\item $u_1=(1,2,3)$;
\item $u_1=(1,2,3)$ et $u_2=(-1,0,1)$;
\item $u_1=(1,2,0)$, $u_2=(2,1,0)$ et $u_3=(1,0,1)$.
\end{enumerate}
\end{Exercice}
 \begin{Exercice}[**, trigonométrie]
Démontrer que :
$$\forall n\in\N : \quad \Vectt \left( x\mapsto \cos(kx) \right)_{0\leq k\leq n}= \Vectt \left( x\mapsto \cos^k x \right)_{0\leq k\leq n}.$$ 
\end{Exercice}

\section{Familles libres, génératrices et bases}



\subsection{Combinaison linéaire}
 \begin{Exercice}[Combinaison linéaire]
Les vecteurs $u$ suivants sont-ils combinaison linéaire des vecteurs $u_i$?
\begin{enumerate}
\item $E=\mathbb R^2$, $u=(1,2)$, $u_1=(1,-2)$, $u_2=(2,3)$;
\item $E=\mathbb R^2$, $u=(1,2)$, $u_1=(1,-2)$, $u_2=(2,3)$, $u_3=(-4,5)$;
\item $E=\mathbb R^3$, $u=(2,5,3)$, $u_1=(1,3,2)$, $u_2=(1,-1,4)$;
\item $E=\mathbb R^3$, $u=(3,1,m)$, $u_1=(1,3,2)$, $u_2=(1,-1,4)$ (discuter suivant la valeur de $m$).
\end{enumerate}
\end{Exercice}
 \begin{Exercice}[Combinaison linéaire]
\'Emile achète pour sa maman une bague contenant 2g d'or, 5g de cuivre et 4g d'argent. Il la paie 6200 euros.\newline
Paulin achète pour sa maman une bague contenant 3g d'or, 5g de cuivre et 1g d'argent. Il la paie 5300 euros.\newline
Frédéric achète pour sa chérie une bague contenant 5g d'or, 12g de cuivre et 9g d'argent. Combien va-t-il la payer?
\end{Exercice}


\subsection{Famille libre}
 \begin{Exercice}[Famille libre]
Les familles suivantes sont-elles libres dans $\mathbb R^3$ (ou $\mathbb R^4$ pour la dernière famille)?
\begin{enumerate}
\item $(u,v)$ avec $u=(1,2,3)$ et $v=(-1,4,6)$;
\item $(u,v,w)$ avec $u=(1,2,-1)$, $v=(1,0,1)$ et $w=(0,0,1)$;
\item $(u,v,w)$ avec $u=(1,2,-1)$, $v=(1,0,1)$ et $w=(-1,2,-3)$;
\item $(u,v,w,z)$ avec $u=(1,2,3,4)$, $v=(5,6,7,8)$, $w=(9,10,11,12)$ et $z=(13,14,15,16)$.
\end{enumerate}
\end{Exercice}
 \begin{Exercice}[Famille libre]
On considère dans $\mathbb R^3$ les vecteurs 
$v_1=(1,1,0)$, $v_2=(4,1,4)$ et $v_3=(2,-1,4)$.
\begin{enumerate}
\item Montrer que la famille $(v_1,v_2)$ est libre. Faire de même pour $(v_1,v_3)$, puis pour $(v_2,v_3)$.
\item La famille $(v_1,v_2,v_3)$ est-elle libre?
\end{enumerate}
\end{Exercice}

\begin{Exercice}[Famille libre]
On considère dans $\mathbb R^3$ les vecteurs 
$$v_1=(1,-1,1),\ v_2=(2,-2,2),\ v_3=(2,-1,2).$$
\begin{enumerate}
\item Peut-on trouver un vecteur $w$ tel que $(v_1,v_2,w)$ soit libre?
Si oui, construisez-en un.
\item Même question en remplaçant $v_2$ par $v_3$.
\end{enumerate}
\end{Exercice}

\begin{Exercice}[Degrés échelonnés]
Soit $(P_1,\dots,P_n)$ une famille de polynômes de $\mathbb C[X]$ non nuls, à degrés échelonnés, c'est-à-dire
$\deg(P_1)<\deg(P_2)<\dots<\deg(P_n)$. Montrer que $(P_1,\dots,P_n)$ est une famille libre.
\end{Exercice}


\begin{Exercice}[Fonctions]
Soit $E=\mathcal F(\mathbb R,\mathbb R)$ l'espace vectoriel des fonctions de $\mathbb R$ dans $\mathbb R$. Les familles suivantes, sont-elles indépendantes ? :
\begin{enumerate}
\item $(x\to\sin x,x\to\cos x)$;
\item $(x\to\sin 2x,x\to\sin x,x\to\cos x)$;
\item $(x\to\cos 2x,x\to\sin^2 x,x\to\cos^2 x)$;
\item $(x\to x,x\to e^x,x\to\sin(x))$.
\end{enumerate}
\end{Exercice}
\subsection{Base et dimension}
\begin{Exercice}[dans $\R^2$]
Montrer que $\left((1,1),(1,-1)\right)$ est une base de $\R^2$ et déterminer les coordonnées du vecteur $(4,-2)$ dans cette base. 
\end{Exercice}

\begin{Exercice}[**, dans $\R^n$]
Montrer que $\left(1+X,X+X^2,\dots, X^{n-1}-X^n\right)$ est une base de $\R^n$, pour tout $n\in\N^*$. 
\end{Exercice}
\begin{Exercice}[Déterminer une base d'un sous-espace vectoriel]
Dans $\R^4$, on considère l'ensemble $F$ des vecteurs $(x_1, x_2, x_3, x_4)$ vérifiant $x_1 + x_2 + x_3 + x_4 = 0$. L'ensemble $F$
est-il un sous-espace vectoriel de $\R^4$ ? Si oui, en donner une base.
\end{Exercice}
\section{Somme de sous-espaces vectoriels}
\begin{Exercice}
On considère dans $\mathbb R^4$ les cinq vecteurs suivants : $v_1=(1,0,0,1)$, $v_2=(0,0,1,0)$, $v_3=(0,1,0,0)$, $v_4=(0,0,0,1)$ et $v_5=(0,1,0,1)$. Dire si les sous-espaces vectoriels suivants sont supplémentaires dans $\mathbb R^4$.
\begin{enumerate}
\item $\textrm{vect}(v_1,v_2)$ et $\textrm{vect}(v_3)$?
\item $\textrm{vect}(v_1,v_2)$ et $\textrm{vect}(v_4,v_5)$?
\item $\textrm{vect}(v_1,v_3,v_4)$ et $\textrm{vect}(v_2,v_5)$?
\item $\textrm{vect}(v_1,v_4)$ et $\textrm{vect}(v_3,v_5)$?
\end{enumerate}
\end{Exercice}

\begin{Exercice}
Par des considérations géométriques répondez aux questions suivantes :
\begin{enumerate}

\item  Deux droites vectorielles de $\mathbb R^3$
sont-elles supplémentaires ?
\item Deux plans vectoriels de $\mathbb R^3$
sont-ils supplémentaires ?
\item A quelle condition un plan vectoriel et une droite vectorielle de $\mathbb R^3$
sont-ils supplémentaires ?
\end{enumerate}
\end{Exercice}

\begin{Exercice}
Trouver un système générateur des sous-espaces vectoriels suivants de $\mathbb R^3$:
\begin{enumerate}
\item $F=\{(x,y,z)\in\mathbb \mathbb R^3;\ x+2y-z=0\}$;
\item $G=\{(x,y,z)\in\mathbb \mathbb R^3;\ x-y+z=0\textrm{ et }2x-y-z=0\}$.
\end{enumerate}
\end{Exercice}
\begin{Exercice}
Soit
$$E = \{(u_n)_{n\in N} \in  \mathbb R^\mathbb N : (u_n)\text{ converge}.\}$$
Montrer que l'ensemble des suites constantes et l'ensemble des suites convergeant vers 0 sont des sous-espaces supplémentaires dans E.
\end{Exercice}

\begin{Exercice}
Soit $E$ l'espace vectoriel des suites réelles, 
$$F=\{(u_n)\in E;\ \forall n\in\mathbb N,\ u_{2n}=0\}$$
$$G=\{(u_n)\in E;\ \forall n\in\mathbb N,\ u_{2n}=u_{2n+1}\}.$$
Démontrer que $F$ et $G$ sont supplémentaires.
\end{Exercice}

\begin{Exercice}
Soit $E=\mathcal F(\mathbb R,\mathbb R)$ l'espace vectoriel des fonctions de $\mathbb R$ dans $\mathbb R$.
On note $F$ le sous-espace vectoriel des fonctions paires (ie $f(-x)=f(x)$ pour tout $x\in\mathbb R$)
et $G$ le sous-espace vectoriel des fonctions impaires (ie $f(-x)=-f(x)$ pour tout $x\in\mathbb R$).
Montrer que $F$ et $G$ sont supplémentaires.
\end{Exercice}
\begin{Exercice}
On considère l'espace vectoriel $\mathcal{M}_n(\mathbb R)$ des matrices carrées d'ordre $n$   à coefficients réels.

Une matrice $M$  de  $\mathcal{M}_n(\mathbb R)$ est dite symétrique si elle est égale à sa transposée.

Une matrice  $M$ de $\mathcal{M}_n(\mathbb R)$  est dite antisymétrique si elle est égale à l'opposée de sa transposée.

On appelle $\mathcal{S}_n(\mathbb R)$  le sous-ensemble de $\mathcal{M}_n(\mathbb R)$  formé des matrices symétriques, et $\mathcal{A}_n(\mathbb R)$  celui formé des matrices antisymétriques.
\begin{enumerate}
\item Démontrer que $\mathcal{S}_n(\mathbb R)$  et  $\mathcal{A}_n(\mathbb R)$ sont des sous-espaces vectoriels de   $\mathcal{M}_n(\mathbb R)$.
\item Démontrer que $\mathcal{S}_n(\mathbb R)$  et  $\mathcal{A}_n(\mathbb R)$ sont supplémentaires.
\end{enumerate}
\end{Exercice}

\begin{Exercice}
Soit $E$ l'espace vectoriel des fonctions de $\mathbb R$ dans $\mathbb R$, $F$ le sous-espace vectoriel des fonctions périodiques de période 1 et $G$ le sous-espace vectoriel des fonctions $f$ telles que $\lim_{+\infty}f=0$. Démontrer que $F\cap G=\{0\}$. Est-ce que $F$ et $G$ sont supplémentaires?
\end{Exercice}

\begin{Exercice}
Soit $A\in\mathbb R[X]$ un polynôme non-nul et $F=\{P\in\mathbb R[X];\ A\textrm{ divise }P\}$.
Montrer que $F$ est un sous-espace vectoriel de $\mathbb R[X]$ et trouver un supplémentaire à $F$.
\end{Exercice}

\begin{Exercice}
Soient $F$ et $G$ deux sous-espaces vectoriels d'un espace vectoriel $E$ tels que
$F+G=E$. Soit $F'$ un supplémentaire de $F\cap G$ dans $F$. Montrer que
$F'\oplus G=E$.
\end{Exercice}



\end{document}
