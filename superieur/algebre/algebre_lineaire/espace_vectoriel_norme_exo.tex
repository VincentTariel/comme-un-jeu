\documentclass{book}
\usepackage{commeunjeustyle}
\begin{document}
\section{Espace vectoriel normé}
\subsection{Norme}
\begin{Exercice}[Normes sur $\K^n$]
Pour $\vec{x} =(x_1,\dots,x_n)\in \K^n$, on pose
\begin{enumerate}
\item $\norme{\vec{x}}_1 = \sum_{k=1}^n |x_k|$
\item $\norme{\vec{x}}_2 = \sqrt{\sum_{k=1}^n |x_k|^2}$
\item $\norme{\vec{x}}_\infty = \max\Big(|x_1|,\dots,|x_n|\Big)$
\end{enumerate}
Démontrer que $\norme{\,}_\infty$, $\norme{\,}_1$ et $\norme{\,}_2$ sont trois normes de $\K^n$.
\end{Exercice}
\begin{Exercice}[Normes sur des espaces de fonctions]
\begin{enumerate}
\item Soit $\mathcal{B}(I, \K)$ l'espace des fonctions bornées sur l'intervalle I de $\R$ à valeurs dans $\K$.  Pour $f \in  \mathcal{B}(I, \K)$, on pose :
$$\norme{f}_\infty=\sup_{x\in I}|f(x)|.$$
Démontrer que $\norme{\,}_\infty$ est une norme sur $\mathcal{B}(I, \K)$.
\item Soit $L_1(I, \K)$ l'espace des fonctions continues et intégrables sur l'intervalle $I$ de $\R$ à valeurs dans $\K$. Pour $f \in L_1(I, \K)$, on pose :
$$\norme{f}_1=\int_I|f(x)|\,\mathrm{dx}.$$
Démontrer que  $\norme{\,}_1$ est une norme sur $L_1(I, \K)$.
\item Soit $L_2(I, \K)$ l'espace des fonctions continues de carré intégrables sur l'intervalle $I$ de $\R$ à valeurs dans $\K$. Pour $f \in L_2(I, \K)$, on pose :
$$\norme{f}_1=\int_I|f(x)|\,\mathrm{dx}.$$
Démontrer que  $\norme{\,}_2$ est une norme sur $L_2(I, \K)$.
\end{enumerate}
\end{Exercice}
\begin{Exercice}[Normes sur des espaces de suites]
\begin{enumerate}
\item Soit $\mathcal{B}(I, \K)$ l'espace des fonctions bornées sur l'intervalle I de $\R$ à valeurs dans $\K$.  Pour $f \in  \mathcal{B}(I, \K)$, on pose :
$$\norme{f}_\infty=\sup_{x\in I}|f(x)|.$$
Démontrer que $\norme{\,}_\infty$ est une norme sur $\mathcal{B}(I, \K)$.
\item Soit $L_1(I, \K)$ l'espace des fonctions continues et intégrables sur l'intervalle $I$ de $\R$ à valeurs dans $\K$. Pour $f \in L_1(I, \K)$, on pose :
$$\norme{f}_1=\int_I|f(x)|\,\mathrm{dx}.$$
Démontrer que  $\norme{\,}_1$ est une norme sur $L_1(I, \K)$.
\item Soit $L_2(I, \K)$ l'espace des fonctions continues de carré intégrables sur l'intervalle $I$ de $\R$ à valeurs dans $\K$. Pour $f \in L_2(I, \K)$, on pose :
$$\norme{f}_1=\int_I|f(x)|\,\mathrm{dx}.$$
Démontrer que  $\norme{\,}_2$ est une norme sur $L_2(I, \K)$.
\end{enumerate}
\end{Exercice}
\begin{Exercice}[normes sur les matrices]
Pour $A\in \MnK$, $A = \Big( a_{i,j} \Big)_{1\leq i,j\leq n}$, on pose:
\begin{itemize}
\item $\norme{A}_1 = \sum_{i=1}^n \sum_{j=1}^n |a_{i,j}|$;
\item $\norme{A}_2 = \sqrt{\Tr(\transposee{\bar A}A)}$;
\item $\norme{A}_\infty= \max_{1\leq i,j\leq n} |a_{i,j}|$;
\item $ N(A) = \max_{1\leq i\leq n} \sum_{j=1}^n |a_{i,j}|$.
\end{itemize}
\begin{enumerate}
\item Montrer qu'il s'agit de normes sur $ \MnK$.
\item Montrer que la norme $ \MnK$ est une \emph{norme d'algèbre}, c.-à-d. que $$\forall A,B \in E,\quad N(AB) \leq N(A) N(B).$$
\end{enumerate}
\end{Exercice}
\begin{Exercice}[Normes de polynômes]
Pour $P=\sum_{k=0}^d a_k X^k \in \K[X]$, on pose:
\begin{itemize}
\item $ N_1(P) = \sum_{k=0}^d |a_k|$;
\item $ N_2(P) = \sqrt{\sum_{k=0}^d |a_k|^2}$;
\item $ N_\infty(P) = \max_{0\leq k\leq d} |a_k|$;
\item $ N(P) = \sup_{x\in[0,1]} |a_k|$.
\end{itemize}
Démontrer qu'il s'agit de normes sur $E$.
\end{Exercice}
\begin{Exercice}[Normes d'applications linéaires]
Soient $E$,$F$ deux $\K$-espaces vectoriels normés et $\mathcal{L}_c(E,F)$ l'espace des applications linéaires continues de $E$ dans $F$.\\
Démontrer  $$ \|u\|=\sup_{x\neq 0}\frac{ \| u(x)\|_F }{\|x\|_E}$$ est une norme sur $\mathcal{L}_c (E,F)$
\end{Exercice}


\begin{Exercice}[Intérieur]
Soit $(E, N)$ un espace vectoriel normé. Soit $F$ un sous-espace vectoriel strict de $E$. Montrer que l'intérieur de $F$ est l'ensemble vide.
\end{Exercice}

\begin{Exercice}[Intérieur]
Démontrer que $\GLnK$ est dense dans $\MnK$
\end{Exercice}


\subsection{Suite}
\begin{Exercice}[Unicité de la limite]
Démontrer que si une suite converge, alors sa limite est unique.
\end{Exercice}

\begin{Exercice}[Bornée]
Démontrer que si une suite converge, alors elle est bornée.
\end{Exercice}
\begin{Exercice}[Suite extraite]
Si une suite converge, alors toute suite extraite converge.
\end{Exercice}


\subsection{Fonctions lipschitziennes}
\begin{Exercice}[Application norme]
Soit $(E, N)$ un espace vectoriel normé.
Démontrer que l'application norme est 1-lipschitzienne.
\end{Exercice}
\begin{Exercice}[Application norme]
Soit $(E, N)$ un espace vectoriel normé.
Démontrer que dans un espace vectoriel normé de dimension finie, tout sous-espace vectoriel est un fermé.
\end{Exercice}
%\Exercice[Translatée d'une boule]
%Soit $(E,\|\cdot\|)$ un espace vectoriel normé, $x,y\in E$. Démontrer que $$x+\bar B(y,r)=\bar B(x+y,r)$$.

%\section{Topologie sur les espaces vectoriels normés}
%\subsection{Boules et normes}
%
%
%\Exercice[Translatée d'une boule]
%Soit $(E,\|\cdot\|)$ un espace vectoriel normé, $x,y\in E$. Démontrer que $$x+\bar B(y,r)=\bar B(x+y,r)$$.
%

%
%
%
%\Exercice[normes de fonctions]
%
%Soit $I$ un intervalle de $?$.
%\begin{enumerate}
%\item
%  On note $\mathscr{L}^1$ l'ensemble des fonctions $f$ continues et intégrables de $I$ dans $??$.
%  Pour $f ? \mathscr{L}^1$, on note
%  \[ \Norm{f}_1 = ?_I \abs{f(x)} \D x. \]
%  Montrer qu'il s'agit d'une norme sur $\mathscr{L}^1$.
%
%\item
%  On note $\mathscr{L}^2$ l'ensemble des fonctions $f$ continues de $I$ dans $??$ telles que $f^2$ est intégrable sur $I$.
%  Pour $f ? \mathscr{L}^2$, on note
%  \[ \Norm{f}_2 = ?{?_I \abs{f(x)}^2 \D x}. \]
%  Montrer que $\mathscr{L}^2$ est un £ev. et que $\Norme_2$ est une norme sur $\mathscr{L}^2$.
%
%\item
%  On note $\mathscr{L}^?$ l'ensemble des fonctions $f$ continues et bornées de $I$ dans $??$.
%  Pour $f ? \mathscr{L}^?$, on note
%  \[ \Norm{f}_? = \sup_{x?I} \abs{f(x)}. \]
%  Montrer qu'il s'agit d'une norme sur $\mathscr{L}^?$.
%
%\end{enumerate}
%% Exercice 1465
%
%
%\Exercice[Inégalités sur les normes]
%Soit $(E,\|\cdot\|)$ un espace vectoriel normé. 
%\begin{enumerate}
%\item Démontrer que, pour tous $x,y\in E$, on a 
%$$\|x\|+\|y\|\leq \|x+y\|+\|x-y\|.$$
%En déduire que 
%$$\|x\|+\|y\|\leq 2\max(\|x+y\|,\|x-y\|).$$
%La constante $2$ peut elle être améliorée?
%\item On suppose désormais que la norme est issue d'un produit scalaire. Démontrer que, pour tous $x,y\in E$, on a 
%$$(\|x\|+\|y\|)^2\leq \|x+y\|^2+\|x-y\|^2.$$
%En déduire que 
%$$\|x\|+\|y\|\leq \sqrt 2\max(\|x+y\|,\|x-y\|).$$
%La constante $\sqrt 2$ peut elle être améliorée?
%\end{enumerate}
%
%\Exercice
%
%Soit $E=??^n$ et $\Norme_1$, $\Norme_2$, $\Norme_?$ les normes usuelles\begin{enumerate}
%\item Montrer que pour tout $x???^n$, on a
%  $\Norm{x}_1 ??n \Norm{x}_2 ?n \Norm{x}_??n \Norm{x}_1$.
%\item Montrer que les constantes sont optimales.
%\item En déduire que $u_n \to?$ a la même signification pour les trois normes.
%\end{enumerate}
%
%
%\Exercice[CNS pour avoir une norme]
%Soient $a_1,\dots,a_n$ des réels et $N:\mathbb R^n\to\mathbb R$ définie par 
%$$N(x_1,\dots,x_n)=a_1|x_1|+\dots +a_n |x_n|.$$
%Donner une condition nécessaire et suffisante portant sur les $a_k$ pour que $N$ soit une norme sur $\mathbb R^n$.
%
%\Exercice[Sup de deux normes]
%Soient $N_1$ et $N_2$ deux normes sur un espace vectoriel $E$. On pose $N=\max(N_1,N_2)$.
%Démontrer que $N$ est une norme sur $E$.
%
%
%\Exercice[Espace de matrices]
%On définit une application sur $\mathcal M_n(\mtr)$ en posant
%$$N(A)=n\max_{i,j}{|a_{i,j}|}\textrm{ si }A=(a_{i,j}).$$
%Vérifier que l'on définit bien une norme sur $\mathcal M_n(\mtr)$, puis qu'il s'agit d'une norme
%d'algèbre, c'est-à-dire que 
%$$N(AB)\leq N(A)N(B)\textrm{ pour toutes matrices }A,B\in \mathcal M_n(\mtr).$$
%
%
%\Exercice[Sup sur les polynômes]
%Soit $A$ une partie non vide de $\mathbb R$. Pour tout polynôme $P\in\mathbb R[X]$, on pose
%$$\|P\|_A=\sup_{x\in A}|P(x)|.$$
%Quelles conditions $A$ doit-elle satisfaire pour que l'on obtienne une norme sur $\mathbb R[X]$?
%
%
%\Exercice[Inégalités de Hölder et de Minkowski]
%Soient $(x,y,p,q)\in\mtr_+^*$ tels que $1/p+1/q=1$, et $a_1,\dots,a_n,b_1,\dots,b_n$ $2n$ réels strictement positifs.
%\begin{enumerate}
%\item Montrer que 
%$$xy\leq \frac{1}{p}x^p+\frac{1}{q}y^q.$$
%\item On suppose dans cette question que $\sum_{i=1}^n a_i^p=\sum_{i=1}^n b_i^q=1.$ Montrer que $\sum_{i=1}^n a_ib_i\leq 1$.
%\item En déduire la splendide inégalité de Hölder :
%$$\sum_{i=1}^n a_ib_i\leq\left(\sum_{i=1}^n a_i^p\right)^{1/p}\left(\sum_{i=1}^n b_i^q\right)^{1/q}.$$
%\item On suppose en outre que $p>1$. Déduire de l'inégalité de Hölder l'inégalité de Minkowski :
%$$\left(\sum_{i=1}^n (a_i+b_i)^p\right)^{1/p}\leq\left(\sum_{i=1}^na_i^p\right)^{1/p}+\left(\sum_{i=1}^n b_i^p\right)^{1/p}.$$
%\item On définit pour  $x=(x_1,\dots,x_n)\in \mathbb R^n$ 
%$$\|x\|_p=(|x_1|^p+\dots+|x_n|^p)^{1/p}.$$
%Démontrer que $\|\cdot\|_p$ est une norme sur $\mathbb R^n$.
%\end{enumerate}
%
%
%\Exercice[Normes classiques sur les polynômes]
%Soit $E=\mtr[X]$ l'espace vectoriel des polynômes. On définit sur $E$ trois normes par, si $P=\sum_{i=0}^p a_iX^i$ :
%$$N_1(P)=\sum_{i=0}^p |a_i|,\ N_2(P)=\left(\sum_{i=0}^p |a_i|^2\right)^{1/2},\ N_\infty(P)=\max_i |a_i|.$$
%Vérifier qu'il s'agit de 3 normes sur $\mtr[X]$. Sont-elles équivalentes deux à deux?
%
%
%\Exercice[Normes classiques sur les fonctions continues]
%Soit $E=\mathcal C([0,1],\mathbb R)$. On définit les normes $\|\cdot\|_1$, $\|\cdot\|_2$ et $\|\cdot\|_\infty$ par 
%$$\|f\|_1=\int_0^1|f(t)|dt,\ \|f\|_2=\left(\int_0^1|f(t)|^2\right)^{1/2}\textrm{ et }\|f\|_\infty=\sup_{x\in [0,1]}|f(x)|.$$
%Démontrer que ces trois normes ne sont pas équivalentes deux à deux.
%
%
%\Exercice[Deux normes équivalentes sur $\mathcal C^1$]
%Soit $E=\mathcal C^1([0,1],\mathbb R)$. On définit 
%$$N(f)=|f(0)|+\|f'\|_\infty,\ N'(f)=\|f\|_{\infty}+\|f'\|_\infty.$$
%\begin{enumerate}
%\item Démontrer que $N$ et $N'$ sont deux normes sur $E$.
%\item Démontrer que $N$ et $N'$ sont équivalentes.
%\item Sont-elles équivalentes à $\|\cdot\|_\infty$?
%\end{enumerate}
%
%
%\Exercice[Normes sur les polynômes]
%Soit $a\geq 0$. Pour $P\in\mathbb R[X]$, on définit
%$$N_a(P)=|P(a)|+\int_0^1 |P'(t)|dt.$$
%\begin{enumerate}
%\item Démontrer que $N_a$ est une norme sur $\mathbb R[X]$.
%\item Soit $a,b\geq 0$ avec $a< b$ et $b>1$. Démontrer que $N_a$ et $N_b$ ne sont pas équivalentes.
%\item Démontrer que si $(a,b)\in [0,1]^2$, alors $N_a$ et $N_b$ sont équivalentes.
%\end{enumerate}
%
%\subsection{Ouverts ou fermés}
%\Exercice[Exemples d'ouverts et de fermés de $\mathbb R$]
%Dans l'espace vectoriel normé $\mathbb R$, déterminer si les parties suivantes sont ouvertes ou fermées : $\mathbb N$, $\mathbb Z$, $\mathbb Q$, $[0,1[$, $[0,+?[$, $]0,1[\cup {2}$, $\{1/n, n \in\mathbb N^*\}$, $\bigcap_{n\geq 1}]-1/n,1/n[$.
%\Exercice[Ouverts ou fermés]
%Démontrer que les deux ensembles suivants sont ouverts  :
%$$F=\left\{(x,y)\in\mtr^2;\ x^2<\exp(\sin y)+ 12\right\},\quad\quad G=\{(x,y)\in\mtr^2; -1<\ln (x^2+1)<1\}.$$
%\Exercice[Stabilité]
%Soit $(A_n)_{n \in \mathbb{N}}$ une suite de parties ouvertes de $\mathbb{R}$. 
%\begin{itemize}
%\item Est-ce que la réunion des An est encore une partie
%ouverte ? Et leur intersection ?
%\item Même question pour une famille de parties fermées.
%\end{itemize}
%\subsection{Topologie des espaces de matrices}
%\Exercice[Ouvert/fermé/compact]
%\begin{enumerate}
%\item Démontrer que l'ensemble des matrices symétriques est un fermé de $MnR$.
%\item Montrer que l'ensemble $GLnR$ des matrices inversibles est un ouvert de $MnR$.
%\item Montrer que l'ensemble des matrices orthogonales $OnR$  est un compact de $MnR$.
%\end{enumerate}
%\section{Applications}
%\subsection{Limite de suite}
%
%\Exercice[Convergence des suites extraites]
%Soit $(u_n)$ une suite de nombres réels.
%\begin{enumerate}
%\item On suppose que $(u_{2n})$ et $(u_{2n+1})$ convergent vers la même limite. Prouver que $(u_n)$ est convergente.
%\item Donner un exemple de suite telle que $(u_{2n})$ converge, $(u_{2n+1})$ converge, mais $(u_{n})$ n'est pas convergente.
%\item On suppose que les suites $(u_{2n})$, $(u_{2n+1})$ et $(u_{3n})$ sont convergentes. Prouver que $(u_n)$ est convergente.
%\end{enumerate}
%
%
%\Exercice[Suites extraites vérifiant certaines propriétes]
%Soit $(u_n)$ une suite de nombre réels.
%\begin{enumerate}
%\item On suppose que $(u_n)$ est croissante et qu'elle admet une suite extraite convergente. Que dire de $(u_n)$?
%\item On suppose que $(u_n)$ est croissante et qu'elle admet une suite extraite majorée. Que dire de $(u_n)$?
%\item On suppose que $(u_n)$ n'est pas majorée. Montrer qu'elle admet une suite extraite qui diverge vers $+\infty$.
%\end{enumerate}
%\Exercice[Exemples de valeurs d'adhérence]
%\begin{enumerate}
%\item Quelles sont les valeurs d'adhérence de la suite $(-1)^n$? de la suite $\cos(n\pi/3)$?
%\item Donner un exemple de suite qui ne converge pas et qui possède une unique valeur d'adhérence.
%\end{enumerate}
%
%
%
%\Exercice[Suites de Cauchy]
%Une suite $(u_n)$ de nombre réels est appelée suite de Cauchy si, pour tout $\veps>0$, il existe un entier $N$ tel que,
%pour tout $p,q\geq N$, on a 
%$$|u_p-u_q|<\veps.$$
%\begin{enumerate}
%\item Montrer que toute suite convergente est une suite de Cauchy.
%\item On souhaite prouver la réciproque à la question précédente.
%Soit $(u_n)$ une suite de Cauchy. 
%\begin{enumerate}
%\item Montrer que $(u_n)$ est bornée.
%\item On suppose que $(u_n)$ admet une suite extraite convergente. Montrer que $(u_n)$ est convergente.
%\item Conclure.
%\end{enumerate}
%\end{enumerate}
%
%\subsection{Continuité et limite}
%\Exercice[Continuité et équation fonctionnelle]
%Soit $E$ un espace vectoriel normé, et $h:E\to E$ une application continue admettant une limite $\ell$ en $0$ et vérifiant $h(x)=h(x/2)$ pour tout $x\in E$. Démontrer que $h$ est constante.
%
%\Exercice[Calcul de limites détaillé]
%\begin{enumerate}
%\item Montrer que si $x$ et $y$ sont des réels, on a :
%$$2|xy|\leq x^2+y^2$$
%\item Soit $f$ l'application de $A=\mtr^2\backslash\{(0,0)\}$ dans $\mtr$ définie par 
%$$f(x,y)=\frac{3x^2+xy}{\sqrt{x^2+y^2}}.$$
%Montrer que, pour tout $(x,y)$ de $A$, on a :
%$$|f(x,y)|\leq 4\|(x,y)\|_2,$$
%où $\|(x,y)\|_2=\sqrt{x^2+y^2}.$
%En déduire que $f$ admet une limite en $(0,0)$.
%\end{enumerate}
%
%\Exercice[Diverses limites]
%Les fonctions suivantes ont-elles une limite en l'origine?
%\begin{enumerate}
%\item $\dis f(x,y,z)=\frac{xy+yz}{x^2+2y^2+3z^2}$;
%\item $\dis f(x,y)=\left(\frac{x^2+y^2-1}{x}\sin x,\frac{\sin(x^2)+\sin(y^2)}{\sqrt{x^2+y^2}}\right)$.
%\item $\dis f(x,y)=\frac{1-\cos(xy)}{xy^2}$.
%\end{enumerate}
%
%\Exercice[Prolongement par continuité]
%Démontrer que la fonction définie par $f(x,y)=\frac{\sin (xy)}{xy}$ se
%prolonge en une fonction continue sur $\mathbb R^2$.
%\Exercice[Avec la dérivée]
%
%Soit $f:\mathbb R\to\mathbb R$ une fonction de classe $C^1$. On définit $F:\mathbb R^2\to\mathbb R$ par
%$$F(x,y)=\left\{
%\begin{array}{ll}
%\frac{f(x)-f(y)}{x-y}&\textrm{ si }x\neq y\\
%f'(x)&\textrm{ sinon.}
%\end{array}
%\right.
%$$
%Démontrer que $F$ est continue sur $\mathbb R^2$. 
%\subsection{Application linéaire}
%\Exercice[Sont-elles continues?]
%Déterminer si l'application linéaire $T:(E,N_1)\to (F,N_2)$ est continue dans les cas suivants :
%\begin{enumerate}
%\item $E=\mathcal C([0,1],\mathbb R)$ muni de $\|f\|_1=\int_0^1 |f(t)|dt$ et $T:(E,\|.\|_1)\to (E,\|.\|_1),\ f\mapsto fg$ où
%$g\in E$ est fixé.
%\item $E=\mathbb R[X]$ muni de $\|\sum_{k\geq 0}a_k X^k\|=\sum_{k\geq 0}|a_k|$ et $T:(E,\|.\|)\to (E,\|.\|)$, $P\mapsto
%P'$.
%\item $E=\mathbb R_n[X]$ muni de $\|\sum_{k=0}^n a_k X^k\|=\sum_{k=0}^n |a_k|$ et $T:(E,\|.\|)\to (E,\|.\|)$, $P\mapsto
%P'$.
%\item $E=\mathbb R[X]$ muni de $\|\sum_{k\geq 0}a_k X^k\|=\sum_{k\geq 0}k!|a_k|$ et $T:(E,\|.\|)\to (E,\|.\|)$, $P\mapsto
%P'$.
%\item $E=\mathcal C([0,1],\mathbb R)$ muni de $\|f\|_2=\left(\int_0^1 |f(t)|^2dt\right)^{1/2}$, $F=\mathcal C([0,1],\mathbb R)$ muni de $\|f\|_1=\int_0^1 |f(t)|dt$
%et $T:(E,\|.\|_2)\to (F,\|.\|_1),\ f\mapsto fg$ où
%$g\in E$ est fixé.
%\end{enumerate}
%
%
%
%\Exercice[Endomorphisme] 
%
%Soit $E=\mathcal C([0,1],\mathbb R)$. Pour $f\in E$, on pose
%$$\|f\|_1=\int_0^1 |f(t)|dt,$$
%dont on admettra qu'il s'agit d'une norme sur $E$. Soit
%$\phi$ l'endomorphisme de $E$ défini par 
%$$\phi(f)(x)=\int_0^x f(t)dt.$$
%\begin{enumerate}
%\item Justifier la terminologie : "$\phi$ est un endomorphisme de $E$."
%\item Démontrer que $\phi$ est continue.
%\item Pour $n\geq 0$, on considère $f_n$ l'élément de $E$ défini par
%$f_n(x)=ne^{-nx}$, $x\in[0,1]$. Calculer $\|f_n\|_1$ et $\|\phi(f_n)\|_1$.
%\item On pose $\|\!|\phi\|\!|=\sup_{f\neq 0_E}\frac{\|\phi(f)\|_1}{\|f\|_1}$.
%Déterminer $\|\!|\phi\|\!|$.
%\end{enumerate}
%
%\Exercice[Applications linéaires sur les polynômes]
%Soit $E=\mathbb R[X]$, muni de la norme $\|\sum_i a_i X^i\|=\sum_i |a_i|$.
%\begin{enumerate}
%\item Est-ce que l'application linéaire $\phi:(E,\|.\|)\to (E,\|.\|)$, $P(X)\mapsto P(X+1)$
%est continue sur $E$?
%\item Est-ce que l'application linéaire $\psi:(E,\|.\|)\to (E,\|.\|)$, $P(X)\mapsto AP$, où
%$A$ est un élément fixé de $E$,
%est continue sur $E$?
%\end{enumerate}
%
%
%\Exercice[Norme d'une application linéaire continue]
%Soit $E$ un espace vectoriel normé et $\mathcal L_c(E)$ l'ensemble des applications linéaires continues sur $E$. Pour $u\in\mathcal L_c(E)$, on pose 
%$$\|u\|=\sup\{\|u(x)\|;\ \|x\|\leq 1\}.$$
%\begin{enumerate}
%\item Démontrer que ceci définit une norme sur $\mathcal L_c(E)$.
%\item Démontrer que, pour tout $x\in E$ et tout $u\in\mathcal L_c(E)$, on a 
%$$\|u(x)\|\leq \|u\|\times \|x\|.$$
%En déduire que, pour tous  $u,v\in \mathcal L_c(E)$, alors $\|u\circ v\|\leq \|u\|\times \|v\|.$
%\end{enumerate}
%\subsection{Application Lipschitzienne}
%\Exercice[Fonction usuel]
%Soit $f$ la fonction définie par :$$\Fonction{f}{\mathbb{R}}{\mathbb{R}}{x}{x^2}$$
%$f$ est-elle Lipschitzienne ?
%\Exercice[Dérivée]
%Soit $f$ la fonction définie par :$$\Fn{f}{A\subset\mathbb{R}}{\mathbb{R}}$$ dérivable dont la fonction dérivée est bornée sur $A$.\\
%$f$ est-elle Lipschitzienne ?
%
%\Exercice[Continuité de la norme]
%Soit $E$ un espace vectoriel normé, de norme $\|?\|$.\\ 
%Démontrer que l'application norme : $\|?\|:E\to \mathbb{R}^+$ est continue. 





\end{document}





