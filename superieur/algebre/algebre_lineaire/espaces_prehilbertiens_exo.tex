\documentclass{book}
\usepackage{commeunjeustyle}
\begin{document}
\chapter*{Espaces préhilbertiens réels : exercices}
%% -----------------------------------------------------------------------------
\section{Produit scalaire}
 \begin{Exercice}[*]

Vérifier que $\PS..$ est un produit scalaire sur $E$ dans les cas suivants.
\begin{enumerate}
\item $E = \Mn{n}{\R}$
  et $\PS AB = \tr{ \transposee{A}B}$.\\
\item $E = \mathcal{C}([-1,1],\R)$
  et $\PS fg = \int_{-1}^1 \frac{f(t)g(t)}{\sqrt{1-t^2}}\,\mathrm{dt}$.
\item $E = \{f:\R\to \R\text{ continues}: f^2 \text{ intégrable sur }\R\}$
  et $\PS fg = \int_\R fg$.
\item $E = \R[X]$
  et $ \PS PQ =\int_0^1 P(t)Q(t)\,\mathrm{dt}$.
\item $E = \R[X]$
  et $\PS PQ =\sum_{n=0}^{+\infty} \frac{P(n)Q(n)}{2^n}$.
\end{enumerate}
\end{Exercice}
 \begin{Exercice}[*]

Soit $n\in\N^*$ fixé, $E = \R_n[X]$ et $F = \{P\in E :P(0)=P(1)=0\}$.
Pour $(P,Q)\in E$, on pose \[ \phi(P,Q) = -\int_0^1 (PQ''+P''Q). \]
\begin{enumerate}
\item Vérifier que $F$ est un espace vectoriel.
\item Donner une base et la dimension de $F$.
\item $\phi$ définit-il un produit scalaire sur $E$? sur $F$?
\end{enumerate}
\end{Exercice}

 \begin{Exercice}[*]
Soit $E$ l'ensemble des suites réelles $(u_n)_{n\in\N}$ telles que la série de terme général $u_n^2$ converge.
Pour $u$ et $v$ dans $E$, on pose
\[ \PS uv = \sum_{n=0}^{+\infty} u_n v_n. \]
\begin{enumerate}
\item Montrer que $E$ est un espace vectoriel. On le note usuellement $l^2$.
\item Montrer que $\PS uv$ existe.
\item Montrer qu'il s'agit d'un produit scalaire.
\end{enumerate}
\end{Exercice}
 \begin{Exercice}[Applications de l'inégalité de Cauchy-Schwarz]

Soient $x_1,\dots,x_n\in\R$.
\begin{enumerate}
\item Démontrer que 
$$\left(\sum_{k=1}^n x_k\right)^2\leq n\sum_{k=1}^n x_k^2$$
et étudier les cas d'égalité.
\item On suppose en outre que $x_k>0$ pour chaque $k\in\{1,\dots,n\}$ et que $x_1+\dots+x_n=1$.
Démontrer que 
$$\sum_{k=1}^n \frac 1{x_k}\geq n^2$$
et étudier les cas d'égalité.
\end{enumerate}
\end{Exercice}

\begin{Exercice}[Applications de l'inégalité de Cauchy-Schwarz]
Soit $E=\mathcal C([a,b],\mathbb R^*)$. Déterminer $\inf_{f\in E}\left(\int_a^b f\times \int_a^b \frac 1f\right)$. Cette borne inférieure est-elle atteinte?
\end{Exercice}

\section{Orthogonalité}
 \begin{Exercice}[*]

Soit $(E,\PS.. )$ un espace euclidien et $\mathcal{B} =(\vec{e_1},\dots,\vec{e_n})$ une BON de $E$.
On pose \[ \Fonction{\phi}{\LE\times\LE}{\R}{(u,v)}{\sum_{i=1}^n \PS{u(\vec{e_i})}{v(\vec{e_i})}} \]
Montrer que $\phi$ est un produit scalaire sur $\LE$,
et déterminer une base orthonormale pour ce produit scalaire.
\end{Exercice}
 \begin{Exercice}[Matrice symétrique]
Soit $\mathcal{E} = \Mn{n}{\R}$ muni du produit scalaire usuel 
$\PS AB = \Tr\bigl(\transposee{A} B\bigr)$.\\
Montrer que $ \SnR$ et $\AnR$ sont supplémentaires orthogonaux,
  où $\SnR$ désigne l'ensemble des matrices symétriques
  et $\AnR$ l'ensemble des matrices antisymétriques.
\end{Exercice}
 \begin{Exercice}[Polynômes de Legendre]
On munit le $\R$-espace vectoriel $E = \mathcal{C}([-1,1],\R)$
du produit scalaire usuel défini par
\[ \forall(f,g)\in E^2 :\quad \PS fg = \int_{-1}^1 f(t) g(t) \mathrm{dt}. \]
On pose pour tout $n\in\N$,
\[ L_n(X) = \frac{1}{2^n n!}\frac{\mathrm{d}^n}{\mathrm{d}X^n} \Big[ (X^2-1)^n \Big] \]
Les polynômes $(L_n(X))_{n\in\N}$ s'appellent \impo{polynômes de Legendre}.
On pourra introduire $H_n(X) = (X^2-1)^n$.
\begin{enumerate}
\item Montrer que $L_n$ est un polynôme de degré $n$
  dont on précisera le coefficient dominant.
\item En utilisant la formule de Leibniz, calculer $L_n(1)$ et $L_n(-1)$.
\item Avec une intégration par parties multiple, calculer $\PS{L_n}{L_n}$.
\item Calculer $\PS{Q}{L_n}$ lorsque $Q$ est un polynôme de $\R_{n-1}[X]$.
\item En déduire $\PS{L_n}{L_m}$ lorsque $n?m$.
\item Comparer $(L_n)_{n\in\N}$ à l'orthonormalisée de la base canonique.
\end{enumerate}
\end{Exercice}
 \begin{Exercice}[Polynômes de Tchebychev]
Soit $E = \R[X]$.\\
On pose, pour $(P,Q)\in E^2$,
\[ \PS PQ = \frac 2 \pi \int_{-1}^1\sqrt{1-t^2} \, P(t) Q(t) \mathrm{dt} \]\begin{enumerate}
\item Montrer que $\PS..$ est bien un produit scalaire sur $E$.
\item Soit $n\in\N$. Montrer qu'il existe un unique polynôme $U_n$
  tel que
  \[ \forall \theta\in \R:\quad  \sin\bigl( (n+1)\theta\bigr) = \sin(\theta) U_n(\cos \theta). \]

  Les polynômes $\bigl(U_n(X)\bigr)_{n\in \N}$ s'appellent
 \impo{ polynômes de Tchebychev de seconde espèce}.
\item Comparer $(U_n)_{n\in\N}$ à l'orthonormalisée de la base canonique.
\end{enumerate}
\end{Exercice}
\section{Projection orthogonale}
 \begin{Exercice}[Orthonormalisation de Schmidt]
Dans $\mathbb R^3$ muni du produit scalaire canonique, orthonormaliser en suivant le procédé de Schmidt la base suivante :
$$u=(1,0,1),\ v=(1,1,1),\ w=(-1,-1,0).$$
\end{Exercice}
 \begin{Exercice}[Trouver une base orthonormale]

Déterminer une base orthonormale de $\mathbb R_2[X]$ muni du produit scalaire 
$$\langle P,Q\rangle=\int_{-1}^1 P(t)Q(t)dt.$$
\end{Exercice}
 \begin{Exercice}[Projection orthogonale dans $\mathbb R^4$]
Soit $E=\mathbb R^4$ muni de son produit scalaire canonique et de la base canonique $\mathcal B=(e_1,e_2,e_3,e_4)$. On considère $G$ le sous-espace vectoriel défini par les équations 
$$\left\{
\begin{array}{rcl}
x_1+x_2&=&0\\
x_3+x_4&=&0.
\end{array}
\right.
$$
\begin{enumerate}
\item Déterminer une base orthonormale de $G$.
\item Déterminer la matrice dans $\mathcal B$ de la projection orthogonale $p_G$ sur $G$.
\item Soit $x=(x_1,x_2,x_3,x_4)$ un élément de $E$. Déterminer la distance de $x$ à $G$.
\end{enumerate}
\end{Exercice}


 \begin{Exercice}[Matrice symétrique]
Soit $\mathcal{E} = \Mn{n}{\R}$ muni du produit scalaire usuel
$\PS AB = \Tr\bigl(\transposee{ A} B\bigr)$.
\begin{enumerate}
\item Montrer que $\mathcal{S}$ et $\mathcal{A}$ sont supplémentaires orthogonaux,
  où $\mathcal{S}$ désigne l'ensemble des matrices symétriques
  et $\mathcal{A}$ l'ensemble des matrices antisymétriques.
\item Montrer que
  \[ \forall A\in\mathcal{E} :\quad \Tr A \leq \sqrt{n\Tr(\transposee{A} A)}. \]
  Étudier le cas d'égalité.
\item Pour $A\in\mathcal{E}$, on pose
  \[ \Fonction{f_A}{\mathcal{S}}{\R}{S}{\sum_{i=1}^n \sum_{j=1}^n (a_{i,j} - s_{i,j})^2} \]
  Déterminer le minimum de $f_A$ et la matrice qui réalise ce minimum.
\end{enumerate}
\end{Exercice}
 \begin{Exercice}[Optimum]
Déterminer $(a,b,c,d)\in\R^4$ tels que l'intégrale
\[ \int_{-\pi/2}^{\pi/2} \Big( \sin x - ax^3 - bx^2 - cx - d \Big)^2 \mathrm{dx} \]
soit minimale.
\end{Exercice}
\end{document}





