\documentclass{book}
\usepackage{commeunjeustyle}

\begin{document}
\chapter*{Matrices : feuille d'exercices}


\section{Calcul matriciel} 
\begin{Exercice}[Des calculs de produits]
Calculer lorsqu'ils sont définis les produits $AB$ et $BA$ dans chacun des cas suivants :
\begin{enumerate}
\item $\displaystyle A= \left(\begin{array}{cc}
1  &  0  \\
0  &  0 \\
\end{array}\right),\quad 
B=\left(\begin{array}{cc}
0  &  0  \\
0  &  1 \\
\end{array}\right)
$
\item $\displaystyle A=\left(\begin{array}{ccc}
0 & 2 & 1\\
1 & 1 & 0\\
-1&-2 &-1\\
\end{array}\right),\quad
B=\left(\begin{array}{ccc}
2 & 0 & 1\\
-1& 1 & 2\\
\end{array}\right)
$
\item $\displaystyle A=\left(\begin{array}{cc}
1 & 2 \\
1 & 1 \\
0 & 3 \\
\end{array}\right),
\quad B=\left(\begin{array}{cccc}
-1 & 1 & 0& 1 \\
2 & 1 & 0& 0 \\
\end{array}\right)
$
\end{enumerate}
\end{Exercice}
\begin{Exercice}[Commutant]
Soient $a$ et $b$ des réels non nuls, et $A=\left( \begin{array}{cc} a & b\\ 
 0 &a \end{array} \right).$ Trouver toutes les matrices $B\in\mathcal M_2(\mathbb R)$ qui commutent avec $A$,
c'est-à-dire telles que $AB=BA$.

\end{Exercice}

\begin{Exercice}[Annulateur]

On considère les matrices $A=\left(\begin{array}{ccc}
1&0&0\\
0&1&1\\
3&1&1\end{array}\right)$, $B=\left(\begin{array}{ccc}
1&1&1\\
0&1&0\\
1&0&0\end{array}\right)$ et $C=\left(\begin{array}{ccc}
1&1&1\\
1&2&1\\
0&-1&-1\end{array}\right)$. Calculer $AB$, $AC$. Que constate-t-on? La matrice $A$ peut-elle être inversible? 
Trouver toutes les matrices $F\in\mathcal M_3(\mathbb R)$ telles que $AF=0$ (où $0$ désigne la matrice nulle).
\end{Exercice}

\begin{Exercice}[Produit non commutatif]

Déterminer deux éléments $A$ et $B$ de
$\mathcal M_2({\mathbb R})$ tels que : $AB=0$ et $BA\not = 0$.
\end{Exercice}


\begin{Exercice}[Matrices stochastiques en petite taille]
On dit qu'une matrice $A\in\mathcal M_n(\mathbb R)$ est une matrice stochastique si la somme des coefficients
sur chaque colonne de $A$ est égale à 1. Démontrer que le produit de deux matrices stochastiques est
une matrice stochastique si $n=2$. Reprendre la question si $n\leq 1$.
\end{Exercice}

\begin{Exercice}[Puissance $n$-ième, par récurrence]

Calculer la puissance $n$-ième des matrices suivantes : 
$$A=\left(\begin{array}{cc}
1&-1\\
-1&1\\
\end{array}\right),\ B=\left(\begin{array}{cc}
1&1\\
0&2\\
\end{array}\right).$$
\end{Exercice}

\begin{Exercice}[Puissance $n$-ième - avec la formule du binôme]

Soit $$A=\left(
\begin{array}{ccc}
1&1&0\\
0&1&1\\
0&0&1
\end{array}\right),\quad
I=\left(
\begin{array}{ccc}
1&0&0\\
0&1&0\\
0&0&1
\end{array}\right)\textrm{ et }
B=A-I.$$
Calculer $B^n$ pour tout $n\in\mathbb N$. En déduire $A^n$.
\end{Exercice}

\begin{Exercice}[Puissance $n$-ième - avec un polynôme annulateur]

\begin{enumerate}
\item Pour $n\geq 2$, déterminer le reste de la division euclidienne de $X^n$ par $X^2-3X+2$.
\item Soit  $A=\begin{pmatrix} 
0&1&-1\\
-1&2&-1\\
1&-1&2
\end{pmatrix}$. Déduire de la question précédente la valeur de $A^n$, pour $n\geq 2$.
\end{enumerate}
\end{Exercice}

\begin{Exercice}[Inverser une matrice  sans calculs!]

\begin{enumerate}
\item Soit $A=\left(
\begin{array}{ccc}
-1&1&1\\
1&-1&1\\
1&1&-1
\end{array}\right)$. Montrer que $A^2=2I_3-A$, en déduire que $A$ est inversible et calculer $A^{-1}$.
\item  Soit $ A=\begin{pmatrix} 1 & 0 & 2 \cr
0 & -1 & 1 \cr
1 & -2 & 0 \cr \end{pmatrix} .$ Calculer $
A^3-A .$ En déduire
que $ A $ est inversible puis déterminer $ A^{-1} .$ 
\item Soit $A=\begin{pmatrix} 
0&1&-1\\
-1&2&-1\\
1&-1&2
\end{pmatrix}$. Calculer $A^2-3A+2I_3$. En déduire que $A$ est inversible, et calculer $A^{-1}$.
\end{enumerate}
\end{Exercice}

\begin{Exercice}[Inverse avec calculs!]

Dire si les matrices suivantes sont inversibles et, le
cas échéant, calculer leur inverse :
$$A=\left(
\begin{array}{rcl}
1&1&2\\
1&2&1\\
2&1&1
\end{array}
\right),\quad
B=\left(
\begin{array}{rcl}
0&1&2\\
1&1&2\\
0&2&3
\end{array}
\right).$$
\end{Exercice}

\begin{Exercice}[Matrice nilpotente]

Soit $A\in\mathcal M_n(\mathbb R)$ une matrice nilpotente, c'est-à-dire qu'il existe $p\geq 1$ tel que $A^p=0$. Démontrer que la matrice $I_n-A$ est inversible, et déterminer son inverse.
\end{Exercice}
\section{Puissance d'une matrice}
\begin{Exercice}[Modélisation matricielle de suites définies par une récurrences linéaires]
On considère les suites $(u_n)$, $(v_n)$ et $(w_n)$ définies par leur premier terme
$u_0$, $v_0$ et $w_0$ et les relations suivantes :
$$\left\{
\begin{array}{rcl}
u_{n+1}&=&-4u_n-6v_n\\
v_{n+1}&=&3u_n+5v_n\\
w_{n+1}&=&3u_n+6v_n+5w_n
\end{array}
\right.$$
pour $n\geq 0$. On pose $X_n=\left(
\begin{array}{c}u_n\\v_n\\w_n\end{array}\right)$.
\begin{enumerate}
\item Exprimer $X_{n+1}$ en fonction d'une matrice $A$ et de $X_n$.
\item  Exprimer $X_{n}$ en fonction d'une matrice $A$ et de $X_0$.
\end{enumerate}
\end{Exercice}
\begin{Exercice}[ Puissance d'une matrice triangulaire de diagonale nulle]

Démontrer qu'une matrice triangulaire de diagonale nulle  $A=\begin{pmatrix}0&a_{1,2}&\cdots &\cdots &a_{1,n}\\0&0&&&a_{2,n}\\\vdots &\ddots &\ddots &&\vdots \\\vdots &&\ddots &\ddots &a_{n-1,n}\\0&\cdots &\cdots &0&0\\\end{pmatrix}$ est nilpotente
\end{Exercice}

\section{Trace}
\begin{Exercice}[Matrice nilpotente]

Soit $A\in\mathcal M_n(\mathbb R)$ une matrice nilpotente, c'est-à-dire qu'il existe $p\geq 1$ tel que $A^p=0$. Démontrer que la matrice $I_n-A$ est inversible, et déterminer son inverse.
\end{Exercice}

\begin{Exercice}[Matrice nilpotente]
Soit $A\in\mathcal M_n(\mathbb R)$.\\
Démontrer que $\Tr(\transposee{A} A)=\sum_{i=1}^n\sum_{j=1}^p a_{ij}^2$

\end{Exercice}
\end{document}





