\exo{ 0402 , tariel, 01/11/2017, 1-2, {}}[*, Altern�e]
\'Etudier la nature des s�ries $\sum u_n$ suivantes :
$$\begin{array}{lll}
\mathbf 1.\ u_n=\frac{\sin n^2}{n^2}&&\mathbf 2.\ u_n=\frac{(-1)^n\ln n}{n}\\
\mathbf 3.\  u_n=\frac{\cos (n^2\pi)}{n\ln n}
\end{array}$$
%
\begin{correction}
\begin{enumerate}
\item On a : $|u_n|\leq \frac{1}{n^2}$, donc la s�rie converge absolument.
\item La s�rie est altern�e, et la suite $(\ln(n)/n$ est d�croissante  � partir d'un certain rang et  converge vers 0. Donc par application du crit�re des s�ries altern�es, la s�rie converge.
\item On a $cos(n^2 \pi)=(?1)^n$. La suite $(1/\ln(n)n$ est d�croissante et  converge vers 0.  Donc la s�rie converge par application du crit�re des s�ries altern�es. 
\end{enumerate}
\end{correction}
\finexo