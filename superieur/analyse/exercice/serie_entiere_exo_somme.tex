 \exo{ 0402 , tariel, 01/11/2017, 1-2, {}}[*, Rayon de convergence et somme]
Calculer le rayon de convergence  puis la somme de :
\label{exo:2}
\begin{multicols}{3}
  	\begin{enumerate}
    \item $ \sum\limits_{n \geqslant 0} {\frac{{n - 1}}{{n!}}x^n } $
     
    \item $ \sum\limits_{n \geqslant 0} {\frac{{(n + 1)(n - 2)}}{{n!}}x^n } $

	\item $ \sum\limits_{n \geqslant 2} {\frac{{x^n }}{{n(n - 1)}}} $
	
	\item $ \sum\limits_{n \geqslant 0} \frac{3n }{n+2} x^n$
	
	\item $ \sum\limits_{n \geqslant 0} \frac{x^{4n} }{ (4n)! } $	
  \end{enumerate} 
\end{multicols} 
\begin{correction}
\renewcommand {\theenumi}{\alph{enumi}}
  	\begin{enumerate}
    \item %$ \sum\limits_{n \geqslant 0} {\frac{{n - 1}}{{n!}}x^n } $
    \begin{itemize}
    	\item Rayon de CV $R = + \infty$ (utiliser la r�gle de D'Alembert \dots)
    	
    	\item Pour tout $x \in \R$ :

L'�galit� est bien valable car chacune des s�ries du membre de droite a aussi $+ \infty$ pour rayon de CV 
    	
\hspace{0.7cm} $ \sum\limits_{n = 0}^{ + \infty } \frac{n - 1}{n!}x^n =  
\sum\limits_{n = 0}^{ + \infty } {\frac{{n}}{{n!}}x^n }  - \sum\limits_{n = 0}^{ + \infty } {\frac{{1}}{{n!}}x^n } $
    	
\hspace{0.7cm} $\phantom{ \sum\limits_{n = 0}^{ + \infty } \frac{n - 1}{n!}x^n  } =  
\sum\limits_{n = 1}^{ + \infty } \frac{{n}}{{n!}}x^n  - \sum\limits_{n = 0}^{ + \infty } \frac{x^n}{n!}  $

\hspace{0.7cm} $\phantom{ \sum\limits_{n = 0}^{ + \infty } \frac{n - 1}{n!}x^n  } = 
x \sum\limits_{n = 1}^{ + \infty } \frac{1}{ (n-1)! }x^{n-1}  - \sum\limits_{n = 0}^{ + \infty } \frac{x^n}{n!}  $

\hspace{0.7cm} $\phantom{ \sum\limits_{n = 0}^{ + \infty } \frac{n - 1}{n!}x^n  } = 
\sum\limits_{n = 0}^{ + \infty } \frac{1}{ n! }x^{n}  - \sum\limits_{n = 0}^{ + \infty } \frac{x^n}{n!}  $

\hspace{0.7cm} $\phantom{ \sum\limits_{n = 0}^{ + \infty } \frac{n - 1}{n!}x^n  } = x \mathrm{e}^x  -  \mathrm{e}^x $
    \end{itemize}

     
    \item %$ \sum\limits_{n \geqslant 0} \frac{(n + 1)(n - 2)}{n!}x^n  $
    \begin{itemize}
    	\item Rayon de CV $R = + \infty$ (utiliser la r�gle de D'Alembert \dots)
    	
    	\item Pour tout $x \in \R$ : sachant que $(n+1)(n-2) = n^2 - n - 2 = n(n-1) - 2$
 
L'�galit� est bien valable car chacune des s�ries du membre de droite a aussi $+ \infty$ pour rayon de CV 
    	
\hspace{0.7cm} $\sum\limits_{n = 0}^{ + \infty } \frac{(n + 1)(n - 2)}{n!}x^n  =  
\sum\limits_{n = 0}^{ + \infty } \frac{ n(n-1) }{ n! }x^n   - 2 \sum\limits_{n = 0}^{ + \infty } \frac{ 1 }{ n!} x^n  $
    	
\hspace{0.7cm} $\phantom{  \sum\limits_{n = 0}^{ + \infty } \frac{(n + 1)(n - 2)}{n!}x^n   } =  
\sum\limits_{n = 2}^{ + \infty } \frac{ n(n-1) }{ n! } x^n  - 2 \sum\limits_{n = 0}^{ + \infty } \frac{x^n}{n!}  $

\hspace{0.7cm} $\phantom{  \sum\limits_{n = 0}^{ + \infty } \frac{(n + 1)(n - 2)}{n!}x^n   } =   
x^2 \sum\limits_{n = 2}^{ + \infty } \frac{ 1 }{ (n-2) ! } x^{n-2}  - 2 \sum\limits_{n = 0}^{ + \infty } \frac{x^n}{n!}  $

\hspace{0.7cm} $\phantom{  \sum\limits_{n = 0}^{ + \infty } \frac{(n + 1)(n - 2)}{n!}x^n   } =  
x^2 \sum\limits_{n = 0}^{ + \infty } \frac{ 1 }{ n! } x^n  - 2 \sum\limits_{n = 0}^{ + \infty } \frac{x^n}{n!}  $

\hspace{0.7cm} $\phantom{  \sum\limits_{n = 0}^{ + \infty } \frac{(n + 1)(n - 2)}{n!}x^n   } = 
x^2 \mathrm{e}^x  -  2 \mathrm{e}^x $
    \end{itemize}
    


	\item %$ \sum\limits_{n \geqslant 2} \frac{x^n }{n(n - 1)} $
    \begin{itemize}
    	\item Rayon de CV $R = 1$ (utiliser la r�gle de D'Alembert \dots)
    	
    	\item Pour tout $x \in ]-1 ; 1[$ : sachant que par d�composition en �l�ments simples $\frac{1}{n(n-1)} = \frac{1}{n-1} - \frac{1}{n}$
    	
    	 L'�galit� est bien valable car chacune des s�ries du membre de droite a aussi $1$ pour rayon de CV  
    	
\hspace{0.7cm} $\sum\limits_{n = 2}^{ + \infty } \frac{x^n }{n(n - 1)}  =  
\sum\limits_{n = 2}^{ + \infty } \frac{x^n }{n - 1} -  \sum\limits_{n = 2}^{ + \infty } \frac{ x^n }{ n }$
    	
\hspace{0.7cm} $\phantom{  \sum\limits_{n = 2}^{ + \infty } \frac{x^n }{n(n - 1)}  }=    
x \sum\limits_{n = 2}^{ + \infty } \frac{x^{n-1} }{n - 1} -  \sum\limits_{n = 1}^{ + \infty } \frac{ x^n }{ n } + x$

\hspace{0.7cm} $\phantom{  \sum\limits_{n = 2}^{ + \infty } \frac{x^n }{n(n - 1)}  }=    
x \sum\limits_{n = 1}^{ + \infty } \frac{x^{n}}{n} -  \sum\limits_{n = 1}^{ + \infty } \frac{ x^n }{ n } + x$

\hspace{0.7cm} $\phantom{  \sum\limits_{n = 2}^{ + \infty } \frac{x^n }{n(n - 1)}  }=  
- x \ln(1-x)   +  \ln(1-x) + x $
    \end{itemize}	

	\item %$ \sum\limits_{n \geqslant 0} \frac{3n }{n+2} x^n$
    \begin{itemize}
    	\item Rayon de CV $R = 1$ (utiliser la r�gle de D'Alembert \dots)
    	
    	\item Pour tout $x \in ]-1 ; 1[$ : sachant que par d�composition en �l�ments simples $\frac{3n}{n+2} = 3\left( 1 - 2 \frac{1}{n+2} \right)$
   
  L'�galit� ci-dessous est vraie car chacune des s�ries du membre de droite a aussi $1$ pour rayon de CV     
    	
\hspace{0.7cm} $\sum\limits_{n = 0}^{ + \infty } \frac{3n }{n+2}   =  
3 \left( \sum\limits_{n = 0}^{ + \infty } x^n  -  2 \sum\limits_{n = 0}^{ + \infty } \frac{ x^n }{ n+2 } \right)$

Soit $x=0$ et dans ce cas la somme est nulle, soit $x \neq 0$ et on peut factoriser par $x$ :

\hspace{0.7cm} $\sum\limits_{n = 0}^{ + \infty } \frac{3n }{n+2}   =  
3 \left( \sum\limits_{n = 0}^{ + \infty } x^n  -  \frac{2}{x^2} \sum\limits_{n = 0}^{ + \infty } \frac{ x^{n+2} }{ n+2 } \right)$
    	
\hspace{0.7cm} $\phantom{  \sum\limits_{n = 0}^{ + \infty }  \frac{3n }{n+2}   }  =    
3 \left( \sum\limits_{n = 0}^{ + \infty } x^n  -  \frac{2}{x^2} \sum\limits_{ n = 2 }^{ + \infty } \frac{ x^{n} }{ n }  \right)$

\hspace{0.7cm} $\phantom{  \sum\limits_{n = 0}^{ + \infty }  \frac{3n }{n+2}   }  =    
3 \left( \sum\limits_{n = 0}^{ + \infty } x^n  -  \frac{2}{x^2} \sum\limits_{ n = 1 }^{ + \infty } \frac{ x^{n} }{ n } + -  \frac{2}{x^2} x   \right)$

\hspace{0.7cm} $\phantom{  \sum\limits_{n = 0}^{ + \infty }  \frac{3n }{n+2}   }  =    
3 \left(  \frac{1}{1-x}  + \frac{2}{x^2} \ln(1-x) +  \frac{2}{x}   \right)$
    \end{itemize}		
	
	\item %$ \sum\limits_{n \geqslant 0} \frac{x^{4n} }{ (4n)! } $	
	Il faut le voir mais on a imm�diatement $R = + \infty$ par lin�arit� avec :
	
	\hspace{0.7cm}$ \sum\limits_{n \geqslant 0} \frac{x^{4n} }{ (4n)! } = \tfrac{1}{2} \left( \mathrm{ch}(x) + \cos(x) \right)$
	
 \end{enumerate} 
\end{correction}
\finexo 