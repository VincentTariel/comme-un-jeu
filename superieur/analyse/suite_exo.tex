\documentclass{book}
\usepackage{commeunjeustyle}
\begin{document}
\begin{Exercice}[Convergence implique bornée]
 Montrer que toute suite convergente est bornée.
\begin{Correction}
Soit $(u_n)$ une suite convergeant vers $l \in \R$. Par
d\'efinition
$$\forall \epsilon > 0 \quad \exists N \in \N \quad  \forall n\geqslant N \qquad |u_n-\ell| < \epsilon.$$
Choisissons $\epsilon = 1$, nous obtenons le  $N$ correspondant.
Alors pour $n\geqslant N$, nous avons $|u_n-\ell| < 1$ ; 
autrement dit $\ell -1 <
u_n < \ell + 1$. Notons $M = \max_{n=0,\ldots,N-1}  \{u_n\}$  et
puis $ M' = \max (M,\ell+1)$. Alors  pour tout $n \in \N$ $u_n
\leq M'$. De m\^eme en posant $m = \min_{n=0,\ldots,N-1} \{u_n\}$ et
$m' = \min(m,\ell -1)$ nous obtenons pour tout $n\in \N$, $u_n
\geq m'$.
\end{Correction}
\end{Exercice}
\begin{Exercice}[Nature]
\'Etudier la nature des suites suivantes, et déterminer leur limite éventuelle :
$$\begin{array}{lcl}
\displaystyle \mathbf 1.\ u_n=\frac{\sin(n)+3\cos\left(n^2\right)}{\sqrt{n}}&&\displaystyle \mathbf 2.\ u_n=\frac{2n+(-1)^n}{5n+(-1)^{n+1}}\\
\displaystyle \mathbf 3.\ u_n=\frac{n^3+5n}{4n^2+\sin(n)+\ln(n)}&&\displaystyle \mathbf 4.\ u_n=
\sqrt{2n+1}-\sqrt{2n-1}\\
\displaystyle \mathbf 5.\ u_n=3^ne^{-3n}.
\end{array}$$
\end{Exercice}
\begin{Exercice}[Somme télescopique]
\begin{enumerate}
\item Déterminer deux réels $a$ et $b$ tels que
$$\frac{1}{k^2-1}=\frac{a}{k-1}+\frac{b}{k+1}.$$
\item En déduire la limite de la suite 
$$u_n=\sum_{k=2}^n \frac{1}{k^2-1}.$$
\item Sur le même modèle, déterminer la limite de la suite 
$$v_n=\sum_{k=0}^n\frac{1}{k^2+3k+2}.$$
\end{enumerate}
\end{Exercice}

\begin{Exercice}[Exemple de suites adjacentes]
Démontrer que les suites $(u_n)$ et $(v_n)$ données ci-dessous forment
des couples de suites adjacentes.
$$
\begin{array}{ll}
\mathbf{1.}\quad \displaystyle u_n=\sum_{k=1}^n \frac1{k^2}\textrm{ et }v_n=u_n+\frac 1n\\
\mathbf{2.}\quad \displaystyle u_n=\sum_{k=1}^n\frac{1}{k+n}\textrm{ et }v_n=\sum_{k=n}^{2n}\frac 1k.
\end{array}$$
\end{Exercice}

\begin{Exercice}[Avec des quantificateurs]
Soit $(u_n)$ une suite de nombres réels. \'Ecrire avec des quantificateurs les propositions suivantes : 
\begin{enumerate}
\item $(u_n)$ est bornée.
\item $(u_n)$ n'est pas croissante.
\item $(u_n)$ n'est pas monotone.
\item $(u_n)$ n'est pas majorée.
\item $(u_n)$ ne tend pas vers $+\infty$.
\end{enumerate}
\end{Exercice}

\begin{Exercice}[Moyenne de Ces\`aro]
Soit $(u_n)_{n\geq 1}$ une suite réelle. On pose $S_n=\frac{u_1+\dots+u_n}{n}$.
\begin{enumerate}
\item On suppose que $(u_n)$ converge vers 0. Soient $\varepsilon>0$ et $n_0\in\mathbb N$ tel que, pour
$n\geq n_0$, on a $|u_n|\leq\varepsilon$.
\begin{enumerate}
\item Montrer qu'il existe une constante $M$ telle que, pour $n\geq n_0$, on a 
$$|S_n|\leq \frac{M(n_0-1)}{n}+\varepsilon.$$
\item En déduire que $(S_n)$ converge vers 0.
\end{enumerate}
\item On suppose que $u_n=(-1)^n$. Que dire de $(S_n)$? Qu'en déduisez-vous?
\item On suppose que $(u_n)$ converge vers $l$. Montrer que $(S_n)$ converge vers $l$.
\item On suppose que $(u_n)$ tend vers $+\infty$. Montrer que $(S_n)$ tend vers $+\infty$.
\item Trouver un exemple de suite qui diverge mais dont la moyenne de Cesàro converge.
\end{enumerate}

\end{Exercice}
% Exercice 98

\begin{Exercice}[Convergence des suites extraites]
Soit $(u_n)$ une suite de nombres réels.
\begin{enumerate}
\item On suppose que $(u_{2n})$ et $(u_{2n+1})$ convergent vers la même limite. Prouver que $(u_n)$ est convergente.
\item Donner un exemple de suite telle que $(u_{2n})$ converge, $(u_{2n+1})$ converge, mais $(u_{n})$ n'est pas convergente.
\item On suppose que les suites $(u_{2n})$, $(u_{2n+1})$ et $(u_{3n})$ sont convergentes. Prouver que $(u_n)$ est convergente.
\end{enumerate}
\end{Exercice}


\begin{Exercice}[Suite Héron]
Etudier la suite :
  $$u_0 > \sqrt{2} , u_{n+1} = \frac12\left(u_n + \frac{2}{u_n}\right)$$
\end{Exercice}
\end{document}
