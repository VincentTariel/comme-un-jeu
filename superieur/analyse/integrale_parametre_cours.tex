\documentclass{book}
\usepackage{commeunjeustyle}
\begin{document}
\chapter*{Intégrale à paramètre}
Une intégrale  à paramètre est une fonction d'une variable entière ou
réelle, définie par intégration d'une fonction de deux variables, la variable d'intégration et le paramètre, :
$$F(x)=\int_J f(x,t)\,\mathrm dt.$$
On dit que la fonction $F$ est une intégrale dépendant du paramètre $x$. On définit ainsi une fonction $F$ dont on veut découvrir les propriétés à partir
de celles de $f$.\\ 
En première partie, nous verrons le cas où le paramètre est un entier $$u_n=F(n)=\int_J f(n,t)\,\mathrm dt=\int_J f_n(t)\,\mathrm dt.$$ Le cours sur les suites et séries de fonctions nous a fourni un théorème pour permuter la limite et l'intégrale $$\lim_{n\to \infty}\int_J f_n(t)\,\mathrm dt=\int_J \lim_{n\to \infty}f_n(t)\,\mathrm dt$$ si la suite de fonctions convergent uniformément sur $J$ et si $J$ est un segment. Que peut on dire lorsque $J$ est un intervalle quelconque ? Lorsqu'il y a convergence simple et non plus uniforme ?
Notre but est  de donner des théorèmes plus généraux.\\
En seconde partie,  nous verrons le cas où le paramètre est un réelle et nous introduirons des théorèmes la permutation de la limite et de l'intégrale 
$$\lim_{x\to x_0}\int_J f(x,t)\, \mathrm dt=\int_J \lim_{x\to x_0} f(x,t)\, \mathrm dt,$$
et la permutation de la dérivée et  l'intégrale 
$$\frac{d}{dx} \int_J f(x,t)\,\mathrm dt = \int_J \frac{\partial f}{\partial x }(x,t) \,\mathrm dt,$$ 
sur un intervalle $J$ quelconque. 





\section{Intégrale à paramètre entier : $ u_n =F(n)= \int_J f_n(t)\mathrm dt.$}
\subsection{Permutation limite-intégrale ou série-intégrale: cas segment et convergence uniforme}
\textit{Rappel du cours sur les suites et séries de fonctions.}
\begin{Theoreme}[Permutation limite/intégrale]
Soit $(f_n)_{n\in\N}$ une suite de fonctions de $[a,b]$ dans $\K$.
On suppose que:
\begin{itemize}
\item
  pour tout $n\in \N $, $f_n$ est continue sur le segment $[a,b]$;
\item
  $(f_n)_{n\in\N}$ converge uniformément vers $f$ sur $[a,b]$.
\end{itemize}

Alors $f$ est continue (donc intégrable) sur $[a,b]$ et
\[  \lim_\ninf \int_a^b f_n(t)\mathrm dt = \int_a^b \lim_\ninf f_n(t)\mathrm dt = \int_a^b f(t)\mathrm dt.  \]
\end{Theoreme}
\begin{Theoreme}[Permutation somme/intégrale]
Soit $\sum_n f_n $ une série de fonctions de $[a,b]$ dans $\K$.
On suppose que:
\begin{itemize}
\item
  pour tout $n\in \N $, $f_n$ est continue sur le segment $[a,b]$
\item
  la série de fonctions $\sum_n f_n$ converge uniformément sur $[a,b]$ vers $f$
\end{itemize}
Alors $f$ est continue sur $[a,b]$ (donc intégrable) et
\[ \sum_{n=0}^{+\infty} \int_a^b f_n(t)\mathrm dt = \int_a^b \sum_{n=0}^{+\infty} f_n(t)\mathrm dt= \int_a^b f(t)\mathrm dt.   \]
\end{Theoreme}
\subsection{Permutation limite-intégrale ou série-intégrale : cas  intervalle ouvert et convergence simple}
\begin{Theoreme}[Convergence dominée] Soit $(f_n)_{n\in\N}$ une suite de fonctions continues par morceaux de $J$ dans $\K$, et $f,\phi:J\to\K$ continues par morceaux avec $\phi $ positive. On suppose que
\begin{enumerate}
\item \impo{Convergence simple de $(f_n)$ vers $f$} : pour tout $t\in J$, la suite numérique $(f_n(t))$ converge vers $f(t)$;
\item \impo{Hypothèse de domination} : pour tout $t\in J$ et tout $n\geq 1, |f_n(t)|\leq\phi (t)$,  ;
\item \impo{Intégrabilité de la fonction dominante} :  la fonction $\phi $ est intégrable sur $J$.
\end{enumerate}
Alors toutes les fonctions $f_n$ et $f$ sont intégrables sur $J$, et on a :
$$\lim_{n\to +\infty}\int_J f_n(t)\,\mathrm dt= \int_J \lim_{n\to +\infty} f_n(t)\,\mathrm dt=\int_J f(t)\,\mathrm dt$$.
\end{Theoreme}
\begin{Remarque}
Le théorème précédent est valable sur un intervalle quelconque qui peut aussi être un segment.\\
L'hypothèse de convergence uniforme est remplacée par une autre hypothèse, l'"hypothèse de domination". Il s'agit de
fournir une fonction $\phi$ (continue par morceaux et intégrable sur $J$) majorant toutes les fonctions $f_n$. Ceci signifie que l'on
doit majorer chaque $|f_n(t)|$ par une expression \impo{dépendante de $t$ et indépendante de $n$} et qui soit une fonction de $t$
intégrable sur $J$. 
\end{Remarque}

\begin{Exemple}
Déterminer la limite, lorsque $n$ tend vers $+\infty$, de la  suite suivante :
$$u_n=\int_1^{+\infty}\frac{dt}{1+t^n}.$$
Soit $f_n:t\mapsto(\frac{1}{1+t^n})$.
\begin{enumerate}
\item \textit{Convergence simple sur $[1,+\infty[$:}\\ 
Soit $t>1$ fixé.\\
La suite numérique $(\frac{1}{1+t^n})_{n\in\mathbb{N}}$ converge vers 0 si $t>1$ et $1$ si $t=1$. Donc la suite de fonctions $(f_n)_{n\in\mathbb{N}}$ converge simplement vers la fonction  $t\mapsto \begin{cases} 0\text{ si } t>1 \\1\text{ si }t=1 \end{cases}$.
\item \textit{Hypothèse de domination :}\\ 
Pour $n\geq 2$ et $t\geq  1$, on a
$$\left|\frac{1}{1+t^n}\right| \leq \frac{1}{1+t^2}\leq \frac{1}{t^2}=\phi(t).$$.
\item \textit{Intégrabilité de la fonction dominante :}\\
$\int_1^{+\infty}\frac{dt}{t^2}$ converge et donc intégrable sur $[1,+\infty[$.
\end{enumerate}
Les hypothèses du théorème de convergence dominée sont vérifiées. On en déduit que :
$$\lim_{n\to +\infty}u_n=\lim_{n\to +\infty} \int_1^{+\infty}\frac{dt}{1+t^n} \underset{\text{convergence dominée}}{=}\int_1^{+\infty}\lim_{n\to +\infty} \frac{1}{1+t^n}\,dt= \int_1^{+\infty} 0\, dt=0.$$
\end{Exemple}
\begin{Exemple}[Wallis]
Déterminer la limite, lorsque $n$ tend vers $+\infty$, de la  suite suivante :
$$W_n=\int_0^{\pi/2}\sin^n t \, \mathrm dt.$$
Soit $f_n:t\mapsto \sin^n t$.
\begin{enumerate}
\item \textit{Convergence simple sur $[0,\pi/2]$:}\\ 
Soit $t\in[0,\pi/2]$ fixé.\\
La suite géométrique $(\sin^n t)_{n\in\mathbb{N}}$ converge vers 0 si $t\neq \pi/2$ et $1$ si $t= \pi/2$. Donc la suite de fonctions $(f_n)_{n\in\mathbb{N}}$ converge simplement vers la fonction $f:t\mapsto \begin{cases} 0\text{ si } t\neq \pi/2\\1\text{ si }t=\pi/2 \end{cases}  $ qui est continue par morceaux sur $[0,\pi/2]$.
\item \textit{Hypothèse de domination :}\\ 
Pour $n\geq 1$ et $t\in [0,\pi/2] $, on a
$$\left|f_n(t)\right| \leq 1=\phi(t).$$.
\item \textit{Intégrabilité de la fonction dominante :}\\
$\int_0^{\pi/2}1 \mathrm dt $ converge et donc intégrable sur $[0,\pi/2]$.
\end{enumerate}
Les hypothèses du théorème de convergence dominée sont vérifiées. On en déduit que :
$$\lim_{n\to +\infty}W_n\underset{\text{convergence dominée}}{=}\int_0^{\pi/2}\lim_{n\to +\infty} \sin^n t\, \mathrm dt= \int_0^{\pi/2} 0\, \mathrm dt=0.$$
\end{Exemple}



\begin{Theoreme}[Intégration terme à terme] Soit $\sum_n f_n $ une série de fonctions continues par morceaux de $J$ dans $\K$, et $f:J\to\K$ continue par morceaux. On suppose que
\begin{enumerate}
\item \impo{Convergence simple de $\sum_n f_n $ vers $f$}: pour tout $t\in J$, la suite numérique $\sum_n f_n(t)$ converge vers $f(t)$;
\item \impo{Hypothèse de domination} : la série numérique $\sum_n \int_J|f_n(t)|dt$ converge.
\end{enumerate} 
Alors f est intégrable sur $J$, et on a :
$$\sum_{n=0}^{+\infty} \int_J f_n(t)\,\mathrm dt = \int_J \sum_{n=0}^{+\infty} f_n(t)\,\mathrm dt= \int_J f(t)\,\mathrm dt.$$
\end{Theoreme}

\begin{Exemple} Démontrer que $\sum_{n=0}^{+\infty}\int_0^1 t^{2n}(1-t)\,\mathrm dt=\ln(2)$.
\begin{enumerate}
\item \textit{Convergence simple sur $[0,1[$:}\\ 
La série géométrique $\sum_n t^{2n}$ converge simplement vers $t\mapsto \frac{1}{1-t^2}.$ Donc la  suite $(1-t)\sum_n t^{2n}$ converge simplement vers $t\mapsto \frac{1}{1-t}.$
\item \textit{Hypothèse de domination :}\\ 
On a $\int_0^1 t^{2n}(1-t) \mathrm dt = [\frac{t^{2n+1}}{2n+1}-\frac{t^{2n+2}}{2n+2}]_0^1=\frac{1}{2n+1}-\frac{1}{2n+2}=\frac{1}{(2n+1)(2n+2)}\sim_{+\infty}\frac{1}{4n^2}$. Comme  la série de Riemann $\sum_n \frac{1}{n^2}$ converge, la série $\sum_n \int_0^1 t^{2n}(1-t) \,\mathrm dt$ converge par règle d'équivalence sur les séries à termes positifs.
\end{enumerate}
Les hypothèses du théorème d'intégration terme à terme sont vérifiées. On en déduit que :
$$\sum_{n=0}^{+\infty}\int_0^1 t^{2n}(1-t)\mathrm dt=\int_0^1 \sum_{n=0}^{+\infty} t^{2n}(1-t)\mathrm dt=\int_0^1 \frac{1}{1+t}\mathrm dt=\ln(2).$$
\end{Exemple}
%% -----------------------------------------------------------------------------
\section{Intégrale à paramètre réelle : $F(x) = \int_J f(x,t) \mathrm dt$ }
\textit{Contexte :}\\
 Soit $I$ et $J$ deux intervalles (non vides) de $\R$ et
\[ \Fonction{f}{I×J}\K{(x,t)}{f(x,t).} \]
On s'intéresse à la fonction $F$ définie par
\[ F(x) = \int_J f(x,t) \mathrm dt. \]

\subsection{Domaine de définition $I=D_F$}
\begin{Proposition} 
La fonction $F$ est définie sur $D_F$ si, pour tout
$x\in D_F$ fixé, l'intégrale $\int_J f(x,t)\, \mathrm dt $ converge.
\end{Proposition}
\begin{Proposition} 
$x \in D_F$ si et seulement si  $\int_J f(x,t)\, \mathrm dt$ existe  (est bien définie).\\ 
Il y a deux possibilités :
\begin{itemize}
\item  si $J$ est un segment et si $t\mapsto f (x ,t )$ est continue alors l'intégrale est bien définie.
\item Si $J$ n'est pas un segment et si $t\mapsto f (x ,t )$ est continue, alors l'intégrale est bien définie si l'intégrale
impropre $\int_J f(x,t) \mathrm dt$ converge.
\end{itemize}
\end{Proposition}
\begin{Exemple}Démontrer que $F:x\mapsto \int_{0}^{+\infty} e^{-xt^2}\, \mathrm dt$ est définie sur $]0,+\infty[$.\\
Soit $x\in ]0,+\infty[$ fixé. La fonction $t\mapsto e^{-xt^2}$ est continue et positive sur $[0,+\infty[$.\\
En $+\infty$, on a : $$e^{-xt^2}={\underset {\overset {t\rightarrow +\infty }{}}{o}}\left(\frac{1}{t^2}\right).$$
Comme $\int_{1}^{+\infty} \frac{1}{t^2}\,\mathrm dt$ converge, par théorème de comparaison, $\int_{1}^{+\infty}e^{-xt^2}\, \mathrm dt$ converge. Par raccordement, $\int_{0}^{+\infty} e^{-xt^2}\, \mathrm dt$ converge, donc $F$ est bien définie sur   $]0,+\infty[$.
\end{Exemple}
\subsection{Continuité}

\begin{Theoreme}[continuité sous le signe $\int$]
On suppose:
\begin{enumerate}
\item
  $f$ est \impo{continue par rapport à $x$}, c.-à-d.
  \begin{itemize}
  \item
    pour tout $t\in J$ fixé, la fonction $x \mapsto f(x,t)$ est continue sur $I$.
  \end{itemize}
\item
  $f$ est \impo{continue par morceaux par rapport à $t$}, c.-à-d.
  \begin{itemize}
  \item
    pour tout $x\in I$ fixé, la fonction $t \mapsto f(x,t)$ est continue par morceaux sur $J$.
  \end{itemize}
\item
  \impo{Hypothèse de domination :}\\
  Il existe une fonction $\phi$ telle que
  \begin{itemize}
  \item
    pour tout $(x,t)\in I\times J$, on a $\|f(x,t)\|\leq\phi (t)$,
  \item
    $\phi $ est continue par morceaux et intégrable sur $J$.
  \end{itemize}
\end{enumerate}
Alors:
\begin{itemize}
\item
  La fonction $F \colon x \mapsto \int_J f(x,t) \mathrm dt$ est définie et continue sur $I$.
\end{itemize}
\end{Theoreme}
\begin{Remarque}
Ce théorème est une interversion de limite et d'intégrale :
$$\lim_{x\to x_0}F(x)=\lim_{x\to x_0}\int_J f(x,t)\, \mathrm dt=\int_J \lim_{x\to x_0} f(x,t)\, \mathrm dt=\int_J f(x_0,t)\, \mathrm dt=F(x_0)$$
\end{Remarque}

\begin{Exemple}Démontrer que $F:x\mapsto \int_{0}^{+\infty} e^{-xt^2} \mathrm dt$ est continue sur $[a,b]$ avec $b>a>0$.\\
\begin{enumerate}
\item \textit{Continuité:}\\ 
Soit $x\in [a,b]$ fixé. La fonction $t\mapsto e^{-xt^2}$ est continue  sur $[0,+\infty[$.\\
Soit $t\in [0,+\infty[$ fixé. La fonction $x\mapsto e^{-xt^2}$ est continue  sur  $[a,b]$.
\item  \textit{Domination:}\\
On a :
$$\forall x\in [a,b], \forall t\in [0,+\infty[: \quad  \|e^{-xt^2}\|\leq e^{-at}.$$
$t\mapsto e^{-at}$ est intégrable sur $[0,+\infty[$.\\
\end{enumerate}
Les hypothèses de théorème de continuité sous le signe $\int$ sont vérifiées donc la fonction $F$ est continue sur $[a,b]$.
\end{Exemple}
Dans certains cas, il n'est pas possible de "dominer" la fonction $f$ sur l'intervalle $I$ tout entier. On peut cependant se contenter de vérifier l'hypothèse de domination sur tout segment $K$ inclus dans $I$, le théorème précédent s'applique encore.
\begin{Theoreme}[continuité sous le signe $\int$ version locale]
On peut remplacer l'hypothèse de domination par:
\begin{itemize}
\item
  \impo{Hypothèse de domination locale:}
  Pour tout segment $K$ inclus dans $I$,
  il existe une fonction $\phi_K \colon J \to \R$ telle que
  \begin{itemize}
  \item
    pour tout $(x,t)\in K\times J$, on a $\|f(x,t)\|\leq\phi _K(t)$,
  \item
    $\phi _K$ est continue par morceaux et intégrable sur $J$.
  \end{itemize}
\end{itemize}
\end{Theoreme}
\begin{Exemple}
Démontrer que $F:x\mapsto \int_{0}^{+\infty} e^{-xt^2}\mathrm dt$ est continue sur $]0,+\infty[$.\\
On a démontré que $F$  est continue pour tout segment $[a,b]$ inclus dans $]0,+\infty[$. Donc $F$ est continue sur $]0,+\infty[$. 
\end{Exemple}
Si l'intervalle d'intégration est un segment $[c,b]$, l'hypothèse de domination est
automatique sur tout segment $[a ,b ] \subset I$ en posant $\phi(t)=M=\sup_{(x,t)\in[a,b]\times[c,d]}|f(x,t)|$ avec $f$ continue (voir cours sur les fonctions de plusieurs variables).
\begin{Exemple}
Démontrer que $F:x\mapsto \int_{0}^{1}\frac{ e^{xt}}{1+t^2}\,\mathrm dt$ est continue sur $\R$.\\
Soit $[a,b]\subset \R$.
\begin{enumerate}
\item \textit{Continuité:}\\ 
Soit $x\in [a,b]$ fixé. La fonction $t\mapsto \frac{ e^{xt}}{1+t^2}$ est continue  sur $[0,1]$.\\
Soit $t\in [0,1]$ fixé. La fonction $x\mapsto \frac{ e^{xt}}{1+t^2}$ est continue  sur  $[a,b]$.
\item  \textit{Domination:}\\
On a :
$$\forall x\in [a,b], \forall t\in [0,1]: \quad  \left|\frac{ e^{xt}}{1+t^2}\right|\leq M=\sup_{(x,t)\in[a,b]\times[0,1]}|\frac{ e^{xt}}{1+t^2}| .$$
$t\mapsto M$ est intégrable sur $[0,1]$.\\
\end{enumerate}
Les hypothèses de théorème de continuité sous le signe $\int$ sont vérifiées donc la fonction $F$ est continue sur $[a,b]$.
On a démontré que $F$  est continue pour tout segment $[a,b]$ inclus dans $\R$. Donc $F$ est continue sur $\R$. 
\end{Exemple}


%% -----------------------------------------------------------------------------
\subsection{Dérivabilité}
\begin{Theoreme}[Dérivation sous le signe $\int$]
On suppose:
\begin{enumerate}
\item
   pour tout $x\in I$ fixé, la fonction $t \mapsto f(x,t)$
    est continue par morceaux et intégrable sur $J$.
\item $f$ admet une dérivé partielle, $\frac{\partial f}{\partial x} (x,t)$ vérifiant les hypothèses de théorème continuité sous le signe $\int$ 
\end{enumerate}
Alors:
\begin{itemize}
\item
  La fonction $F \colon x \mapsto \int_J f(x,t) \mathrm dt$ est définie et de classe $\mathcal{C}^1$ sur $I$.
\item
  Pour tout $x\in I$, on a \[  F'(x) = \int_J \frac{\partial f}{\partial x} (x,t) \mathrm dt. \]
\end{itemize}
\end{Theoreme}
\begin{Remarque}
Ce théorème est une interversion de limite et de dérivée :
$$F'(x)=\frac{d}{dx} \int_J f(x,t)\mathrm dt = \int_J \frac{\partial f}{\partial x }(x,t) \mathrm dt.$$
\end{Remarque}
\begin{Exemple}
On pose, pour $a>0$, $F(x)=\int^{+\infty}_{-\infty}e^{-itx}e^{-at^2}dt$ la transformée de Fourier de la Gaussienne.
Démontrer que $F$ est de classe $\mathcal{C}^1$ sur $\R$ et vérifie, pour tout $x\in\R, F'(x)=\frac{-x}{2a}F(x)$, puis
 déterminer l'expression de $F$.
\begin{enumerate}
\item   pour tout $x\in \R$ fixé, la fonction $t \mapsto e^{-itx}e^{-at^2}$
    est continue par morceaux et intégrable sur $J$ car $|e^{-itx}e^{-at^2}|\leq e^{-at^2}$ et $\int^{+\infty}_{-\infty}e^{-at^2}dt$ converge.
\item $f$ admet une dérivé partielle, $\frac{\partial f}{\partial x} (x,t)=-ite^{-itx}e^{-at^2}$ vérifiant les hypothèses de théorème continuité sous le signe $\int$, soit :
\begin{enumerate}
\item \textit{Continuité :}\\ 
Soit $x\in \R$ fixé. La fonction $t\mapsto -ite^{-itx}e^{-at^2}$ est continue  sur $\R$.\\
Soit $t\in \R$ fixé. La fonction $x\mapsto -ite^{-itx}e^{-at^2}$ est continue  sur  $\R$.
\item  \textit{Domination :}\\
On a :
$$\forall x\in \R, \forall t\in \R: \quad  \left|-ite^{-itx}e^{-at^2}\right|\leq |t|e^{-at^2}.$$
$t\mapsto |t|e^{-at^2}$ est intégrable sur $\R$.\\
\end{enumerate}
\end{enumerate} 
Les hypothèses de théorème de dérivation sous le signe $\int$ sont vérifiées donc F est de classe $\mathcal{C}^1$ sur $\R$ et vérifie,
$$F'(x)=\int^{+\infty}_{-\infty} \frac{\partial f}{\partial x} (x,t) \mathrm dt =\int^{+\infty}_{-\infty} -ite^{-itx}e^{-at^2}\mathrm dt.$$
Une intégration par parties permet de se ramener à 
$$F'(x)=\left[\frac{i}{2a}e^{-itx}e^{-at^2} \right]_{-\infty}^{+\infty}- \int^{+\infty}_{-\infty}\frac{i}{2a}e^{-at^2} (-ix)e^{-itx}\, \mathrm dt.$$
Soit
$$F'(x)=\frac{-x}{2a}F(x).$$
On résout l'équation différentielle vérifiée par F : $F(x)=F(0)e^{-x^2/4a}$. De plus,
$F(0)=\int^{+\infty}_{-\infty}e^{-at^2}dt=\int^{+\infty}_{-\infty}e^{-u^2} \frac{\mathrm du}{\sqrt{a}}= \sqrt{\frac \pi a}$.\\
Finalement $F(x)= \sqrt{\frac \pi a} e^{-x^2/4a}$.
\end{Exemple}

\begin{Theoreme}[continuité sous le signe $\int$ version locale]
On peut remplacer l'hypothèse de domination par:
\begin{itemize}
\item
  \impo{Hypothèse de domination locale:}
  Pour tout segment $K$ inclus dans $I$,
  il existe une fonction $\phi_K \colon J \to \R$ telle que

  \begin{itemize}
  \item
    pour tout $(x,t)\in K\times J$, on a $\left|\frac{\partial f}{\partial x} (x,t)\right|\leq\phi _K(t)$,
  \item
    $\phi _K$ est continue par morceaux et intégrable sur $J$.
  \end{itemize}
\end{itemize}
\end{Theoreme}

\begin{Theoreme}[Extension aux fonctions de classe $\mathcal{C}^p$]
On suppose:
\begin{enumerate}
\item
  $f$ est de classe $\mathcal{C}^p$ par rapport à $x$, c.-à-d.
  \begin{itemize}
  \item
    pour tout $t\in J$ fixé, la fonction $x \mapsto f(x,t)$ est de classe $\mathcal{C}^p$ sur $I$.
  \end{itemize}
\item
  pour tout $0\leq k<p$,
 $\frac{\partial^k f}{\partial x^k}$ est continue par morceaux
 et intégrable par rapport à $t$, c.-à-d.

  \begin{itemize}
  \item
    pour tout $k\in \{0,\dots,p-1\}$ et pour tout $x\in I$ fixés,
    la fonction $t \mapsto \frac{\partial^k f}{\partial x^k} (x,t)$
    est continue par morceaux et intégrable sur $J$.
  \end{itemize}
\item
  $\frac{\partial^p f}{\partial x^p}$ est continue par morceaux par rapport à $t$, c.-à-d.
  \begin{itemize}
  \item
    pour tout $x\in I$ fixé, la fonction $t \mapsto \frac{\partial^p f}{\partial x^p} (x,t)$
    est continue par morceaux sur $J$.
  \end{itemize}
\item
  \impo{Hypothèse de domination:}
  Il existe une fonction $\phi$ telle que
  \begin{itemize}
  \item
    pour tout $(x,t)\in I×J$, on a $\left| \frac{\partial^p f}{\partial x^p} (x,t) \right| \leq \phi (t)$,
  \item
    $\phi $ est continue par morceaux et intégrable sur $J$.
  \end{itemize}
\end{enumerate}
Alors:
\begin{itemize}
\item
  La fonction $F \colon x \mapsto \int_J f(x,t) \mathrm dt$ est définie et de classe $C^p$ sur $I$.
\item
  Pour tout $k\in \{0,\dots,p\}$, pour tout $x\in I$,
  on a \[ F^{(k)}(x) = \int_J \frac{\partial^k f}{\partial x^k}(x,t) \mathrm dt. \]
\end{itemize}
\end{Theoreme}
\end{document}
