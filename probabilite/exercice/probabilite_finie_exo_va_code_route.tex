\exo{ 0402 , tariel, 01/11/2017, 1-2, {}}[Code de la route!]
L'examen du code de la route se compose de 40 questions. Pour chaque question, on a le choix entre 4 r�ponses possibles. Une seule de ces r�ponses est correcte. Un candidat se pr�sente � l'examen. Il arrive qu?il connaisse la r�ponse
� certaines questions. Il r�pond alors � coup s�r. S?il ignore la r�ponse, il
choisit au hasard entre les 4 r�ponses propos�es. On suppose toutes les questions
ind�pendantes et que pour chacune de ces questions, la probabilit� que
le candidat connaisse la vraie r�ponse est $p$. On note, pour $1\leq i\leq 40$, $A_i$ l'�v�nement : "le candidat donne la bonne r�ponse � la $i$-�me 
question". On note $S$ la variable al�atoire �gale au nombre total de bonnes r�ponses.
\begin{enumerate}
\item Calculer $P(A_i)$.
\item Quelle est la loi de $S$ (justifier!)?
\item A quelle condition sur $p$ le candidat donnera en moyenne au moins 36 bonnes r�ponses?
\end{enumerate}
\finexo