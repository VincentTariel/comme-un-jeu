%Exercices probabilit�s finies
\exo{ 0402 , tariel, 01/11/2017, 1-2, {}}[Compagnie d'assurance]
Une compagnie d'assurance r�partit ses clients en trois classes $R_1$, $R_2$ et $R_3$ : les bons risques, les risques moyens, et les mauvais risques.
Les effectifs de ces trois classes repr�sentent $20\%$ de la population totale pour la classe $R_1$, $50\%$ pour la classe $R_2$, et 
$30\%$ pour la classe $R_3$. Les statistiques indiquent que les probabilit�s d'avoir un accident au cours de l'ann�e pour une personne de l'une de ces trois classes sont respectivement de 0.05, 0.15 et 0.30.
\begin{enumerate}
\item Quelle est la probabilit� qu'une personne choisie au hasard dans la population ait un accident dans l'ann�e?
\item Si M.Martin n'a pas eu d'accident cette ann�e, quelle est la probabilit� qu'il soit un bon risque?
\end{enumerate}
\finexo