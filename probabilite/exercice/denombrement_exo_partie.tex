\exo{ 0402 , tariel, 01/11/2017, 1-2, {}}[*, Nombre de parties d'un ensemble]
D�nombrer  l'ensemble des parties de $\{1,\dots,n\}$, not� $ \mathcal{P}(\{1,\dots,n\})$. 
\begin{correction}
D�montrons que cette fonction est bijective :
\[
f:
  \begin{array}{rcl}
    \mathcal{P}(\{1,\dots,n\}) & \longrightarrow & \mathcal{F}(\{1,\dots,n\},\{0,1\}) \\
      A & \longmapsto & \mathbf 1_A \\
  \end{array}
\]avec $\mathbf 1_A$ fonction indicatrice de l'ensemble $A$
\[
\mathbf 1_A :\begin{array}{rcl}  \{1,\dots,n\} & \longrightarrow & \{0,1\}  \\
i & \longmapsto & \left\{\begin{matrix}  1 \ \mbox{si} \ i \ \in \ A \\ 0 \ \mbox{si} \ i \ \notin \ A \end{matrix}\right. \end{array}
\]
\begin{itemize}
\item injective : soit $A,B\in \mathcal{P}(\{1,\dots,n\})$ tel que $f(A)=f(B)$, d'o� $\mathbf 1_A=\mathbf 1_B$ donc $A=B$.
\item surjectivit� : soit $\phi\in  \mathcal{F}(\{1,\dots,n\},\{0,1\})$. On pose $A=\{i\in \{1,\dots,n\}:\phi(i)=1\}$. on a bien $f(A)=\phi$.
\end{itemize}
Comme $f$ est bijective $\text{Card}(\mathcal{P}(\{1,\dots,n\}))=\text{Card}(\mathcal{F}(\{1,\dots,n\},\{0,1\}))=2^n$.
\end{correction}

\finexo