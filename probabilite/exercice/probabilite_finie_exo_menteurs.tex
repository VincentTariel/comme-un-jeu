%Exercices probabilit�s finies
\exo{ 0402 , tariel, 01/11/2017, 1-2, {}}[la cha�ne des menteurs]

On suppose qu'un message binaire ($0$ ou $1$) est transmis depuis un �metteur $M_0$ � travers une cha�ne $M_1, M_2, \dots, M_n$ de messagers menteurs, qui transmettent correctement le message avec une probabilit� $p$, mais qui changent sa valeur avec une probabilit� $1-p$.

Si l'on note $a_n$ la probabilit� que l'information transmise par $M_n$ soit identique � celle envoy�e par $M_0$ (avec comme convention que $a_0=1$), d�terminer $a_{n+1}$ en fonction de $a_n$, puis une expression explicite de $a_n$ en fonction de $n$, ainsi que la valeur limite de la suite $(a_n)_{n\in\N}$. Le r�sultat est-il conforme � ce � quoi l'on pouvait s'attendre?
\finexo