\exo{ 0402 , tariel, 01/11/2017, 1-2, {}}[*, d�mocratie]
On �lit les 2 d�l�gu�s au hasard parmi une classe de 22 �l�ves avec 14 gar�ons/8 filles et 15 responsables et 7 irresponsables
\begin{enumerate}
\item Combien y a-t-il de mani�res de choisir les 2 d�l�gu�s ? 
\item avec la parit� en plus ?
\item avec 2 personnes irresponsables ?
\end{enumerate}

\begin{correction}
\begin{enumerate}
\item  Une �lection correspond � un tirage sans remise et sans ordre. Donc le nombre de mani�res est $\begin{pmatrix}22\\2\end{pmatrix}$
\item parit� = " choisir 1 gar�on  (parmi 14) et choisir 1 fille (parmi 8) ". D'apr�s le principe de d�composition, on a $\begin{pmatrix}14\\1\end{pmatrix}\begin{pmatrix}8\\1\end{pmatrix}$.
\item irresponsables = " choisir 2 irresponsables  parmi 15. Soit $\begin{pmatrix}15\\2\end{pmatrix}$.
\end{enumerate}
\end{correction}

\finexo