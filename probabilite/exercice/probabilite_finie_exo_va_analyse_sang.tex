\exo{ 0402 , tariel, 01/11/2017, 1-2, {}}[Analyse de sang]
On cherche � d�pister une maladie d�tectable � l'aide d'un examen sanguin. On suppose que dans notre population, il y a une proportion $p$ de personnes qui n'ont pas cette maladie.
\begin{enumerate}
\item On analyse le sang de $r$ personnes de la population, avec $r$ entier au moins �gal � 2. On suppose que l'effectif de la population est suffisamment grand pour que le choix de ces $r$ personnes s'apparente � un tirage avec remise. Quelle est la probabilit� qu'aucune de ces personnes ne soit atteinte de la maladie?
\item On regroupe le sang de ces $r$ personnes, puis on proc�de � l'analyse de sang. Si l'analyse est n�gative, aucune de ces personnes n'est malade et on arr�te. Si l'analyse est positive, on fait toutes les analyses individuelles (on avait pris soin de conserver une partie du sang recueilli avant l'analyse group�e). On note $Y$ la variable al�atoire qui donne le nombre d'analyses de sang effectu�es. Donner la loi de probabilit� de $Y$ et calculer son esp�rance en fonction de $r$ et de $p$.
\item On s'int�resse � une population de $n$ personnes, et on effectue des analyses collectives apr�s avoir m�lang� les pr�l�vements par groupe de $r$ personnes, o� $r$ est un diviseur de $n$. Montrer que le nombre d'analyses que l'on peut esp�rer �conomiser, par rapport � la d�marche consistant � tester imm�diatement toutes les personnes, est �gal � $np^r-\frac nr$.
\item Dans cette question, on suppose que $p=0,9$ et on admet qu'il existe un r�el $a>1$ de sorte que la fonction $x\mapsto p^x-\frac{1}x$ est croissante sur $[1,a]$ et d�croissante sur $[a,+\infty[$. \'Ecrire un algorithme permettant de d�terminer 
pour quelle valeur de l'entier $r$ le nombre $p^r-\frac 1r$ est maximal.
\end{enumerate} 
\finexo