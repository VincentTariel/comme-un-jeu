\exo{ 0402 , tariel, 01/11/2017, 1-2, {}}[*, Anagramme]
D�nombrer les anagrammes des mots suivants : MATHS, RIRE.
   
\begin{correction}
Un anagramme du mot MATHS correspond � une permutation des lettres, soit  5! fa�ons.\\
Pour le second mot, voici deux r�ponses possibles : 
\begin{enumerate}
\item en utilisant le principe de d�composition, on place d'abord la lettre R, soit 2 lettres parmi 4, d'o� 4!/(2!2!) combinaisons, puis la lettre I (il reste 2 places), soit 2 combinaisons , et enfin la lettre E (il reste 1 place), soit 1 combinaison. En multipliant ces combinaisons, on obtient  4!/2! combinaisons.
\item R est pr�sent deux fois et si on permute cette lettre, on trouve le m�me mot.
On doit donc diviser le nombre total de permutations par le nombres de permutations entre lettres identiques, d'o�  4!/2! combinaisons.
\end{enumerate}
\end{correction}

\finexo