\exo{ 0402 , tariel, 01/11/2017, 1-2, {}}[*, derivation]
On consid�re la s�rie enti�re de la variable r�elle   $\sum _{n\geq 3}\frac {x^{n}}{(n+1)(n-2)}$
\begin{enumerate}
\item D�terminer le rayon de convergence $ R$ de cette s�rie enti�re. Est-elle convergente pour $|x|=R$ ?
\item Pour tout nombre r�el $x$ tel que la s�rie enti�re pr�c�dente converge, on note $S(x)$ sa somme.\\
Expliciter la d�riv�e de la fonction $x\mapsto xS(x)$  sur $ ]-R,R[$.\\
En d�duire $ S(x)$ pour $x$ appartenant � $]-R,R[$.\\
\item Calculer la somme de chacune des s�ries num�riques suivantes :
$$ \sum _{n\geq 3}(-1)^{n}{\frac {R^{n}}{(n+1)(n-2)}},$$
$$ \sum _{n\geq 3}{\frac {R^{n}}{(n+1)(n-2)}}.$$ 
\end{enumerate}
\begin{correction}
\begin{enumerate}
\item $R=1$. En effet, $\frac {|x|^{n}}{(n+1)(n-2)}\sim {\frac {|x|^{n}}{n^{2}}}$. Donc si $|x|\leq 1$, la s�rie est absolument convergente (par comparaison avec la s�rie de Riemann convergente $\sum _{n\geq 1}{\frac {1}{n^{2}}}$) tandis que si $|x|>1$ , $\frac {|x|^{n}}{n^{2}}\to +\infty $ et la s�rie diverge grossi�rement.
\item On peut naturellement d�river la fonction sur son ouvert de convergence, soit ici $ ]-R,R[$.
$$xS(x)=\sum _{n=3}^{\infty }{\frac {x^{n+1}}{(n+1)(n-2)}}.$$
On a donc $(xS)'(x)=\sum _{n=3}^{\infty }{\frac {x^{n}}{n-2}}=x^{2}\sum _{k\geq 1}^{\infty }{\frac {x^{k}}{k}}=-x^{2}\ln(1-x)$.
Une int�gration par parties, suivie d'une int�gration de fraction rationnelle, permet d'en d�duire 
$xS(x)$, puis
$$S(x)={\frac {1}{3}}\left({\frac {1-x^{3}}{x}}\ln(1-x)+1+{\frac {x}{2}}+{\frac {x^{2}}{3}}\right).$$
(Une autre m�thode aboutissant � ce r�sultat est d'�crire :
$$3S(x)=\sum _{n=3}^{\infty }\left({\frac {x^{n}}{n-2}}-{\frac {x^{n}}{n+1}}\right)={\frac {x^{3}}{1}}+{\frac {x^{4}}{2}}+{\frac {x^{5}}{3}}+{\frac {x^{3}-1}{x}}\sum _{n\geq 4}{\frac {x^{n}}{n}}=x^{3}+{\frac {x^{4}}{2}}+{\frac {x^{5}}{3}}+{\frac {1-x^{3}}{x}}\left(\ln(1-x)+x+{\frac {x^{2}}{2}}+{\frac {x^{3}}{3}}\right).$$
\item Par continuit�, $S(-1)={\frac {1}{3}}\left({\frac {2}{-1}}\ln 2+1-{\frac {1}{2}}+{\frac {(-1)^{2}}{3}}\right)={\frac {5}{18}}-{\frac {2}{3}}\ln 2$ et $S(1)={\frac {1}{3}}\left(0+1+{\frac {1}{2}}+{\frac {1^{2}}{3}}\right)={\frac {11}{18}}.$
\end{enumerate}
\end{correction}
\finexo