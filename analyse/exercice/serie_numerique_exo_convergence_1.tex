\exo{ 0402 , tariel, 01/11/2017, 1-2, {}}[*, Nature]
�tudier la convergence des s�ries $\sum u_n$ suivantes :
\begin{multicols}{3}
    \begin{enumerate}
        \item $u_n =  \frac{n+1}{3^n}$ 
        \item $u_n=\frac{n}{n^3+1}$
        \item $u_n=\frac{\sqrt n}{n^2+\sqrt n}$
        \item $u_n=n\sin(1/n)$
        \item $u_n=\frac{1}{\sqrt{n}}\ln\left(1+\frac{1}{\sqrt{n}}\right)$
        \item $u_n=\frac{\sqrt {n+1}-\sqrt{n}}{n}$
        \item $u_n=\frac{(-1)^n +n}{n^2+1}$
        \item $u_n=\ln\left(\frac{n^2+n+1}{n^2+n-1}\right)$
    \end{enumerate}
    \end{multicols}  



\begin{correction}
    \begin{enumerate}
    \item on a : $$\frac{u_{n+1}}{u_n} =\frac{n+1}{3n}\tend[n\to\infty]\frac{1}{3}$$ D'apr�s le crit�re de d'Alembert, la s�rie est convergente.
        \item $\sum \frac{n}{n^3+1}$ est une s�rie � termes positifs. On a :
$$\frac{n}{n^3+1}\Sim_{n\to\infty}\frac{1}{n^2}.$$
Comme la s�rie de Riemann $\sum \frac{1}{n^2}$ est convergente, la s�rie $\sum \frac{n}{n^3+1}$ est convergente par r�gle de comparaison.
        \item $\sum \frac{\sqrt n}{n^2+\sqrt n}$ est une s�rie � termes positifs. On a :
$$\frac{\sqrt n}{n^2+\sqrt n} =\frac{\sqrt n}{n^2(1 +\sqrt n /n^2)}  \Sim_{n\to\infty}\frac{1}{n^{3/2}}.$$
Comme la s�rie de Riemann $\sum \frac{1}{n^{3/2}}$ est convergente, la s�rie $\sum \frac{n}{n^3+1}$ est convergente par r�gle d'�quivalence.
\item Comme  $ n\sin(1/n)\Sim_{n\to\infty} 1$, le terme g�n�rale de s�rie ne converge pas vers 0, donc la s�rie diverge grossi�rement.
\item $\sum \frac{1}{\sqrt{n}}\ln\left(1+\frac{1}{\sqrt{n}}\right)$ est une s�rie � termes positifs. On a :
$$\frac{1}{\sqrt{n}}\ln\left(1+\frac{1}{\sqrt{n}}\right)   \Sim_{n\to\infty}\frac{1}{n}.$$ 
Comme la s�rie de Riemann $\sum \frac{1}{n}$ est divergente, la s�rie $\sum \frac{1}{\sqrt{n}}\ln\left(1+\frac{1}{\sqrt{n}}\right)$  est divergente par r�gle d'�quivalence.
\item $\sum \frac{\sqrt n}{n^2+\sqrt n}$ est une s�rie � termes positifs. On a :
$$\frac{\sqrt {n+1}-\sqrt{n}}{n} =\frac{\sqrt{n}(\sqrt {1+1/n}-1)}{n}  \Sim_{n\to\infty}\frac{\sqrt{n}(\frac{1}{2n})}{n}\Sim_{n\to\infty}\frac{1}{2n^{3/2}}   .$$
Comme la s�rie de Riemann $\sum \frac{1}{n^{3/2}}$ est convergente, la s�rie $\sum \frac{n}{n^3+1}$ est convergente par r�gle d'�quivalence.
 \item $\sum \frac{(-1)^n +n}{n^2+1}$ est une s�rie � termes positifs. On a :
$$\frac{(-1)^n +n}{n^2+1}   \Sim_{n\to\infty}\frac{1}{n}.$$ 
Comme la s�rie de Riemann $\sum \frac{1}{n}$ est divergente, la s�rie $\sum \frac{(-1)^n +n}{n^2+1}$  est divergente par r�gle d'�quivalence.
 \item  
 
 
 
  $\sum \ln\left(\frac{n^2+n+1}{n^2+n-1}\right)$ est une s�rie � termes positifs. 
\begin{itemize}
\item Version courte :
$$u_n=\ln\left(1+\frac{1}{n}+\frac{1}{n^2}\right)-\ln\left(1+\frac{1}{n}-\frac{1}{n^2}\right)\underset{n\rightarrow+\infty}{=}\left(\frac{1}{n}+O\left(\frac{1}{n^2}\right)\right)-\left(\frac{1}{n}+O\left(\frac{1}{n^2}\right)\right)=O\left(\frac{1}{n^2}\right).$$
\item Version longue :
$$u_n=\ln\left(1+\frac{1}{n}+\frac{1}{n^2}\right)-\ln\left(1+\frac{1}{n}-\frac{1}{n^2}\right)$$
$$u_n=\left(\frac{1}{n}+\frac{1}{n^2}\right)-\frac{\left(\frac{1}{n}+\frac{1}{n^2}\right)^2}{2}-\left(\left(\frac{1}{n}-\frac{1}{n^2}\right)-\frac{\left(\frac{1}{n}-\frac{1}{n^2 }\right)^2}{2}\right)+ o\left(\frac{1}{n^2}\right)$$
$$u_n=\left(\frac{1}{n}+\frac{1}{n^2}\right)-\frac{\left(\frac{1}{n}+\frac{1}{n^2}\right)^2}{2}-\left(\left(\frac{1}{n}-\frac{1}{n^2}\right)-\frac{\left(\frac{1}{n}-\frac{1}{n^2 }\right)^2}{2}\right)+ o\left(\frac{1}{n^2}\right)$$
$$ u_n= \frac{2}{n^2} +o\left(\frac{1}{n^2}\right) \Sim_{n\to\infty}\frac{2}{n^2}$$
\end{itemize}  
  
Comme la s�rie de Riemann $\sum \frac{1}{n^2}$ est convergente, la s�rie $\sum \ln\left(\frac{n^2+n+1}{n^2+n-1}\right)$ converge par r�gle de comparaison.
    \end{enumerate}
\end{correction}
\finexo