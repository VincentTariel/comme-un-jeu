\exo{ 0402 , tariel, 01/11/2017, 1-2, {}}[*, Nature]
�tudier la convergence des s�ries $\sum u_n$ suivantes :\\
    \begin{enumerate}
        \item  $u_n=\frac{1}{n+(-1)^n\sqrt{n}}$
        \item $u_n=\left(\frac{n+3}{2n+1}\right)^{\ln n}$
        \item  $u_n=\frac{1}{\ln(n)\ln(\ch n)}$
        \item  $u_n=e-\left(1+\frac{1}{n}\right)^n$
      \end{enumerate}      
       



\begin{correction}
    \begin{enumerate}
        \item $\sum \frac{1}{n+(-1)^n\sqrt{n}}$ est une s�rie � termes positifs. On a :
$$\frac{1}{n+(-1)^n\sqrt{n}} = \frac{1}{n(1+(-1)^n\sqrt{n}/n)}\Sim_{n\to\infty}\frac{1}{n}.$$
Comme la s�rie de Riemann $\sum \frac{1}{n}$ est divergente, la s�rie $\sum \frac{1}{n+(-1)^n\sqrt{n}}$ est divergente par r�gle de comparaison.
        \item $\sum \left(\frac{n+3}{2n+1}\right)^{\ln n}$ est une s�rie � termes positifs. On a :
$$\left(\frac{n+3}{2n+1}\right)^{\ln n} =e^{\ln(n)\ln\left(\frac{n+3}{2n+1}\right) } \Sim_{n\to\infty}e^{\ln(n)\ln\left(\frac{1}{2}\right) }\Sim_{n\to\infty}\frac{1}{n^{\ln 2}} .$$
Comme la s�rie de Riemann $\sum \frac{1}{n^{\ln 2}}$ est divergente, la s�rie $\sum \left(\frac{n+3}{2n+1}\right)^{\ln n}$ est divergente par r�gle de comparaison.
\item $\sum\frac{1}{\ln(n)\ln(\ch n)}$ est une s�rie � termes positifs. On a :
$$\ln(\ch(n)) = \ln\left(  \frac{e^n+e^{-n}}{2}\right)\Sim_{n\to\infty}\ln(e^n/2  )\Sim_{n\to\infty}n-2\Sim_{n\to\infty}n.$$
�tudions la nature de la s�rie $\sum \frac{1}{\ln(n)n}$ � l'aide d'une comparaison s�rie int�grale.\\
La fonction $x\rightarrow x\ln x$ est continue, croissante et strictement positive sur $]1,+\infty[$ (produit de deux fonctions strictement positives et croissantes sur $]1,+\infty[$). Par suite, la fonction $x\rightarrow\frac{1}{x\ln x}$ est continue, d�croissante sur $]1,+\infty[$ et de limite 0 en l'infini. Les hypoth�ses �tant v�rifi�e, la s�rie $\sum \frac{1}{\ln(n)n}$ est de m�me nature que la suite $\left(\int_{2}^n \frac{1}{x\ln x}dx\right)=\left(\ln(\ln(n))-2\right)$. Cette suite diverge donc la s�rie aussi.\\
Enfin par th�or�me de comparaison, la s�rie $\sum\frac{1}{\ln(n)\ln(\ch n)}$ diverge.
\item On a :
$$ e-\left(1+\frac{1}{n}\right)^n=e-e^{n\ln(1+\frac{1}{n})}=e-e^{n(\frac{1}{n}-\frac{1}{2n^2}+O(\frac{1}{n^2}))}=e(1-e^{-\frac{1}{2n}+O(\frac{1}{n})})\Sim_{n\to\infty}\frac{e}{2n}  $$
Comme la s�rie de Riemann $\sum \frac{1}{n}$ est divergente, la s�rie $\sum e-\left(1+\frac{1}{n}\right)^n$ est divergente par r�gle de comparaison.
    \end{enumerate}
\end{correction}
\finexo