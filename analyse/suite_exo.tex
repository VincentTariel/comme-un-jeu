
\documentclass[12pt,a4paper]{article}
\usepackage{amsfonts,amsmath,amssymb,graphicx}
\usepackage[francais]{babel}
\usepackage[latin1]{inputenc}
\usepackage{answers}
\usepackage{a4wide,jlq2eams} 
\usepackage{multicol}
%---- Dimensions des marges ---
\setlength{\paperwidth}{21cm} 
\setlength{\paperheight}{29.7cm}
\setlength{\evensidemargin}{0cm}
\setlength{\oddsidemargin}{0cm} 
\setlength{\topmargin}{-2.5cm}
\setlength{\headsep}{0.7cm} 
\setlength{\headheight}{1cm}
\setlength{\textheight}{25cm} 
\setlength{\textwidth}{17cm}




%---- Structure Exercice -----
\newtheorem{Exc}{Exercice}
\Newassociation{correction}{Soln}{mycor}
\Newassociation{indication}{Indi}{myind}
%\newcommand{\precorrection}{~{\bf \footnotesize [Exercice corrig\'e]}}
%\newcommand{\preindication}{~{\bf \footnotesize [Indication]}}
\renewcommand{\Solnlabel}[1]{\bf \emph{Correction #1}}
\renewcommand{\Indilabel}[1]{\bf \emph{Indication #1}}

\def\exo#1{\futurelet\testchar\MaybeOptArgmyexoo}
\def\MaybeOptArgmyexoo{\ifx[\testchar \let\next\OptArgmyexoo
                        \else \let\next\NoOptArgmyexoo \fi \next}
\def\OptArgmyexoo[#1]{\begin{Exc}[#1]\normalfont}
\def\NoOptArgmyexoo{\begin{Exc}\normalfont}

\newcommand{\finexo}{\end{Exc}}
\newcommand{\flag}[1]{}

\newtheorem{question}{Question}



%---- Style de l'entete -----                     % -> � personnaliser <-
\setlength{\parindent}{0cm}
\newcommand{\entete}[1]
{
{\noindent
  \textsf{Feuille d'exercices}    \hfill   \textsf{ISEN}          % Universit�
}
\hrule
\hrule
\begin{center} 
\textbf{\textsf{\Large 
     Suite num�rique     % TITRE
}}
\end{center}
\hrule
}


\begin{document}

%--- Gestion des corrections ---
 \Opensolutionfile{mycor}[ficcorex]
 \Opensolutionfile{myind}[ficind]
 \entete{\'Enonc�s}

\input{exercice/suite_exo_borne}
\exo{ 0402 , tariel, 01/11/2017, 1-2, {}}[*, Nature]
\'Etudier la nature des suites suivantes, et d�terminer leur limite �ventuelle :
$$\begin{array}{lcl}
\displaystyle \mathbf 1.\ u_n=\frac{\sin(n)+3\cos\left(n^2\right)}{\sqrt{n}}&&\displaystyle \mathbf 2.\ u_n=\frac{2n+(-1)^n}{5n+(-1)^{n+1}}\\
\displaystyle \mathbf 3.\ u_n=\frac{n^3+5n}{4n^2+\sin(n)+\ln(n)}&&\displaystyle \mathbf 4.\ u_n=
\sqrt{2n+1}-\sqrt{2n-1}\\
\displaystyle \mathbf 5.\ u_n=3^ne^{-3n}.
\end{array}$$
\finexo
\exo{ 0402 , tariel, 01/11/2017, 1-2, {}}[*,Somme t�lescopique]
\begin{enumerate}
\item D�terminer deux r�els $a$ et $b$ tels que
$$\frac{1}{k^2-1}=\frac{a}{k-1}+\frac{b}{k+1}.$$
\item En d�duire la limite de la suite 
$$u_n=\sum_{k=2}^n \frac{1}{k^2-1}.$$
\item Sur le m�me mod�le, d�terminer la limite de la suite 
$$v_n=\sum_{k=0}^n\frac{1}{k^2+3k+2}.$$
\end{enumerate}
\finexo

\exo{ 0402 , tariel, 01/11/2017, 1-2, {}}[*, Exemple de suites adjacentes]
D�montrer que les suites $(u_n)$ et $(v_n)$ donn�es ci-dessous forment
des couples de suites adjacentes.
$$
\begin{array}{ll}
\mathbf{1.}\quad \displaystyle u_n=\sum_{k=1}^n \frac1{k^2}\textrm{ et }v_n=u_n+\frac 1n\\
\mathbf{2.}\quad \displaystyle u_n=\sum_{k=1}^n\frac{1}{k+n}\textrm{ et }v_n=\sum_{k=n}^{2n}\frac 1k.
\end{array}$$
\finexo

\exo{ 0402 , tariel, 01/11/2017, 1-2, {}}[*, Avec des quantificateurs]
Soit $(u_n)$ une suite de nombres r�els. \'Ecrire avec des quantificateurs les propositions suivantes : 
\begin{enumerate}
\item $(u_n)$ est born�e.
\item $(u_n)$ n'est pas croissante.
\item $(u_n)$ n'est pas monotone.
\item $(u_n)$ n'est pas major�e.
\item $(u_n)$ ne tend pas vers $+\infty$.
\end{enumerate}
\finexo

\exo{ 0402 , tariel, 01/11/2017, 1-2, {}}[*, Moyenne de Ces\`aro]
Soit $(u_n)_{n\geq 1}$ une suite r�elle. On pose $S_n=\frac{u_1+\dots+u_n}{n}$.
\begin{enumerate}
\item On suppose que $(u_n)$ converge vers 0. Soient $\varepsilon>0$ et $n_0\in\mathbb N$ tel que, pour
$n\geq n_0$, on a $|u_n|\leq\varepsilon$.
\begin{enumerate}
\item Montrer qu'il existe une constante $M$ telle que, pour $n\geq n_0$, on a 
$$|S_n|\leq \frac{M(n_0-1)}{n}+\varepsilon.$$
\item En d�duire que $(S_n)$ converge vers 0.
\end{enumerate}
\item On suppose que $u_n=(-1)^n$. Que dire de $(S_n)$? Qu'en d�duisez-vous?
\item On suppose que $(u_n)$ converge vers $l$. Montrer que $(S_n)$ converge vers $l$.
\item On suppose que $(u_n)$ tend vers $+\infty$. Montrer que $(S_n)$ tend vers $+\infty$.
\end{enumerate}
\item Trouver un exemple de suite qui diverge mais dont la moyenne de Ces�ro converge.
\finexo
% Exercice 98

\exo{ 0402 , tariel, 01/11/2017, 1-2, {}}[*, Convergence des suites extraites]
Soit $(u_n)$ une suite de nombres r�els.
\begin{enumerate}
\item On suppose que $(u_{2n})$ et $(u_{2n+1})$ convergent vers la m�me limite. Prouver que $(u_n)$ est convergente.
\item Donner un exemple de suite telle que $(u_{2n})$ converge, $(u_{2n+1})$ converge, mais $(u_{n})$ n'est pas convergente.
\item On suppose que les suites $(u_{2n})$, $(u_{2n+1})$ et $(u_{3n})$ sont convergentes. Prouver que $(u_n)$ est convergente.
\end{enumerate}
\finexo


\exo{ 0402 , tariel, 01/11/2017, 1-2, {}}[*, Suite H�ron]
Etudier la suite :
  $$u_0 > \sqrt{2} , u_{n+1} = \frac12\left(u_n + \frac{2}{u_n}\right)$$
\finexo

%\exo{ 0402 , tariel, 01/11/2017, 1-2, {}}[*,Questions de cours]
%Soit $(u_n)$ une suite de nombre r�els croissante.
%\begin{enumerate}
%\item On suppose que $(u_n)$ converge vers $l$. D�montrer que pour tout entier $n$, on a $u_n\leq l$.
%\item On suppose que $(u_n)$ n'est pas major�e. D�montrer que $(u_n)$ tend vers $+\infty$.
%\end{enumerate}
%\finexo
%--- Fin des exercices ---
%------------------------------------------------------
%------------------------------------------------------

%--- Gestion des corrections ---
% \ \newpage
% \setcounter{page}{1}
% \entete{Indications}
%
% \Closesolutionfile{myind}
% \Readsolutionfile{myind}

 \ \newpage
 \setcounter{page}{1}
 \entete{Corrections}

 \Closesolutionfile{mycor}
 \Readsolutionfile{mycor}

\end{document}
\endinput




