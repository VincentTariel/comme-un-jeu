\documentclass[a4paper]{book}
\usepackage{t1enc}
\usepackage[latin1]{inputenc}
\usepackage[french]{minitoc}
 \usepackage{amsmath}
\usepackage{fancyhdr,amsmath,amsthm,amssymb,fancybox}
\usepackage[francais]{babel}
\usepackage{amsmath}
\usepackage{tikz}
\usetikzlibrary{shapes,backgrounds}
\usepackage{tkz-fct}   
\usepackage{a4wide,jlq2eams} 
\usepackage{graphicx}
\usepackage{hyperref}
\usepackage{slashbox}
\usepackage{thmbox}
\usepackage{xcolor}
\usepackage{sectsty}
\usepackage{longtable} 
\usepackage{multicol}
\usepackage{caption}
\usepackage{pgfplots}
\definecolor{amaranth}{rgb}{0.9, 0.17, 0.31}
\sectionfont{\color{magenta}}
\subsectionfont{\color{red}}
\subsubsectionfont{\color{red}}
\newcommand{\defi}[1]{\textbf{\textcolor{orange}{#1}}}

\setlength{\shadowsize}{1.5pt}
 
\pagestyle{fancy}
\addtolength{\headwidth}{\marginparsep}
\addtolength{\headwidth}{\marginparwidth} 
\renewcommand{\chaptermark}[1]{\markboth{#1}{}}
\renewcommand{\sectionmark}[1]{\markright{\thesection\ #1}}
\fancyhf{}
\fancyhead[LE,RO]{\bfseries\thepage}
\fancyhead[LO]{\bfseries\rightmark}
\fancyhead[RE]{\bfseries\leftmark}
\fancypagestyle{plain}{%
   \fancyhead{} % get rid of headers
   \renewcommand{\headrulewidth}{0pt} % and the line
}

\setcounter{minitocdepth}{3}


\renewcommand{\thesection}{\Roman{section}} 
\renewcommand{\thesubsection}{\Alph{subsection}}
\thmboxoptions{S,bodystyle=\itshape\noindent}
\newtheorem[L]{Lem}{Lemme}[section]
\newtheorem[L]{Th}[Lem]{Th�or�me}
\newtheorem[L]{Cor}[Lem]{Corollaire}
\newtheorem[L]{Prop}[Lem]{Proposition}

\newtheorem[S,bodystyle=\upshape\noindent]{Df}{D�finition}
\newtheorem[S,bodystyle=\upshape\noindent]{Ex}{Exemple}
\newtheorem[S,bodystyle=\upshape\noindent]{NB}{Remarque}
\newtheorem[S,bodystyle=\upshape\noindent]{Methode}{M�thode}
\newtheorem[S,bodystyle=\upshape\noindent]{intr}{Introduction}



\newcommand\SUI{(u_n)_{n\in\N}}
\newcommand\SER{ \sum u_n}
\newcommand\CC[1]{C^{#1}}
\newcommand{\sumni}{\sum_{n=0}^{+\infty}}
\newcommand\Sanxn{ \sum_n a_n x^n}
\newcommand\Sanzn{ \sum_n a_n z^n}
\newcommand\Sbnzn{ \sum_n b_n z^n}
\newcommand\Scnzn{ \sum_n c_n z^n}
\def\abs#1{\mathopen|{#1}\mathclose|}
\def\Abs#1{\left|{#1}\right|}
\newcommand{\Rpinf}{\R \cup \{+\infty\}}
\newcommand\DS{\displaystyle}
\newcommand\To[1]{\xrightarrow[#1]{}}
\newcommand\Toninf{\To{\ninf}}
%%%%%%%%%%%%%%%%%%%%%%%%%%%%





\newcommand{\FIK}{\mathcal{F}(I,\K  )}
\newcommand{\fn}{(f_n)_{n\in \N   }}
\newcommand{\Sfn}{\sum _n f_n}
\newcommand{\TTT}{]a,b]}
\newcommand{\TTTT}{]a,b[}
\begin{document}



\chapter{Int�grales g�n�ralis�es}
Ce chapitre a pour objectif d'�tendre la notion d'int�grale sur un intervalle qui n'est pas n�cessairement un
segment, par exemple, $$\int_{1 }^{+\infty } \frac{1}{x^2} \,\mathrm dx = 1.$$
\begin{center}
\begin{tikzpicture}[scale=0.8]
\begin{axis}[
	   axis y line = middle,
       axis x line = middle,
       samples     = 200,
       domain      = 0:1,
       xmin = 0, xmax = 18,
       ymin = 0, ymax = 1,
]
\addplot[domain=1:18,fill=blue!20] {1/x^2} node[pos=0.3,above] {$x\mapsto 1/x^2$}\closedcycle ;
\end{axis}
\end{tikzpicture}
\end{center}
\section{G�n�ralit�s}
L'int�grale impropre d�signe l'int�grale d'une fonction sur un intervalle. Elle est d�finie comme une extension de l'int�grale usuelle sur un segment  par un passage � la limite sur les bornes d'int�gration de l'int�grale.   

\subsection{D�finition}
\begin{Df}[Int�grale impropre sur $[a,b[$]
Soit $f$ une fonction continue par morceaux sur $[a,b [$ avec $b \in \R$ ou $b = +\infty$.\\
Si $\int_a^t f(x) \,\mathrm dx$ admet une limite finie lorsque $t$ tend vers $b$, on dit que l'\defi{int�grale impropre} converge et on
note $\int_a^b f(x) \,\mathrm dx$ cette limite. Sinon, on dit qu'elle diverge.
\end{Df}

\end{document}

