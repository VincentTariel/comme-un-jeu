\documentclass{yann}
\usepackage[most]{tcolorbox}

\newcommand{\FIK}{\mathcal{F}(I,𝕂)}
\newcommand{\fn}{(f_n)_{n∈ℕ}}
\newcommand{\Sfn}{∑_n f_n}

\begin{document}

\title{Suites et séries de fonctions : feuille d'exercices}
\maketitle
\Exercice[Domaine de définition]
Soit $\zeta(x)=\sum_{n=1}^\infty \frac{1}{n^x}$ et $\mu(x)=\sum_{n=1}^\infty \frac{(-1)^n}{n^x}$.\\
Déterminer le domaine de définition de la fonction $\zeta$, appelé fonction de Rienman et de la fonction $\mu$. 
% voir h prepa analalyse dans le cours



\Exercice[Convergence simple]
Soit $(f_n), (g_n)$ et $h_n$, les suites de fonctions définies par   $\forall n \in \mathbb{N}^* : f_n: x\mapsto x^n,\quad g_n: x\mapsto (1+\frac x n)^n,\quad h_n:x\mapsto x n^a e^{-nx}$.
Démontrer que ces fonctions converge simplement sur l'intervalle $[0,1]$ et déterminer la limite. 

\Exercice[Convergence uniforme]
Soit la suite de fonctions $(f_n)$ définie par : 
$$\Fonction{f_n}{\mathbb{R}^+}{\mathbb{R}}{x}{x^2e^{-nx}}, \quad \forall n \in \mathbb{N}.$$
Étudier la convergence de cette suite de fonctions. 
% voir h prepa analalyse dans le cours
\Exercice[Vrai/Faux]
Soit $(f_n)$ une suite de fonctions qui converge simplement vers une fonction $f$ sur un intervalle $I$. Dire si les assertions suivantes sont vraies ou fausses :
\begin{enumerate}
\item Si les $f_n$ sont croissantes, alors $f$ aussi.
\item Si les $f_n$ sont strictement croissantes, alors $f$ aussi.
\item Si les $f_n$ sont périodiques de période $T$, alors $f$ aussi.
\item Si les $f_n$ sont continues en $a$, alors $f$ aussi.
\end{enumerate}
Reprendre l'exercice en remplaçant la convergence simple par la convergence uniforme.


% Exercice 1775
\Exercice[\'Etude de convergence simple et uniforme détaillée]
Pour $x\in\mathbb R$, on pose $f_n(x)=1+x+\dots+x^{n-1}$.
\begin{enumerate}
\item \'Etudier la convergence simple de la suite de fonctions $(f_n)$. On note $f(x)$ la limite de la suite $(f_n(x))$ lorsque cette limite existe.
\item On pose, pour $x\in ]-1,1[$, $\varphi_n(x)=f(x)-f_n(x)$. Vérifier que 
$$\varphi_n(x)=\frac{x^n}{1-x}.$$
\item Quelle est la limite de $\varphi_n$ en $1$? En déduire que la convergence n'est pas uniforme sur $]-1,1[$. 
\item Soit $a\in ]0,1[$. Démontrer que $(f_n)$ converge uniformément vers $f$ sur $[-a,a]$.
\end{enumerate}


% Exercice 1776
\Exercice[Convergence uniforme sur un intervalle plus petit...]
On pose, pour $n\geq 1$ et $x\in ]0,1]$, $f_n(x)=nx^n\ln(x)$ et $f_n(0)=0$.
\begin{enumerate}
\item Démontrer que $(f_n)$ converge simplement sur $[0,1]$ vers une fonction $f$ que l'on précisera. On note ensuite $g=f-f_n$.
\item \'Etudier les variations de $g$. 
\item En déduire que la convergence de $(f_n)$ vers $f$ n'est pas uniforme sur $[0,1]$.
\item Soit $a\in ]0,1]$. En remarquant qu'il existe $n_0\in\mathbb N$ tel que $e^{-1/n}\geq a$ pour tout $n\geq n_0$, démontrer que la suite $(f_n)$ converge uniformément vers $f$ sur $[0,a]$.
\end{enumerate}





\Exercice[Avec paramètre]
Soit $a\geq 0$. On définit la suite de fonctions $(f_n)$ sur $[0,1]$ par $f_n(x)=n^a x^n(1-x)$.
Montrer que la suite $(f_n)$ converge simplement vers 0 sur $[0,1]$, mais que la convergence est uniforme si et seulement si
$a<1.$


\Exercice[Convergence uniforme et fonctions bornées]
Soit $(f_n)$ une suite de fonctions \emph{bornées}, $f_n:\mathbb R\to\mathbb R$.
On suppose que la suite $(f_n)$ converge uniformément vers $f$. Montrer que $f$ est
bornée. Le résultat persiste-t-il si on suppose uniquement la convergence simple?


\Exercice[Convergence simple et fonctions décroissantes]
Soit $(f_n)$ une suite de fonctions décroissantes définies sur $[0,1]$ telle que $(f_n)$ converge simplement vers la fonction nulle. Montrer que la convergence est en fait uniforme.




\Exercice[Exemples et contre-exemples]
Pour $x\geq 0$, on pose $u_n(x)=\frac{x}{n^2+x^2}.$
\begin{enumerate}
\item Montrer que la série $\sum_{n=1}^{+\infty}u_n$ converge simplement sur $\mathbb R_+$.
\item  Montrer que la série $\sum_{n=1}^{+\infty}u_n$ converge uniformémement sur tout intervalle $[0,A]$,
avec $A>0$.
\item Vérifier que, pour tout $n\in\mathbb N$, $\sum_{k=n+1}^{2n}\frac{n}{n^2+k^2}\geq\frac 15$.
\item En déduire que la série $\sum_{n\geq 1}u_n$ ne converge pas uniformément sur $\mathbb R_+$.
\item Montrer que la série $\sum_{n=1}^{+\infty}(-1)^n u_n$ converge uniformément sur $\mathbb R_+$.
\item Montrer que la série $\sum_{n=1}^{+\infty}(-1)^n u_n$ converge normalement sur tout intervalle $[0,A]$,
avec $A>0$.
\item Montrer que la série $\sum_{n=1}^{+\infty}(-1)^n u_n$ ne converge pas normalement sur $\mathbb R_+$.
\end{enumerate}


\Exercice[Série alternée]
On considère la série de fonctions $S(x)=\sum_{n=1}^{+\infty}\frac{(-1)^n}{x+n}$. 
\begin{enumerate}
\item Prouver que $S$ est définie sur $I=]-1,+\infty[$.
\item Prouver que $S$ est continue sur $I$.
\item Prouver que $S$ est dérivable sur $I$, calculer sa dérivée et en déduire que $S$ est croissante sur $I$.
\item Quelle est la limite de $S$ en $-1$? en $+\infty$?
\end{enumerate}



\Exercice[Convergence normale]
Soit $\zeta(x)=\sum_{n=1}^\infty \frac{1}{n^x}$.\\
Démontrer que la fonction $zeta$ converge normalement sur tout intervalle $[a,\infty[$ avec $a>1$.\\
En déduire la continuité de la fonction  $\zeta$ sur $]1,\infty[$.\\
Donner la limite $\lim\limits_{x\to \infty}\zeta(x)$.
 
 
 \Exercice[Equivalent Série-Intégrale]
On considère la série de fonctions $\sum f_n$, où pour tout $n>0$, $\Fonction{f_n}{\mathbb{R}^{+*}}{\mathbb{R}}{x}{\frac{1}{\sh (nx)}}$. 
\begin{enumerate}
\item Donner le domaine de définition de la fonction $f(x)=\sum_{n=0}^{\infty} f_n(x)$.
\item Démontrer que la fonction $f$ est continue sur $]0,+\infty[$.
\item Donner un équivalent de $0$ de $f$. 
\end{enumerate}

%page 113 H-prépa analyse


 \Exercice[Convergence suite de fonctions]
On considère la suite de fonctions $(f_n)$, où pour tout $n>0$, $\Fonction{f_n}{[0,1]}{\mathbb{R}}{x}{x^n(1-x)^n}$.  
\begin{enumerate}
\item Donner le domaine de définition de la fonction $f(x)=\lim_{n\to\infty} f_n(x)$.
\item Démontrer que la suite de fonctions $f_n$ converge uniformément sur $[0,1]$.
\end{enumerate}



 \Exercice[Convergence suite de fonctions]
On considère la suite de fonctions $\sum f_n$, où pour tout $n>1$, $\Fonction{f_n}{[0,1]}{\mathbb{R}}{x}{\frac{xe^{-nx}}{\ln n}}$.  
\begin{enumerate}
\item Donner le domaine de définition de la fonction $f(x)=\sum_{n=2}^\infty f_n(x)$.
\item Démontrer que la fonction $f$ est continue sur $\mathbb{R}^{+*}$.
\item Démontrer que la fonction $f$ est continue sur $\mathbb{R}^{+}$.
\end{enumerate}


 \Exercice[Espérance et variance de loi géométrique]
Soit $X$, une variable aléatoire suivant une loi géométrique de paramètre p, $\mathcal{G}(p)$,c'est à dire  $$\forall k \in \mathbb{N}^*:\mathbb{P}(X=k)=(1-p)^{k-1}p.$$
\begin{enumerate}
\item Calculer l'espérance de $X$
\item Calculer la variance de $X$
\end{enumerate}

\end{document}
