\begin{Soln}{1}
Soit $(u_n)$ une suite convergeant vers $l \in \R$. Par
d\'efinition
$$\forall \epsilon > 0 \quad \exists N \in \N \quad  \forall n\geqslant N \qquad |u_n-\ell| < \epsilon.$$
Choisissons $\epsilon = 1$, nous obtenons le  $N$ correspondant.
Alors pour $n\geqslant N$, nous avons $|u_n-\ell| < 1$ ;
autrement dit $\ell -1 <
u_n < \ell + 1$. Notons $M = \max_{n=0,\ldots,N-1}  \{u_n\}$  et
puis $ M' = \max (M,\ell+1)$. Alors  pour tout $n \in \N$ $u_n
\leq M'$. De m\^eme en posant $m = \min_{n=0,\ldots,N-1} \{u_n\}$ et
$m' = \min(m,\ell -1)$ nous obtenons pour tout $n\in \N$, $u_n
\geq m'$.
\end{Soln}
